\chapter{Das Bachelorstudium}
\section{Zusammenfassung}
Im Zuge der Harmonisierung des europäischen Hochschulraums, auch unter "`Bologna-Prozess"' bekannt, verstehen die Hochschulpolitikerinnen und Hochschulpolitiker in Deutschland vor allem die flächendeckende Umstellung der Studiengänge auf konsekutive Bachelor- und Masterstudiengänge. Bachelor-Absolventinnen und Absolventen können sich anschließend für einen weiterführenden Master-Studiengang bewerben. Der Titel eines Bachelors ist dabei ein vollwertiger akademischer Grad, der bei der Bewerbung für einen Masterplatz grundsätzliche Voraussetzung ist. 

Das Bachelor-Studium gliedert sich in:
\begin{enumerate}
 \item Eine Grundlagenphase
 \item eine Vertiefungsphase
 \item ein Modul zur Erlangung zusätzlicher Schlüsselqualifikationen \\(General Studies)
 \item und die Bachelorarbeit
\end{enumerate}
Die Studienordnung für den Bachelor ist „modularisiert“ aufgebaut. Die Regelstudienzeit beträgt 3 Jahre, also 6 Semester. Der modulare Aufbau soll eine überschaubare Strukturierung und eine bessere Schwerpunktbildung ermöglichen. Für ein erfolgreiches Studium müssen 180 Creditpoints (CP) erworben werden. Diese sammelt man durch bestandene Module, die jeweils wieder in einzelne Veranstaltungen unterteilt sind.

Ein Modul ist eine Sammlung von Veranstaltungen. Veranstaltungen sind z.B. Vorlesungen (V), Übungen (Ü), Seminare (S) oder Praktika (P). Die Veranstaltungen sind zu thematischen Modulen zusammengefasst. Erst wenn man alle Veranstaltungen zu einem Modul bestanden hat, ist das Modul abgeschlossen.

In jedem Modul müssen Studienleistungen erbracht werden, die sich auf eine oder mehrere Veranstaltung des Moduls beziehen. Studienleistungen sind zum Beispiel Protokolle, Referate, Hausarbeiten, mündliche Prüfungen oder Klausuren. Einige davon sind prüfungsrelevant, d. h. sie gehen in die Modulnote ein. Ihr könnt aus eurer Prüfungsordnung entnehmen, ob und wenn ja mit welchen Prozentsatz eine Studienleistung in die Modulnote eingeht.

Aus den Modulnoten wird schlussendlich die Bachelor"=Abschlussnote berechnet. Dabei gehen manche Modulnoten gar nicht, anderen mit einfacher und wieder andere zweifacher Gewichtung in die Bachelor"-Abschlussnote ein. Details dazu stehen ebenfalls in der Studienordnung.

Falls nun einmal der Fall eintritt, dass ihr eine prüfungsrelevante Studienleistung nicht bestanden habt, darf diese weitere zwei Mal wiederholt werden. Laut der Bachelor-Rahmenordnung wird man nach dem dritten Scheitern exmatrikuliert.

Die Module erstrecken sich über einen Zeitraum von maximal einem Jahr (2 Semester). Welches Modul man wann belegt ist nicht vorgeschrieben, wohl kann man sich aber an den Studienverlaufsplänen für den jeweiligen Studiengang orientieren. Einige Module setzen das Bestehen anderer Veranstaltungen voraus. Welche Module absolviert werden müssen, ist fest vorgeschrieben. Im weiteren Verlauf hat man einige Wahlmöglichkeiten, mit welchen Veranstaltungen man die Vertiefungsmodule auffüllt.

Die mit dem ECTS (European Credit Transfer System) ermittelten Leistungspunkte werden zur Berechnung des studentischen Arbeitsaufwands genutzt. Ein ECTS-Punkt entspricht 30 Arbeitsstunden eines durchschnittlichen Studierenden und beinhaltet sowohl die Zeit der Lehrveranstaltungen in der Uni, als auch die Zeit der Vor- und Nachbereitung zu Hause.

\section{Veranstaltungsformen}
\textbf{Vorlesungen (V)} dienen der allgemeinen Einführung in ein Thema und sind von der Kommunikation her meist relativ einseitig. Vorne wird geredet, ihr hört zu. Aber Vorlesungen sind das beste Mittel, um sich viel Stoff in kurzer Zeit gut aufbereitet reinzuziehen – zumindest bei guten Dozenten welche sich meist auch über Nachfragen und kurze Diskussionen freuen.
\\
\textbf{Seminare (S)} sind zur Vertiefung einer bestimmten Thematik gedacht und geben die Möglichkeit zur Beteiligung der Studierenden. Sie sind am ehesten mit dem "`guten alten Schulunterricht"' zu vergleichen, wobei das Gewicht deutlicher auf Referate und Gruppenarbeiten durch die Studierenden liegt. Die Anforderungen variieren zwischen den Seminaren, es besteht jedoch seit diesem Jahr keine Anwesenheitspflicht mehr! Nur noch in begründeten Ausnahmefällen - dazu zählen beispielsweise Exkursionen und Geländeübungen sowie auch Laborpraktika - können die Studierenden zur regelmäßigen Teilnahme verpflichtet werden. Solltet ihr dennoch eurer Meinung nach unberechtigt zu einer Anwesenheitskontrolle gezwungen werden, wendet euch an uns!

\textbf{Übungen (Ü)} können sehr unterschiedlich ausfallen und sind (zumindest von Studentenseite) nicht immer von Seminaren zu unterscheiden. Generell sollen die Studierenden eigenständig oder unter Anleitung Aufgaben der Übungszettel lösen, um so den zuvor in Vorlesungen thematisierten Stoff zu verinnerlichen. Teilweise finden Übungen allerdings auch in Form von Exkursionen statt (z. B. Physische Geographie I). Die Anmeldung zu den Seminaren und teilweise auch den Übungen findet über das LSF/ QISPOS-Portal statt (siehe Prüfungsanmeldungen/ Seminarplatzvergabe).

\textbf{Tutorien (TUT)} werden von Studierenden höherern Semesters angeboten und sind im Wesentlichen eine Hilfestellung zum Verständnis von Vorlesungs- oder Seminarstoff.

\textbf{Exkursionen (Exk)} werden auch \textbf{Geländetage} genannt. Hier soll das bisher eingetrichterte Wissen \enquote{am Objekt} überprüft und vertieft werden. Böse Zungen behaupten, dass die studentischen Entscheidungskriterien folgende seien: 1. wie originell ist das Exkursionsziel; 2. wo wollte ich schon immer mal Urlaub machen; 3. wie finde ich den/die ExkursionsleiterIn; 4. \enquote{wat kost der Spasss}. Je nach Andrang kann es problematisch sein, einen Platz in einer großen Exkursion zu bekommen. Was man allerdings schon zu Beginn seines Studiums beachten sollte, sind die Kosten, die durch Exkursionen entstehen. Bei längeren Exkursionen kommen, je nach Ziel, 500 bis 800 € Grundkosten auf euch zu, die es zu stemmen gilt. Man sollte sich also dementsprechend einen gewissen Betrag zur Seite legen.

Von Exkursionen erfährt man außer aus dem Vorlesungsverzeichnis häufig über Aushänge und die Anmeldung erfolgt in vielen Fällen über einen Listeneintrag bei dem/der zuständigen DozentIn.

\section{Seminarplatzvergabe/Prüfungsanmeldungen}

Die Anmeldung zu Veranstaltungen verläuft grundsätzlich über \textbf{HISLSF}, welches einen Überblick über sämtliche Vorlesungen und Seminare, die in dem jeweiligen Semester angeboten werden, bietet. Es handelt sich hierbei sozusagen um das \textbf{digitale Vorlesungsverzeichnis} mit allen wichtigen Infos zu den Veranstaltungen. Über das HISLSF meldet man sich für belegungspflichtige Veranstaltungen an. Das sind in der Regel Übungen und Seminare. Für Vorlesungen muss man sich meistens nicht anmelden. Die Anmeldephase bei HISLSF unterscheidet sich von der QISPOS-Anmeldephase und liegt meist am Ende des vorherigen Semesters. Informiert euch zudem über eventuelle \textbf{weitere Anmeldungsformalitäten}, da teilweise instituts- bzw. dozentenabhängig weitere Vorgaben gelten!

Zudem dient eine weitere Rubrik dieses Portals, das so genannte \textbf{QISPOS}, der digitalen \textbf{Prüfungsverwaltung und -anmeldung}. Hierfür wird in jedem Semester über einen gewissen Zeitraum hinweg das System für Anmeldungen geöffnet. Dieser Zeitraum beginnt meist einige Wochen nach Semesterbeginn. Versäumt ihr eine Anmeldung in diesem Zeitraum auf dem Portal, ist eine Teilnahme an der jeweiligen Prüfungsleistung nicht möglich. Seit dem vergangenen SoSe besteht auch nicht mehr die Möglichkeit, dass der/die betreffende DozentIn euch nachträglich anmeldet, sodass eine pünktliche Prüfungsanmeldung in QISPOS unbedingt nötig ist. Ihr findet dieses Prüfungssystem unter folgendem Link: \textbf{\url{https://studium.uni-muenster.de/qisserver/}}. Ebenfalls werden auf dieser Internetseite die genauen Daten der Anmeldephase bekannt gegeben.

\section*{Die Computerräume (StudLabs) und das ZDM}
Die Auseinandersetzung mit der harten Computerwirklichkeit bleibt niemandem erspart, denn ein gewisses Grundwissen über den Umgang mit Computern wird inzwischen eigentlich überall vorausgesetzt. Unsere Lehreinheit (also IfG, ILÖK, IfDG, IfGI) besitzt drei StudLabs und einen Medienraum (ZDM). Die drei StudLabs findet ihr im ersten Stock des Geo-Gebäudes (Heisenbergstr. 2) in den Räumen 125, 126 und 130. Alle StudLabs sind grundsätzlich allen Studierenden frei zugänglich. Bevor man eine StudLab betritt, sollte man auf dem an der Tür klebenden Belegungsplan nachschauen, ob der Raum auch wirklich frei ist. Dozenten schauen schon ein wenig verärgert, wenn der 20igste Studi innerhalb von 15 Minuten das Seminar stört.

Um an den Computern im StudLab arbeiten zu können, braucht ihr eure Uni-Kennung, die gleichzeitig eure Zugangsberechtigung ist (\url{m_must01} und euer dazugehöriges Passwort). Ihr bekommt mit euren Studienbescheinigungen und dem Semesterticket einen Abschnitt geliefert, in dem eure Uni-Kennung vorgegeben wurde. Auch euer erstes Zugangspasswort wird euch mit den Studienunterlagen zugeschickt. Dazu erhaltet ihr eine zur Kennung passende Emailadresse \url{m_must01@uni-muenster.de}. Über diese Emailadresse seid ihr dann für alle Uni-Angelegenheiten (und natürlich auch privat) gut zu erreichen. Da man sich aber weder die vom Computer für euch generierte Emailadresse noch das dazu gehörige Passwort merken kann, ist es ratsam, auf der Homepage des Rechenzentrums (\url{http://www.uni-muenster.de/ZIV}) eine Alias-Emailadresse einzurichten (z.B. \url{max.mustermann@uni-muenster.de}) und das Passwort so für sich zu ändern, dass man es sich merken kann. Wie das funktioniert findet ihr auf der Rechenzentrumsseite detailliert erklärt.\\ \textbf{BEACHTE}: Der Mann hinter der Glasscheibe im Rechenzentrum ist ziemlich gereizt, wenn der 50igste Studi am Tag sein Passwort vergessen hat und einen neuen Zugang haben will\dots \mbox{ }Er wird wirklich böse\dots \\
\textbf{ACHTUNG}: Die Uni benutzt eure Uni-Emailadressen, um euch Informationen zukommen zu lassen, die unter Umständen sehr wichtig sein können! Wenn ihr lieber nur euren alten E-Mail-Account (GMX, WEB, etc.) weiternutzen wollt, dann stellt eine Weiterleitung ein. Ihr bekommt dann die Mails direkt in euer altes Postfach weitergeleitet und verpasst nix. Auch dies ist schnell und einfach unter \url{http://www.uni-muenster.de/ZIV} zu erledigen. Es ist extrem wichtig, dass ihr unter eurer Uniadresse erreichbar seit, also testet bitte, ob die Weiterleitung wirklich funktioniert und beachtet, dass das Postfach eurer normalen Emailadresse eventuell nicht genug Speicherplatz hat. Solltet ihr, aus welchen Gründen auch immer, keine E-Mail Adresse zugewiesen bekommen haben, dann könnt ihr sie im Zentrum für Informationsverarbeitung (Unirechenzentrum) in der Einsteinstraße 60 beantragen.

Im ZDM (Zentrum für Digitale Medien, Heisenbergstr. 2, Raum 11) braucht ihr auch eure Kennung mit Passwort, um an den Computern zu arbeiten. Im Vergleich zu den StudLabs kann man im ZDM scannen, brennen, (bunt-)drucken, schneiden etc. – echt praktisch! Ihr arbeitet dort mit den neuesten Programmen und aktueller Technik und solltet auf jeden Fall mal dort vorbeischauen. Die Öffnungszeiten und Serviceangebote des ZDM findet ihr unter \url{http://zdmweb.uni-muenster.de/}.
