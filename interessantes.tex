\chapter{Interessantes}

\section{HowTo--Altklausur}
Altklausuren sind, wie der Name schon sagt, alte Klausuren, die Dozenten in ihren Veranstaltungen schon einmal gestellt haben. Nützlich sind sie deswegen, weil man mit ihnen eine Orientierung hat, welcher Stoff aus der Vorlesung wie intensiv abgefragt werden könnte und weil ihr euer bereits angehäuftes Wissen überprüfen könnt.\\
Da Dozenten sich aber nicht ständig neue Fragen ausdenken möchten, geben sie diese Altklausuren ungern heraus und es ist entsprechend schwierig, an Altklausuren zu kommen. Deswegen hier einige Ideen, wie man trotzdem an die Klausuren kommt - euren Nachfolgern tut ihr dadurch auf jeden Fall einen großen Gefallen, so wie eure Vorgänger euch einen Gefallen getan haben und dessen Vorgänger ihren Nachfolgern usw.\\
Dazu gibt es fünf Möglichkeiten:\\
\textbf{1. Nachfragen:}\\
Ausnahmen bestätigen die Regel und dem ein oder anderen Dozenten liegt der Lernerfolg kommender Generationen so sehr am Herzen, dass er doch ein Klausurexemplar herausrückt, wenn ihr ihn nett darum bittet. Es gilt die Devise: Fragen kostet ja nichts!\\
\\
\textbf{2. Kopieren:}\\
Die sauberste und am wenigsten aufwendigste Methode. Ihr habt bei jeder Klausur bis zu 2 Wochen nach Bekanntgabe der Ergebnisse Recht auf Einsicht in eure Klausur, meist an einem festen Termin, auf dessen Bekanntgabe ihr achten solltet! Lässt euch der Dozent mit der Klausur aus dem Raum, heißt es schnell zum Kopierer und die Klausur kopieren. Den Namen und die Matrikelnummer später unkenntlich machen. Das gilt auch für eure Antworten, wenn ihr nicht wollt, dass sie andere lesen. Und dann ab damit in die Fachschaft. Die Original-Klausur müsst ihr natürlich beim Dozenten wieder abgeben. Meistens lassen euch die Dozenten aber nicht aus den Augen, weshalb diese Strategie nur selten klappt.\\
\\
\textbf{3. Fotografieren:}\\
Jedes Handy hat mittlerweile eine eingebaute Kamera, mit der ihr während der Klausureinsicht in einem günstigen Moment unbemerkt die Klausurblätter fotografieren könnt. Wenn der Dozent dazwischen kommt, ist es nicht schlimm, wenn ihr auch nur einen Teil der Klausur abfotografieren konntet. Die Klausur muss aber noch nachbearbeitet werden, weil diese Fotos eine schlechte Qualität haben und ihr vermutlich eure Daten unkenntlich machen wollt.\\
\\
\textbf{4. Abschreiben:}\\
Natürlich nicht von eurem Nebenmann... Aber Unerschrockene können die Klausur während der Klausur auf einen Zettel abschreiben und diesen nachher zur Fachschaft bringen. Bequemere Variante für Harte: mit der Klausur zum Kopierer. Außerdem habt ihr vielleicht Gelegenheit, ein ungenutztes Klausurexemplar aus dem Hörsaal zu entführen, worüber sich die Umwelt freut und keine zusätzlichen Kopierkosten anfallen ^^
\\
\textbf{5. Gedächtnis abrufen:}\\
Die nächste Methode ist die aufwendigste, klappt dafür aber immer, egal wie penibel die Dozenten darauf achten, dass die Klausuren auch ja nicht mitgenommen werden. Organisiert euch dazu einfach in einer kleinen oder großen Runde direkt nach der Klausur und schreibt die gestellten Fragen auf. Versucht euch möglichst genau an die Antwortmöglichkeiten bei Multiple"=Choice"=Fragen zu erinnern. Diese Art der Altklausurbeschaffung kann natürlich auch über die Email-Liste des Kurses gehen. Ist dieses Gedächtnisprotokoll erstellt, dann schickt es einfach der Fachschaft. 
Die letzte Methode kann auch euch noch mal helfen wenn ihr euch nicht sicher seid ob ihr bestanden habt. Dann habt ihr schon einige Fragen, mit denen ihr euch auf die Nachschreibklausur vorbereiten könnt. 

\section{Erasmus-Erfahrungsbericht Salamanca 2008/2009}

Im ersten Semester ist es zwar vielleicht noch etwas früh, sich mit einem Auslandssemester zu beschäftigen, aber um schon einen ersten Eindruck zu gewinnen hier ein Erfahrungsbericht zum schmökern:\\\\
Um das wichtigste vorweg zu nehmen: Salamanca war, meiner Ansicht nach, eine absolut perfekte Wahl für ein Erasmus-Jahr in Spanien und von allen Städten, die ich dort so gesehen habe, gab es (außer vielleicht Granada) keine, in der ich lieber ein Jahr verbracht hätte. Aber was macht Salamanca so genial für ein Erasmus-Jahr? \par
\textbf{Die Stadt} – Salamanca ist eine relativ kleine Stadt zwischen Madrid und der Grenze zu Portugal in der Meseta Castilla y Leons. Von den 150.000 Einwohnern sind etwa 30.000 Studenten, wobei 7.000 davon aus dem Ausland stammen (Sprachkursler eingeschlossen). Es besitzt eine historische Altstadt, die zum Weltkulturerbe der UNESCO erklärt wurde und dies auch absolut verdient hat. Mehrere beeindruckende Kathedralen, Kirchen und Konvente, wunderschöne alte Fassaden und nicht zuletzt die Plaza Mayor, die zurecht als eine der schönsten Spaniens gilt. Das ganze ist auch nicht nur eine Art Freilichtmuseum, sondern es spielt sich ein Großteil des alltäglichen Lebens hier ab. Im Süden der Altstadt schlängelt sich der Río Tormes, ein recht hübsch anzusehender, aber leider nicht „bebadbarer“ Fluss, an dessen Ufer die Luft im Sommer allerdings deutlich erfrischender und nicht ganz so heiß ist als in der Stadt selbst. Generell kann man fast alle Entfernungen problemlos zu Fuß zurücklegen, Fahrräder gibt’s eher sporadisch und man muss ziemlich auf die spanischen Autofahrer aufpassen, die es einfach nicht gewöhnt sind, Fahrräder auf der Straße zu haben. \\
\textbf{Die Leute} – Die Charros (so nennen sich die Bewohner Salamancas) sprechen im Allgemeinen kein Englisch! Das ist ein unschätzbarer Vorteil für jeden Erasmus-Studenten, der wirklich Spanisch lernen möchte, denn man kommt gar nicht erst in die Versuchung, immer wieder auf sein Englisch zurückzugreifen. Zugegebenermaßen leidet man mit seinen gebrochenen Sprachkünsten am Anfang etwas, aber sobald man die ersten paar Telefonate zur Wohnungssuche auf Spanisch hinter sich gebracht hat, schockt einen eh nichts mehr. Außerdem gehört das zu einem Erasmus-Jahr einfach dazu, auch mal aufgeschmissen zu sein!\par
\textbf{Die Uni} – Die Universidad de Salamanca ist eine der ältesten Spaniens und genießt innerhalb der spanischen Uni-Landschaft einen relativ guten Ruf. Wie im einzelnen das Niveau eurer Kurse ist, hängt jedoch sehr stark von den Professoren ab. Was ausfällt, ist die relativ geringe Kursstärke (ich hatte einen Kurs mit 4 Leuten) und der nahe Kontakt zu den Professoren. Es gibt zwar Tutorias (Sprechstunden), aber im Endeffekt schickt euch auch niemand weg, wenn ihr außerhalb der angegebenen Zeiten da seid und nicht gerade Kaffeepause gemacht werden muss. Die Geo-Kurse finden an der Facultad de Geografía e Historia in einem alten, restaurierten Gebäude mitten in der Altstadt statt. Gerade im Vergleich zur PBC-belasteten Container-Fakultät in Münster ist das wirklich nett. Die Bibliothek hat leider einen relativ begrenzten Bestand an Literatur und speziell neuere Bücher muss man gar nicht erst suchen gehen.\par
\textbf{Das Flair} – Salamanca ist zwar eine kleine, aber eine unglaublich lebendige Stadt. Es gibt sehr viele Bars, Cafés, Clubs, etc. etc. Die Preise sind recht niedrig und eigentlich ist immer was los, egal welcher Wochentag ist, nur in der Klausurenperiode etwas weniger, aber ist auch ganz gut so! Dazu gibt’s noch immer mal kleinere Festivals und Traditionen zu den diversen Feiertagen (Nochevieja Universitaria, Lunes de Aguas, etc.), sodass einem wirklich nicht langweilig werden kann. Jede Fakultät hat außerdem noch ihren eigenen Feiertag, weshalb eigentlich ständig irgendwer verkleidet durch die Altstadt läuft. Überhaupt bestimmen die Studenten das Bild auf der Straße, was der ganzen Stadt eine relativ junge und leicht verrückte Atmosphäre verleit.
Alles in allem bin ich wirklich absolut begeistert von der Stadt und den Möglichkeiten, die sie einem als Erasmus-Studenten bietet. Von daher kann ich jedem, der an ein Erasmus-Aufenthalt in Spanien denkt, Salamanca als Ziel absolut empfehlen! \\
\\
Für Fragen rund um das ERASMUS/SOKRATES Programm steht euch das Erasmus-Büro zu Verfügung. Hier bekommt ihr individuelle Beratung und alles Wissenswerte zum Bewerbungsverfahren.\\ \\
\textbf{Adriana Kirchner und Jan Winkin}\\
\textbf{Erasmus-Büro}   \url{erasmus@uni-muenster.de}	Tel.:\,83--33988\\ 

\section{Geld zu verschenken – Stipendienmöglichkeiten für Studis}
Miete, Internet und Handyrechnungen, außerdem Bücher, Exkursionen, Kaffe, Tiefkühlpizza – im Laufe eures Studiums kommen so einige Ausgaben auf euch zu. Wenn dann noch das eine oder andere ,Erfrischungsgetränk' zu sich genommen wird, herrscht auch schnell mal Ebbe im Portemonnaie. Um den Tidenhub wieder in vernünftige Bahnen zu bringen, gibt es ziemlich viele Möglichkeiten: Mama und Papa anpumpen, jobben gehen, Bafög beantragen, etc. Eine ergänzende Alternative ist ein Stipendium. Das klingt soweit super, aber was ist das, wer fördert eigentlich und wie kommt man da ran?
Stipendien können sehr verschieden aussehen. Manche Stiftungen unterstützten ‚nur’ finanziell, andere legen großen Wert auf Teilnahme an Seminaren oder Workshops. Auch die Bemessungsgrundlage der finanziellen Förderung variiert stark. Meist wird ein Betrag von \unit[60-100]{\officialeuro} pro Monat als so genanntes Büchergeld unabhängig von der finanziellen Situation des Stipendiaten ausgezahlt, eine etwaige weitergehende Förderung orientiert sich oftmals an den Bafög-Kriterien.
Die bekanntesten Stipendiengeber sind sicherlich die zwölf ‚großen’ Begabtenförderungswerke:
\begin{itemize}
 \item Cusanuswerk
 \item Deutschland-Stipendium
 \item Evangelisches Studienwerk e. V. Villigst
 \item Friedrich-Ebert-Stiftung
 \item Friedrich-Naumann-Stiftung
 \item Hans-Seidel-Stiftung
 \item Hans-Böckler-Stiftung
 \item Heinrich-Böll-Stiftung
 \item Konrad-Adenauer-Stiftung
 \item Rosa-Luxemburg-Stiftung
 \item Stiftung der Deutschen Wirtschaft
 \item Studienstiftung des Deutschen Volkes 
\end{itemize}
Diese vergeben die größte Anzahl Stipendien, verzeichnen allerdings auch die höchsten Bewerberzahlen. Daher ist häufig eine Bewerbung bei einer der unzähligen kleineren Stiftungen aussichtsreicher. Einige der Stiftungen haben aufgrund ihres geringen Bekanntheitsgrades sogar so wenige Bewerbungen, dass nicht alle Stiftungsmittel ausgezahlt werden. Zum Teil sind die Förderungen regional begrenzt oder nur für bestimmte Fachrichtungen offen, daher kann es eine ganze Weile dauern bis man eine passende Stiftung gefunden hat. Um die Suche ein wenig abzukürzen, hat e-fellows.net, selbst ein Stipendiengeber, eine Datenbank erstellt, in der viele Stiftungen mit einer Kurzbeschreibung verzeichnet sind. Das ganze finde ihr unter folgendem Link:\\ 
\\
\url{http://www.e-fellows.net/show/detail.php/5789}\\
\\
Die genauen Anforderungen, Bewerbungsabläufe und Fristen sind sehr unterschiedlich und es kann eine ganze Weile dauern, bis man alle erforderlichen Unterlagen beieinander hat. Zudem ist eine Bewerbung teilweise nur bis zum 2. oder 3. Fachsemester möglich. Man sollte daher relativ frühzeitig mit der Suche beginnen, um nicht irgendwelche Fristen zu verpassen. Ganz wichtig ist auch, sich nicht abschrecken zu lassen durch die Anforderungen der Stiftungen oder den Aufwand, der mit einer Bewerbung verbunden ist. Sollte am Ende eine Zusage dabei herauskommen, hat sich der Aufwand in jedem Fall gelohnt und immer dran denken: Die Anderen kochen auch nur mit Wasser!

\section{Die Hilfskraft – Studenticus helpissimus}
Wer kennt sie nicht, die ominöse „Hilfskraft“?!? Sie begegnet einem in Seminaren, im Vorzimmer des Professors, an der Kaffeemaschine, in der Bibliothek oder am Kopierer. Aber was macht so eine Hilfskraft eigentlich wirklich und wieso steht sie dann doch manchmal an der Kaffeemaschine anstatt schlaue Bücher mit zu verfassen? Also in erster Linie geht es aus unserer Sicht darum, ein bisschen Geld zu verdienen. Je nachdem, wie viele Stunden pro Woche ableistet werden, kann eine SH (Studentische Hilfskraft) bis zu 350 Euro im Monat dazu verdienen. Das hört sich gut an, aber der Weg zum Geld kann doch recht steinig sein, vor allem sollte man nicht unterschätzen, dass man diese Zeit bei Frei- und Studienzeit abziehen muss. Außer dem Geld bekommt man aber auch spannende Einblicke in den Lehr- und Forschungsbetrieb an der Uni, je nachdem, um was für eine Stelle am sich bemüht hat.\par
Hilfskräfte gibt es an verschiedenen Orten: in den Arbeitsgruppen der Professorinnen und Professoren (Kopieren, Recherchieren, Korrektur"=Lesen, Schreibarbeit, etc.), in der Bibliothek oder im ZDM (Aufsicht, Hilfe, …) usw. Es empfiehlt sich zumeist, bereits ein bisschen Uni-Luft geschnuppert zu haben, also vielleicht nicht gleich im ersten Studienjahr anzufangen. Die Verfügbarkeit offener Stellen ist sehr unterschiedlich. Während beispielsweise die Geoinformatiker recht viele freie Hilfskraftstellen anbieten können, die dann über die Monitore flackern oder am Schwarzen Brett rumhängen, so muss man sich in den meisten Fällen selbst darum kümmern, indem man einfach nachfragt und seine Arbeitskraft anbietet. Am besten dort, wo einen das Thema oder die Arbeit auch ein bisschen interessiert. Und hier lautet die Devise, nicht enttäuscht zu sein, wenn es nicht sofort klappt, aber vielleicht kommt die- oder derjenige ja darauf zurück. Also erstmal eine Anfrage starten und Namen und Adresse da lassen, häufig wird dann später doch was draus. 
\par
Wie man letztendlich eingesetzt wird, hängt dann ganz vom Chef und den anstehenden Arbeiten ab. Tendenziell wachsen die Aufgaben mit dem fachlichen Wissen der Hilfskraft, was aber nicht bedeutet, dass jede Hilfskraft als Tellerwäscher anfängt. Zurzeit bekommt man rund acht Euro pro Stunde, wobei hier Vorsicht geboten ist, denn bei so manchen Chefs ist „eine Stunde Verdienst“ schnell auch mal zu mehr als zwei Stunden Arbeit geworden. Ob man das dauerhaft mit sich machen lässt, ist einem selbst überlassen. Zumeist sind die Arbeitsbedingungen allerdings wirklich fair und es herrscht eine gute Stimmung. Man darf den Kaffee also auch mittrinken, was die Motivation ihn zu kochen doch deutlich steigert, oder? \par
Letztendlich sind Hilfskraftstellen für alle diejenigen zu empfehlen, die gerne etwas tiefer in die Forschungsarbeit oder die Lehre mit einsteigen wollen und die Uni nicht nur als Lernanstalt sehen. Zugegebenermaßen sind nicht alle Stellen anspruchsvoll, aber zum Geldverdienen taugen sie schon. Und wenn man feststellt, man mag die Arbeit, ergibt sich daraus vielleicht auch nach dem Studium noch eine berufliche Option, denn viele Doktoranden haben sich in ihrer Studienzeit bereits als Hilfskraft verdient gemacht. Sobald man übrigens einen fertigen Abschluss hat, kann man bereits als WH (Wissenschaftliche Hilfskraft) eingestellt. Das gibt dann einige Euro mehr pro Stunde. 

\newpage

\section{Ein Tag meines Lebens als Student}
(unbekannter Autor, aber war nach kleiner Bearbeitung sehr passend und durfte deshalb nicht fehlen)\\
\begin{verse}
\textbf{1. Semester}\\ 
%\\
05:30 Der Quarz-Uhr-Timer mit Digitalanzeige gibt ein zaghaftes "`Piep-Piep"' von sich. Bevor sich dieses zu energischem Gezwitscher entwickelt, sofort ausgemacht \linebreak und aus dem Bett gehüpft. Um die Promenade gejoggt, mit einem Besoffenen zusammengestoßen, anschließend eiskalt geduscht. \\
%\\
06:00 Beim Frühstück Greenpeace Magazin auswendig gelernt und Umweltpolitik der Grünen analysiert. Danach kritischer Blick in den Spiegel, Outfit genehmigt.\\ 
%\\
07:00 Zur Uni gehetzt. Hörsaal erreicht. Pech gehabt: erste Reihe schon besetzt. Niederschmetternd. Beschlossen, morgen doch noch eher aufzustehen.\\ 
%\\
07:30 Vorlesung. Keine Disziplin! Einige Kommilitonen lesen Sportteil der Zeitung oder gehen zum Frühstücken. Alles mitgeschrieben; Füller leer, aber über die Witzchen des Dozenten mitgelacht.\\ 
%\\
08:00 Vorlesung. Verdammt! Extra neongrünen Pulli angezogen und trotz eifrigem Fingerschnippens nicht drangekommen.\\
%\\
10:45 Nächste Vorlesung. Nachbar verlässt mit Bemerkung "`Sinnlose Veranstaltung"' den Raum. Habe mich für ihn beim Prof. entschuldigt.\\ 
%\\
12:00 Mensa Stammessen II. Nur unter größten Schwierigkeiten weitergearbeitet, da in der Mensa zu laut.\\ 
%\\
12:45 In Fachschaft gewesen. Skript immer noch nicht fertig. Wollte mich beim Vorgesetzten beschweren. Keinen Termin bekommen. Daran geht die Welt zugrunde.\\ 
%\\
13:00 Fünf Leute aus meiner 0-Gruppe getroffen. Gleich für drei AG's zur Klausurvorbereitung verabredet.\\
%\\
13:30 Dreiviertelstunde im Copyshop gewesen und die Klausuren der letzten 10 Jahre mit Lösungen kopiert. Dann Tutorium: Ältere Semester haben keine Ahnung.\\ 
%\\
15:30 In der Bibliothek mit den anderen gewesen. Durfte aber statt der dringend benötigen 18 Bücher nur vier mitnehmen.\\ 
%\\
16:00 Proseminar. War gut vorbereitet. Hinterher den Assi über seine Irrtümer aufgeklärt.\\ 
%\\
18:30 Anhand einschlägiger Quellen die Promotionsbedingungen eingesehen und erste Kontakte geknüpft.\\ 
%\\
19:45 Abendessen. Verabredung im "`Blauen Haus"' abgesagt. Dafür Vorlesungen der letzten paar Tage nachgearbeitet.\\ 
%\\
23:00 Videoaufzeichnung von Terra Nova angesehen und im Bett noch den Campbell gelesen. Festgestellt, 18\textminus{}Stunden\textminus{}Tag zu kurz. Werde demnächst die Nacht hinzunehmen.                                                                                                                                                                      \end{verse}

\newpage

\begin{verse}
\textbf{ 13. Semester }\\
10.30 Aufgewacht! Kopfschmerz. Übelkeit. Zu deutsch: KATER.\\ 
%\\
10.45 Der linke große Zeh wird Freiwilliger bei der Zimmertemperaturprüfung. (arrgh!) Zeh zurück. Rechts Wand, links kalt; Ich bin gefangen.\\ 
%\\
11.00 Kampf gegen den inneren Schweinehund: Aufstehen oder nicht -- das ist hier die Frage.\\ 
%\\
11.30 Schweinehund schwer angeschlagen, wende Verzögerungstaktik an und schalte Fernseher ein\\ 
%(inzwischen auch schon verkabelt).\\ 
%\\
12.05 Mittagsmagazin beginnt. Originalton Moderator: "Guten Tag liebe Zuschauer Guten Morgen liebe Studenten." - Auf die Provokation hereingefallen und aufgestanden.\\ 
%\\
13.30 In der Cafeteria der Mensa am Aasee beim Skat mein Mittagessen verspielt.\\ 
%\\
14.30 Im Gasolin hereingeschaut. Direkt Geld gepumpt und 'ne Kleinigkeit gegessen: Das Bier schmeckt auch wieder! Kurze Diskussion mit ein paar Leuten über die letzte Entwicklung des Dollar-Kurses und der Weltpolitik.\\ 
%\\
15.45 Kurz in der Bibliothek gewesen. Nur weg hier, total von Erstsemestern überfüllt.\\ 
%\\
16.00 Fünf Minuten im Tech gewesen. Nichts los! Keine Zeitung, keine Flugblätter - nichts wie raus.\\ 
%\\
%17.00 Stammkneipe hat immer noch nicht geöffnet.\\ 
%\\
18.15 Wichtiger Termin zuhause: Star Trek!\\ 
%\\
18:20 Mist! Kein Star Trek! Stattdessen Live-Übertragung von Stöhn-Seles. SAT 1 war auch schon besser...\\ 
%\\
19.10 Komme zu spät zum Date mit der hübschen blonden Erstsemesterin im Barzillus. Immer dieser Stress!\\ 
%\\
01.00 Die Kneipen schließen auch schon immer früher... Umzug in's Amp.\\ 
%\\
04.20 Tagespensum erfüllt. Das Bett lockt.\\ 
%\\
05.35 Auf der Promenade von Erstsemester über'n Haufen gerannt worden. Hat mich gemein beschimpft.\\ 
%\\
06.45 Bude mühevoll erreicht. Insgesamt \unit[15]{\officialeuro}  ausgegeben. Mehr hatte die Kleine nicht dabei.\\ 
%\\
07.05 Ich schlucke schnell noch ein paar Alkas und schalte kurz das Radio ein. Stimme des Sprechers: "Guten Morgen liebe Zuhörer, gute Nacht liebe Studenten."\\
\end{verse}

\newpage
\section{Gängige Abkürzungen im Uni-Dschungel}

\begin{longtable}{p{0.2\textwidth} p{0.7\textwidth}}
  \textbf{ALsA} & Ausschuss für Lehre und studentische Angelegenheiten\\
  \textbf{AOR} & Akademischer Oberrat (der „Mittelbau“)\\
  \textbf{AR} & Akadiemischer Rat\\
  \textbf{AStA} & Allgemeiner Studierendenausschuss\\
  \textbf{ASV} & Ausländische Studierendenvertretung\\
  \textbf{AVZ} & Allgemeines Verfügungszentrum\\
  \textbf{Ba / BSc} & Bachelor / Bachelor of Science\\
  \textbf{Bib} & Bibliothek\\
  \textbf{CIP--Pool} & Computer Investitions Programm = Computerraum\\
  \textbf{c.t.} & cum tempore = akademisches Viertel (\unit[9]{Uhr} c.t. = \unit[9:15]{Uhr})\\
  \textbf{Dipl.} & Diplom\\
  \textbf{DSW} & Deutsches Studentenwerk\\
  \textbf{ECTS} & European Credit Transfer System\\
  \textbf{EGEA} & European Geographic Association\\
  \textbf{ERASMUS} & Europäisches Austauschprogramm\\
  \textbf{ESG} & Evangelische Studierendengemeinde\\
  \textbf{Exk.} & Exkursion\\
  \textbf{FB} & Fachbereich\\
  \textbf{FBR} & Fachbereichsrat\\
  \textbf{FH} & Fachhochschule\\
  \textbf{FK} & Fachschaftenkonferenz\\
  \textbf{FO} & Frontoffice\\
  \textbf{FP} & Fachprüfung (meistens mündlich, kann schriftlich sein)\\
  \textbf{FS} & Fachschaft \tiny{(oder Fachsemester)}\\ 
  \textbf{FSR} & Fachschaftsrat\\
  \textbf{FSV} & Fachschaftsvertretung\\
  \textbf{fsz} & Freier Zusammenschluss von Studierenden (\underline{der} Dachverband)\\       
  \textbf{GeLaGe} & Geographisch-Landschaftsökologische Gemeinschaftsliste\\ 
  \textbf{GelPr.} & Geländepraktikum\\
  \textbf{Geofs} & Fachschaft Geoinformatik\\
  \textbf{Geogr.} & Geographie oder Geograph/Geographin\\
  \textbf{GeoLök} & Fachschaft Geographie (Diplom, Lehramt, Magister) u. Landschaftsökologie\\
  \textbf{GHR} & Grund-, Haupt-, Realschule\\
  \textbf{GHS} & Geländehauptseminar\\
  \textbf{GPI} & Geologisch-Paläontologisches Institut (Corrensstr.)\\
  \textbf{GS} & Grundseminar oder Grundstudium\\ 
  \textbf{Gym/Ges} & Gymnasium/ Gesamtschule\\
  \textbf{HD} & Hochschuldozent\\
  \textbf{HISLSF} & siehe LSF\\
  \textbf{Hiwi} & Hilfswissenschaftler; Hilfskraft\\
  \textbf{HoMaLa} & Horstmarer Landweg (Studentenwohnheime)\\
  \textbf{HRG} & Hochschulrahmengesetz\\
  \textbf{HS} & Hauptseminar oder Hauptstudium oder Hörsaal (Robert-Koch-Str.)\\
  \textbf{HSP} & Hochschulsport\\
  \textbf{IfDG} & Institut für Didaktik der Geographie (Schlossplatz)\\
  \textbf{IfG} & Institut für Geographie (Schlossplatz)\\
  \textbf{ifgi} & Institut für Geoinformatik (Weselerstr.)\\
  \textbf{ILök\,(ganz früher IfL)} & Institut für Landschaftsökologie (Robert-Koch-Str.)\\
  \textbf{IVV} & Informationsverarbeitungsversorgung\\
  \textbf{KHG} & Katholische Hochschulgemeinde (am Kardinal-von-Galen-Ring)\\
  \textbf{Kom-Voz} & Kommentiertes Vorlesungsverzeichnis\\
  \textbf{KSHG} & Katholische Studierenden- und Hochschulgemeinde (Frauenstr.)\\
  \textbf{LA} & Lehramt\\
  \textbf{LABG} & Lehrerausbildungsgesetz\\
  \textbf{LN} & Leistungsnachweis, auch "`Schein"' genannt\\
  \textbf{LPO} & Lehramtsprüfungsordnung\\
  \textbf{Lök} & Landschaftsökologie\\
  \textbf{LSF / HISLSF} & Elektronisches Vorlesungsverzeichnis der Uni Münster „Lehre, Studium, Forschung“ (s.o.)\\
  \textbf{M.A.} & Magister Artium\\
  \textbf{Ma / MSc} & Master / Master of Science\\
  \textbf{Mag-NF} & Magister-Nebenfach\\
  \textbf{MAP} & Modulabschlussprüfung\\
  \textbf{MPO} & Magisterprüfungsordnung\\
  \textbf{N.N.} & Nomen Nominandum (der Name des Dozierenden wird noch bekannt gegeben)\\
  \textbf{OE} & Orientierungseinheit\\
  \textbf{Prakt.} & Praktikum\\
  \textbf{Prof.} & Professor\\
  \textbf{Proj.} & Projekt\\
  \textbf{QisPos} & Elektronisches System für alle Bachelor zur Anmeldung und Registrierung (s.o.)\\
  \textbf{RHW} & Rudolf-Harbig-Weg (Studiwohnheime)\\
  \textbf{Sepl} & Seminarplatzvergabe (online)\\
  \textbf{S I/II} & Sekundarstufe I/II\\
  \textbf{SP / StuPa} & Studierendenparlament\\ 
  \textbf{SpSt} & Schulpraktische Studien (Büro Scharnhorststr. 100 / Platz der weißen Rose)\\
  \textbf{StuPa} & siehe SP\\
  \textbf{SoSe} & Sommersemester (bitte \textbf{keine} andere Abkürzung verwenden)\\
  \textbf{s.t.}	& sine tempore = ohne akademisches Viertel pünktlich (\unit[9]{Uhr} s.t. = \unit[9:00]{Uhr})\\
  \textbf{SWS} & Semesterwochenstunden\\
  \textbf{TN} & Teilnahmenachweis\\
  \textbf{Ü\,/\,ÜB} & Übung\\
  \textbf{UB\,/\,ULB} & Universitäts- und Landesbibliothek\\
  \textbf{Vorl.\,/\,VL} & Vorlesung\\
  \textbf{VV} & Vollversammlung oder Vorlesungsverzeichnis\\
  \textbf{WS} & Wintersemester\\
  \textbf{ZDM/MD} & Zentrum für Digitale Medien und Mediendidaktik\\
  \textbf{ZSB} & Zentrale Studienberatung (Schloßplatz)\\
\end{longtable} 

\newpage
\section{Die studentische Selbstverwaltung}
\textbf{Studierendenschaft}
\begin{itemize}
 \item alle Studierenden der Universität
 \item wählen jedes Wintersemester VertreterInnen ins SP
\end{itemize}

\textbf{Studierendenparlament (SP)}
\begin{itemize}
  \item hat 31 Mitglieder, VertreterInnen verschiedener hochschulpolitischer Listen, werden für ein Jahr gewählt
  \item wählt aus seiner Mitte den Vorsitz für das SP
  \item wählt AStA-Vorstand und die ordentlichen AStA-Referate
  \item bestätigt die autonomen Referate, die von ihrer jeweiligen Interessengruppe gewählt werden
  \item oberstes Beschlussfassendes Gremium der Studierendenschaft
\end{itemize}

\textbf{Allgemeiner Studierenden Ausschuss (AStA)}
\begin{itemize}
  \item Exekutivorgan der Studierendenschaft (wie Merkel mit ihren Bundesministerien)
  \item besteht aus Vorstand und Referaten:
    \begin{enumerate}
      \item ordentliche Referate \\ (Finanz-, Hochschulpolitik-, Wohn-, Sozial"~, Öffentlichkeits-, Ökologie-, Kultur-, Frieden/Internationalismus- Referate)
      \item autonome Referate \\ (Frauen-, Lesben-, Schwulen- und Behindertenreferat)
      \item halbautonome Referate (Fachschaftenkonferenz)
    \end{enumerate}
  \item auskunftspflichtig gegenüber dem SP
\end{itemize}

\textbf{Fachschaft}
\begin{itemize}
  \item alle Studierenden eines Fachbereiches
  \item wählen jedes Sommersemester die Fachschaftsvertretung                                                  
\end{itemize}

\textbf{Fachschaftsvertretung (FSV)}
\begin{itemize}
  \item entspricht strukturell dem SP, auch hier können verschiedene Listen zur Wahl antreten
  \item wählt eigenen Vorsitz
  \item beschließt Satzung der Fachschaft
  \item wählt den FSR
\end{itemize}

\textbf{Fachschaftsrat (FSR)} 
\begin{itemize}
  \item ausführendes Organ der Fachschaft
  \item Koordinierung der studentischen Politik am Fachbereich
  \item Veröffentlichung von Infos aus dem Fachbereich, dem universitären und dem überregionalen Bereich
  \item Engagement in den Gremien des Fachbereichs
  \item ErstsemesterInnenarbeit, Serviceleistungen
  \item schickt VertreterIn zur Fachschaftenkonferenz
\end{itemize}

\textbf{Fachschaftenkonferenz (FK)} 
\begin{itemize}
  \item Organ, in dem VertreterInnen aller Fachschaften der Universität Münster zusammenkommen
  \item Bindeglied zwischen AStA, SP auf der einen und den Fachschaften auf der anderen Seite
  \item wählt FK-ReferentIn
  \item beratende Mitglieder sind VertreterInnen aus Uni-Kommissionen und dem AStA
  \item FK-ReferentInnen sind dem AStA auskunftspflichtig und der FK rechenschaftspflichtig
\end{itemize}

\section{Gremien der Universität}

\textbf{Institutsvorstand}
\begin{itemize}
 \item Professoren eines Institutes gehören ihm automatisch an
 \item auf jeden vierten Professor  kommt ein Mitglied der anderen Statusgruppen
 \item Umsetzung der Prüfungsordnungen in Studienordnungen 
  \item studentischen VertreterInnen werden von den studentischen Mitgliedern im FBR gewählt
  \item wählt aus der Gruppe der Professoren den/ die Direktor/in
\end{itemize}

\textbf{FBR (Fachbereichsrat)}
\begin{itemize}
  \item höchstes Beschlussfassendes Gremium des Fachbereiches
  \item entscheidet in allen Belangen des Fachbereiches: Berufungen, Finanzen, Lehrangebot
  \item Vorsitz hat der/die DekanIn; wird aus der Gruppe der Profs gewählt
  \item Mitglieder werden von jeweiliger Statusgruppe gewählt
  \item hat mehrere Ausschüsse, zum Beispiel:
    \begin{itemize}
	\item \textbf{AFWN} (Ausschuss für Forschung und wissenschaftlichen \linebreak Nachwuchs)
	\item \textbf{ALsA} ( Ausschuss für Lehre und studentische Angelegenheiten)
	\item \textbf{DPA} (Diplomprüfungsausschuss)
	\item Prüfungsausschuss LA SI/II
	\item Berufungskommissionen
	\item Bachelor/Master-Ausschuss
    \end{itemize}
\end{itemize}

\textbf{Dekan}
\begin{itemize}
 \item vollzieht Promotionen, Habilitationen
  \item hat Eilkompetenz in wichtigen Angelegenheiten
  \item lädt zu den Sitzungen des Fachbereichrates ein
  \item hat Rederecht im Senat
  \item aktuelle Dekanatsregelung im Fachbereich Geowissenschaften:
      \begin{itemize}
	\item 1 Dekan
	\item 2 Prodekane: Finanzdekan und Studiendekan (den Job des Studiendekans kann auch ein Student machen!!!)
      \end{itemize}
\end{itemize}

\textbf{Gleichstellungsbeauftragte/r des Fachbereichs}
\begin{itemize}
 \item FBR stellt eine/n Gleichstellungsbeauftragte/n
  \item offiziell: ein/e Gleichstellungsbeauftragte/n und bis zu drei StellvertreterInnen -- meistens aber eher ein/e studentische/r, ein/e nichtwissenschaftliche/r sowie ein/e wissenschaftliche/r Gleichstellungsbeauftragte/r
  \item für alle Belange zuständig, die Gleichstellung innerhalb des Fachbereiches betreffen
  \item Rede- und Teilnahmerecht in allen Gremien des Fachbereiches, soweit es um die Belange der Gleichstellung geht
\end{itemize}

\textbf{Fakultätsrat}
\begin{itemize}
 \item setzt sich aus Mitgliedern der einzelnen Fachbereichrate zusammen
  \item Vorsitz hat FakultätsdekanIn: Gewählt aus Gruppe der Profs
  \item beschäftigt sich mit fächerübergreifenden Angelegenheiten
\end{itemize}

\textbf{Senat}
\begin{itemize}
 \item hat am 11.7.07 bzw. am 7.2.08 nahezu alle bedeutenden Entscheidungskompetenzen an den Hochschulrat abgegeben
\end{itemize}

\textbf{Hochschulrat}
\begin{itemize}
 \item höchstes Beschlussfassendes Gremium der Universität
  \item Entscheidungen über Verteilung der Stellen und Finanzen, Einrichtung/Aufhebung von Fachbereichen
  \item Beschlüsse über Satzungen und Ordnungen der Universität
  \item Anträge an den Konvent
  \item besteht aus 5 uni-externen und 3 uni-internen Mitgliedern
  \item soll einem Aufsichtsrat entsprechen
  \item wählt Hochschulleitung, stimmt über Hochschulentwicklungs- \linebreak und Wirtschaftsplan ab und kann Einrichtung und Schließung von Studiengängen beschließen
  \item tagt nicht öffentlich
  \item weiteres Beispiel für den Wandel an den Hochschulen zu Kaderschmieden der Industrie, da nun eine Mehrheit aus Wirtschaftsvertretern gänzlich ohne Mitsprache des Mittelbaus oder der Studierenden, die wichtigsten Entscheidungen der Uni trifft
\end{itemize}

\textbf{Rektorat}
\begin{itemize}
 \item hat Rederecht im Senat
  \item besteht aus ProrektorInnen, KanzlerIn und RektorIn
  \item bereitet Senatssitzungen vor
  \item dem Senat gegenüber rechenschaftspflichtig
  \item entscheidet in Verwaltungsangelegenheiten
  \item ProrektorInnen haben ständigen Vorsitz in Kommissionen des Senats
  \item RektorIn beruft Senatssitzungen ein und führt dessen Beschlüsse aus
\end{itemize}

\newpage

\section{Unsere Profs}
\begin{small}
\begin{longtable}{p{0.4\columnwidth} p{0.3\columnwidth} p{0.2\columnwidth}}
  Name & eMail und Telefon & Institut\\ \hline \hline
  Prof.\,Dr.\,Michael Hemmer & \url{michael.hemmer} \newline Tel.:\,83--39365 & Didaktik Geographie\\
  Prof.\,Dr.\,Gabriele Schrüfer & \url{gabriele.schruefer} \newline Tel.:\,83--39349 & Didaktik Geographie\\  \hline
  Prof.\,Dr.\,Ulrike Grabski-Kieron & \url{kieron} \newline Tel.:\,83--33922 & Geographie\\
  Prof.\,Dr.\,Marion Klemme & \url{marion.klemme} \newline Tel.:\,83--33929 & Geographie\\
  Prof.\,Dr.\,Paul Reuber & \url{p.reuber} \newline Tel.:\,83--30035 & Geographie\\
  Prof.\,Dr.\,Gerald Woold & \url{geosek} \newline Tel.:\,83--30026 & Geographie\\ \hline
  Prof.\,Dr.\,Angela Schwering & \url{angela.schwering} \newline Tel.:\,83--33059 & Geoinformatik\\
  Prof.\,Dr.\,Edzer Pebesma & \url{edzer.pebesma} \newline Tel.:\,83--33081 & Geoinformatik\\ 
  Prof.\,Dr.\,Christian Kray & \url{c.kray} \newline Tel.:\,83--33073 &  Geoinformatik \\ \hline
  Prof.\,Dr.\,Christian Blodau & \url{c.blodau} \newline Tel.:\,83--30209& Lök \\ 
  Prof.\,Dr.\newline Tillmann Buttschardt & \url{tillmann.buttschardt} \newline Tel.:\,83--30104 & Lök\\
  Prof.\,Dr.\,Norbert Hölzel & \url{norbert.hoelzel} \newline Tel.:\,83--33994 & Lök\\
  Prof.\,Dr.\,Otto Klemm & \url{otto.klemm} \newline Tel.:\,83--33921 & Lök\\
  Prof.\,Dr.\,Hermann Mattes & \texttt{mattes\textbf{h}} \newline Tel.:\,83--33996 & Lök\\
  Prof.\,Dr.\,Andreas Schulte & \url{andreas.schulte} \newline \url{@wald-zentrum.de} \newline Tel.:\,83--30121 & Lök\\ 
 \hline
  \multicolumn{3}{l}{Bei den \url{eMail Adressen} ist jeweils} \\
  \multicolumn{3}{l}{``@uni-muenster.de'' zu ergänzen!!!}\\
\end{longtable}
\end{small}

%\newpage

%\textbf{StudienberaterInnen:}\\ \\
%\begin{tabular}{p{0.3\columnwidth} p{0.7\columnwidth}}
%Geographie & Dr. Christoph Scheuplein \newline \url{christoph.scheuplein}|Tel.:\,83--33925\\
%&\\
%Geographie \newline (2-Fach Bacherlor) & Prof.\,Dr.\,Gerald Woold \newline \url{geosek}|Tel.:\,83--30025\\
%&\\
%Geographie \newline (Lehrämter) & Dipl.\,Geogr.\,Katja Wrenger \newline \url{katja.wrenger}|Tel.:\,83--39364\\
%&\\
%Geoinformatik & Prof.\,Dr.\,Angela Schwering \newline \url{angela.schwering}|Tel.:\,83--33059\\
%&\\
%Lök & Dr. Andreas Vogel \newline \url{voghild}|Tel.:\,83--33698\\ \hline
%\multicolumn{2}{l}{Bei den \url{eMail Adressen} ist jeweils}\\
%\multicolumn{2}{l}{``@uni-muenster.de'' zu ergänzen!!!}\\
%\end{tabular} 
