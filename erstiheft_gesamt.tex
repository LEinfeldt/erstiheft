%%%%%%%%%%%%%%%%%%%%%%%%%%%%%%%%%%%%%
%%% XeTeX version in UTF-8 encoding
%%%
%%% uses biblatex 0 with utf-8 bug
%%%%%%%%%%%%%%%%%%%%%%%%%%%%%%%%%%%%%


\documentclass[%
a5paper,
pagesize, %stellt papiergröße auf die logische seitengröße um
10pt,
parskip=false,
twoside,
smallheadings,
% draft,
DIV12,
BCOR=10mm,
]{scrreprt}

% \recalctypearea

%%%XeTeX-Code
\usepackage{fontspec}
\usepackage{xunicode}
\usepackage{xltxtra}

%%%
\setromanfont[Mapping=tex-text]{Linux Libertine O}
\setsansfont[Mapping=tex-text, 
%Ligatures=NoCommon
]{Linux Biolinum O}
%\setmonofont[Mapping=tex-text, Scale=0.9]{Courier New}

%%% obsolete with xetex
%%%\usepackage[utf8]{inputenc}



\usepackage[ngerman]{babel}
\usepackage[german=quotes]{csquotes}

% \usepackage[a4, cross, center]{crop} % schnittmarken, frame zeichnet rand um logische seite

%%% obsolete with xetex
%%%\usepackage[pdftex]{graphicx}
\usepackage{graphicx}
% \usepackage{multicol}
% \usepackage{pdfpages}

% \usepackage{amssymb} %ext. mathsymbols
 \usepackage{amsmath} %mathgen
%
 \usepackage{gensymb} % \degree \celsius \perthousand
% \usepackage{textcomp}

\usepackage{units} %using typografic fractions in textmode \unit{23}[kg] \unitfrac[42]{m}{s}
\usepackage{amstext} %mathrm-substitute for non-mathematical test parts in equation mode

%\usepackage{rotating} %rotate
%\usepackage{pdflscape} %rotate anything

\usepackage{colortbl} % shading array cells and regular text
\usepackage{multirow} %spanning over multiple rows
%\usepackage{supertabular} %longtables with multicol
\usepackage{tabularx}
\usepackage{booktabs}
\usepackage[format=plain,font=small,figurewithin=none,tablewithin=none]{caption} %xwithin=none setzt die nummerierung fortlaufend im ggs. zu nach kapitelnummer

\usepackage{eurosym}
\usepackage{longtable}
\usepackage{url}
% \usepackage{timestamp}
%\usepackage{color}
\usepackage[nottoc, notbib]{tocbibind} %add appendix, bibliography, tofigures, totables etc. to toc [nottoc,notlof,notlot]


\usepackage{textcomp}
\usepackage[plainheadsepline, nouppercase]{scrpage2} %linie unter kopfzeile auch auf der ersten seite eines kapitel und gemischte schreibweise erzwingen

\usepackage{datetime} %compilation time
\usepackage{microtype}

\usepackage{lettrine} %Big letter on start

%-----Fußnoten:--------
\pagestyle{scrheadings}
\clearscrheadings
\clearscrplain
\rohead{\footnotesize{Ersti-Info}}
\lehead{\footnotesize{Ersti-Info}}
\lohead{\footnotesize{Geographie - Landschaftsökologie - Geoinformatik}}
\rehead{\footnotesize{Geographie - Landschaftsökologie - Geoinformatik}}
\lefoot{\pagemark}
\rofoot{\pagemark}
%\setheadsepline{.4pt} % Linie unter dem Head
%\setfootsepline{.4pt} % Linie Ganzunten
\renewcommand*{\chapterpagestyle}{scrheadings}


\KOMAoptions{cleardoublepage=empty}
\setkomafont{captionlabel}{\bfseries\small}
\setkomafont{caption}{\small}
%---------------------------------%
%neuer Befehlt mit dem tocdepth innerhalb des dokumentes geändert werden kann
\newcommand{\settocdepth}[1]{%
  \addtocontents{toc}{\protect\setcounter{tocdepth}{#1}}}
%---------------------------------%


\begin{document}
\newcolumntype{C}{>{\small\centering\arraybackslash}X}
\newcolumntype{Y}{>{\small\raggedright\arraybackslash}X}
\newcolumntype{D}{>{\small\arraybackslash}X}
\newcolumntype{F}{>{\small\arraybackslash}p{5mm}}

\renewcaptionname{ngerman}{\figurename}{Abb.}
\renewcaptionname{ngerman}{\tablename}{Tab.}

\setcounter{page}{3}

%\chapter*{Vorwort}
\addcontentsline{toc}{chapter}{Vorwort}
\thispagestyle{plain}
\lettrine[lines=3,loversize=0.2,slope=-15,lhang=0.2]{V}{or} euch liegt das neue Erstsemester-Infoheft 2014, in dem wir euch aktuelle Informationen zu euren Studiengängen, Modulen, Nebenfächern und vielem mehr bieten. Dieses Info-Heft eurer Fachschaft soll euch nicht nur im ersten Semester weiterhelfen, sondern auch während des Studiums immer wieder eine Nachschlagemöglichkeit bieten. Manche Dinge können sich natürlich mit der Zeit ändern, daher empfiehlt sich regelmäßig der Blick auf die Homepages der Institute und unserer Fachschaften. Bitte vergesst auch nicht, dass die Erstellung eines solchen Info-Heftes, besonders mit so vielen fleißig mithelfenden Menschen und so vielen beteiligten Studiengängen, immer eine Menge Arbeit bedeutet. Es ist wohl nicht zu vermeiden, dass sich der ein oder andere Fehler einschleicht oder eine Formatierung nicht optimal sitzt. Wir hoffen, ihr könnt dennoch etwas mit diesem Heft anfangen und wünschen euch viel Spaß beim Lesen!

Wie sich vielleicht schon herumgesprochen hat, sind die Institute erst vor kurzem in ein neues Gebäude umgezogen. Dies wurde zum Einen durch eine PCB Belastung des alten Gebäudes, zum anderen durch die Vergrößerung der Institute mit der Zeit, fällig. Das insgesamt etwa 30 Millionen Euro teure Gebäude berücksichtigt bewusst ökologische Aspekte wie Energieeffizienz und Nachhaltigkeit und wurde u.a. dafür von der Deutschen Gesellschaft für Nachhaltiges Bauen (DGNB) mit dem Vorzertifikat in Silber ausgezeichnet. Der Neubau bietet auf 6700 Quadratmetern Nutzfläche Platz für die Forschungsstelle für Paläobotanik, den Instituten für Didaktik der Geographie (inkl. neuem Lehr-/Lernatelier), Geographie, Geoinformatik und Landschaftsökologie. Dadurch können alle Studierende unseres Fachbereichs von deutlich verkürzten Wegen und Pendelzeiten profitieren.
%Die eine oder der andere von euch wird vielleicht schon erfahren haben, dass voraussichtlich im Frühjahr 2013 ein lang erwarteter Umzug bevorsteht. Bei der derzeitigen Situation, in der die Institute unseres Fachbereichs quer durch die Stadt verteilt sind, handelt es sich ja nur um eine Übergangslösung. Sie wurde nötig, nachdem man festgestellt hatte, dass ein Teil des nicht mehr ganz jungen Gebäudes an der Robert-Koch-Straße 26 mit polychlorierten Biphenylen (PCB) und Asbest belastet war. Am 14. Juli 2011 wurde jedoch der Grundstein für den Neubau der Geowissenschaften an der Ecke Corrensstraße/Mendelstraße gelegt - in angenehmer Nähe zur Mensa II am Coesfelder Kreuz. Das insgesamt etwa 30 Millionen Euro teure Gebäude berücksichtigt bewusst ökologische Aspekte wie Energieeffizienz und Nachhaltigkeit und wurde u.a. dafür von der Deutschen Gesellschaft für Nachhaltiges Bauen (DGNB) mit dem Vorzertifikat in Silber ausgezeichnet. Der Neubau wird auf 6700 Quadratmetern Nutzfläche neben der Forschungsstelle für Paläobotanik den Instituten für Didaktik der Geographie (inkl. neuem Lehr-/Lernatelier), Geographie, Geoinformatik und Landschaftsökologie ein neues zu Hause bieten. Für euch bedeutet das zwar auch, dass ihr euch nach etwa einem halben Jahr Studium noch einmal umgewöhnen müsst, dafür können dann alle Studierende unseres Fachbereichs von deutlich verkürzten Wegen und Pendelzeiten profitieren.

%Beim Personalbestand gab es bereits im vergangenen Semester einigen frischen Wind, insbesondere im Bereich der Geographie. Dort mussten verhältnismäßig viele Stellen neu besetzt werden, was gewisse Herausforderungen für den Lehrbetrieb mit sich brachte. Mittlerweile hat sich die Lage wieder normalisiert und wir hoffen, dass sich die neuen Mitarbeiter gut an der WWU eingelebt haben und ihr Know-How auch für euch gewinnbringend einsetzen. 

Noch ein letzter Absatz in eigener Sache: Um auch weiterhin erfolgreich zu arbeiten, benötigen wir eure Unterstützung. Wer wir sind und was wir eigentlich machen, findet ihr auf den folgenden Seiten oder könnt es auf unserer Homepage nachlesen. Wir würden uns freuen, wenn sich einige von euch angesprochen fühlen und Lust haben, einfach mal vorbeizukommmen und reinzuschnuppern. Wir sind ein lustiger Haufen, beißen nicht und spätestens bei einem gemütlichen Bierchen werdet ihr feststellen, dass Fachschaft nicht nur Arbeit bedeutet! Also, viel Spaß beim Lesen und vor allem bei eurem Studium!
\bigskip
\newline
Eure Fachschaften GeoLök \& Geoinformatik



\cleardoublepage 

\begingroup %group-Trick um Inhaltsverzeichnis und Impressum auf eine Seite zu bekommen 
\let\newpage\relax
\let\clearpage\relax
\vspace*{-1.8cm}
\tableofcontents
\newpage \bigskip \bigskip
\begin{small}
\begin{center}
\textbf{ Öffnungszeiten und Adressen}
\end{center}
\end{small}
\begin{tiny}
\sffamily
\begin{minipage}[t]{0.5\textwidth}
 \begin{center}
\textbf{Geschäftszimmer der Betriebseinheit für die Lehreinheit I}\\
Frau Kreulich\\
% Mo-Fr 10:00--12:30h\\
0251/83--33910 \\
Be1.geo@uni-muenster.de\\
%[0.5cm]
\bigskip 
%----------NÄCHSTER EINTRAG-------
\textbf{Prüfungsamt der Fachbereiche der Mathematisch- Naturwissenschaftlichen Fakultät}\\
Frau Wohlgemuth\\
Orléans-Ring 10, 48149 Münster\\
Kerstin.Wohlgemuth@uni-muenster.de\\
\bigskip
%----------NÄCHSTER EINTRAG-------
\textbf{Front Office Geographie Landschaftsökologie}\\
Heisenbergstr. 2, 48149 Münster\\
0251/83-33986\\
frontofficegeoloek@uni-muenster.de\\
Sprechzeiten: Mo-Do 9:00-16:00 und nach Vereinbarung
\bigskip         	
\end{center}
\end{minipage}
\begin{minipage}[t]{0.5\textwidth}
\begin{center}
 \textbf{Geschäftszimmer Institut für Geographie}\\
  Frau Härtl\\
  0251/83--33992\\
  geosek@uni-muenster.de\\
 \bigskip
%----------NÄCHSTER EINTRAG-------
\textbf{Geschäftszimmer Institut für Didaktik der Geographie}\\
Frau Steinau\\
0251/83--39353\\
ifdg@uni-muenster.de\\
\bigskip
%----------NÄCHSTER EINTRAG-------
\textbf{Studienfachberater}\\
\textit{Lök:} Dr. Andreas Vogel, voghild@uni-muenster.de\\
\textit{2-Fach-BA:} Prof. Gerald Wood, gwood@uni-muenster.de\\
\textit{Geo:} Dr. Christoph Scheuplein, christoph.scheuplein@uni-muenster.de\\
\textit{GI:} Prof. Christian Kray, c.kray@uni-muenster.de\\
\bigskip
\bigskip
\end{center}
\end{minipage}
\begin{minipage}{\textwidth}
\hspace{0.08\textwidth}
\fbox{
  \begin{minipage}[t]{0.3\textwidth}
    \textbf{Fachschaftsvertretung\\ Geoinformatik}\\
    Präsidienst siehe Aushang, Internet\\
    0251/83--33947\\
    \url{fsgi@uni-muenster.de}\\
    \url{http://geofs.uni-muenster.de/geoinf}
  \end{minipage}
     }
\hspace{0.18\textwidth}
\fbox{
  \begin{minipage}[t]{0.3\textwidth}
    \textbf{Fachschaftsvertretung\\ Geographie/Landschaftsökologie}\\
    Präsidienst siehe Aushang, Internet\\
    0251/83--33919\\
    \url{fsgelok@uni-muenster.de}\\
    \url{http://geofs.uni-muenster.de/geoloek}
  \end{minipage}
     }
\end{minipage}
\end{tiny}
\bigskip
\begin{center}
\begin{small}
 \textbf{Impressum} \\ \bigskip
%\end{small}
%\begin{tiny}
Herausgeber: FSV Geographie/Landschaftsökologie\\
und\\
FSV Geoinformatik\\
Heisenbergstraße 2, 48149 Münster\\
% Redaktion: Carlotta Böhm, Tobias Bunte, Magdalena Burger, Judit Hejkal, Janine Hellwig, Moritz Hölscher, Jonas Nötzel, Sebastian Seidel, Lena Stickling\\
Stand: August 2013 \\
\end{small}
\begin{tiny}
\sffamily
\textbf{Alle Angaben ohne Gewähr!!!}
\end{tiny}
\end{center}

\endgroup
  
  \cleardoublepage
  \include{einführung}
  \cleardoublepage
  \chapter{Das Bachelorstudium}
\section{Zusammenfassung}
Im Zuge der Harmonisierung des europäischen Hochschulraums, auch als "`Bologna-Prozess"' bekannt, verstehen die Hochschulpolitikerinnen und Hochschulpolitiker in Deutschland vor allem die flächendeckende Umstellung der Studiengänge auf konsekutive Bachelor- und Masterstudiengänge. Bachelor-Absolventinnen und Absolventen können sich anschließend für einen weiterführenden Master-Studiengang bewerben. Der Titel eines Bachelors ist dabei ein vollwertiger akademischer Grad, der bei der Bewerbung für einen Masterplatz grundsätzliche Voraussetzung ist. 

Das Bachelor-Studium gliedert sich in:
\begin{enumerate}
 \item Eine Grundlagenphase
 \item eine Vertiefungsphase
 \item ein Modul zur Erlangung zusätzlicher Schlüsselqualifikationen \\(General Studies)
 \item und die Bachelorarbeit
\end{enumerate}
Die Studienordnung für den Bachelor ist „modularisiert“ aufgebaut. Die Regelstudienzeit beträgt 3 Jahre, also 6 Semester. Der modulare Aufbau soll eine überschaubare Strukturierung und eine bessere Schwerpunktbildung ermöglichen. Für ein erfolgreiches Studium müssen 180 Creditpoints (CP) erworben werden. Diese sammelt man durch bestandene Module, die jeweils wieder in einzelne Veranstaltungen unterteilt sind.

Ein Modul ist eine Sammlung von Veranstaltungen. Veranstaltungen sind z.B. Vorlesungen (V), Übungen (Ü), Seminare (S) oder Praktika (P). Die Veranstaltungen sind zu thematischen Modulen zusammengefasst. Erst wenn man alle Veranstaltungen zu einem Modul bestanden hat, ist das Modul abgeschlossen.

In jedem Modul müssen Studienleistungen erbracht werden, die sich auf eine oder mehrere Veranstaltung des Moduls beziehen. Studienleistungen sind zum Beispiel Protokolle, Referate, Hausarbeiten, mündliche Prüfungen oder Klausuren. Einige davon sind prüfungsrelevant, d. h. sie gehen in die Modulnote ein. Ihr könnt aus eurer Prüfungsordnung entnehmen, ob und wenn ja mit welchen Prozentsatz eine Studienleistung in die Modulnote eingeht.

Aus den Modulnoten wird schlussendlich die Abschlussnote für den Bachelor berechnet. Dabei gehen manche Modulnoten gar nicht, anderen mit einfacher und wieder andere zweifacher Gewichtung in die Bachelor"-Abschlussnote ein. Details dazu stehen ebenfalls in der Studienordnung.

Falls nun einmal der Fall eintritt, dass ihr eine prüfungsrelevante Studienleistung nicht bestanden habt, darf diese weitere zwei Mal wiederholt werden. Laut der Bachelor-Rahmenordnung wird man nach dem dritten Scheitern exmatrikuliert.

Die Module erstrecken sich über einen Zeitraum von maximal einem Jahr (2 Semester). Welches Modul man wann belegt ist nicht vorgeschrieben, wohl kann man sich aber an den Studienverlaufsplänen für den jeweiligen Studiengang orientieren. Einige Module setzen das Bestehen anderer Veranstaltungen voraus. Welche Module absolviert werden müssen, ist fest vorgeschrieben. Im weiteren Verlauf hat man einige Wahlmöglichkeiten, mit welchen Veranstaltungen man die Vertiefungsmodule auffüllt.

Die mit dem ECTS (European Credit Transfer System) ermittelten Leistungspunkte werden zur Berechnung des studentischen Arbeitsaufwands genutzt. Ein ECTS-Punkt entspricht 30 Arbeitsstunden eines durchschnittlichen Studierenden und beinhaltet sowohl die Zeit der Lehrveranstaltungen in der Uni, als auch die Zeit der Vor- und Nachbereitung zu Hause.

\section{Veranstaltungsformen}
\textbf{Vorlesungen (V)} dienen der allgemeinen Einführung in ein Thema und sind von der Kommunikation her meist relativ einseitig. Vorne wird geredet, ihr hört zu. Aber Vorlesungen sind das beste Mittel, um sich viel Stoff in kurzer Zeit gut aufbereitet reinzuziehen – zumindest bei guten Dozenten welche sich meist auch über Nachfragen und kurze Diskussionen freuen.
\\
\textbf{Seminare (S)} sind zur Vertiefung einer bestimmten Thematik gedacht und geben die Möglichkeit zur Beteiligung der Studierenden. Sie sind am ehesten mit dem "`guten alten Schulunterricht"' zu vergleichen, wobei das Gewicht deutlicher auf Referaten und Gruppenarbeiten durch die Studierenden liegt. Die Anforderungen variieren zwischen den Seminaren, es besteht jedoch keine allgemeine Anwesenheitspflicht mehr! Nur noch in begründeten Ausnahmefällen - dazu zählen beispielsweise Exkursionen und Geländeübungen sowie auch Laborpraktika und bestimmte Seminare - können die Studierenden zur regelmäßigen Teilnahme verpflichtet werden. Solltet ihr dennoch eurer Meinung nach unberechtigt zu einer Anwesenheitskontrolle gezwungen werden, wendet euch an uns oder sprecht direkt mit den Dozenten!

\textbf{Übungen (Ü)} können sehr unterschiedlich ausfallen und sind (zumindest von Studierendenseite) nicht immer von Seminaren zu unterscheiden. Generell sollen die Studierenden eigenständig oder unter Anleitung Aufgaben der Übungszettel lösen, um so den zuvor in Vorlesungen thematisierten Stoff zu verinnerlichen. Teilweise finden Übungen allerdings auch in Form von Exkursionen statt (z. B. Physische Geographie I). Die Anmeldung zu den Seminaren und teilweise auch den Übungen findet über das LSF/ QISPOS-Portal statt (siehe Prüfungsanmeldungen/ Seminarplatzvergabe).

\textbf{Tutorien (TUT)} werden von Studierenden höherern Semesters angeboten und sind im Wesentlichen eine Hilfestellung zum Verständnis von Vorlesungs- oder Seminarstoff.

\textbf{Exkursionen (Exk)} werden auch \textbf{Geländetage} genannt. Hier soll das bisher eingetrichterte Wissen \enquote{am Objekt} überprüft und vertieft werden. Böse Zungen behaupten, dass die studentischen Entscheidungskriterien folgende seien: 1. wie originell ist das Exkursionsziel; 2. wo wollte ich schon immer mal Urlaub machen; 3. wie finde ich den/die ExkursionsleiterIn; 4. \enquote{wat kost der Spasss}. Je nach Andrang kann es problematisch sein, einen Platz in einer großen Exkursion zu bekommen. Was man allerdings schon zu Beginn seines Studiums beachten sollte, sind die Kosten, die durch Exkursionen entstehen. Bei längeren Exkursionen kommen, je nach Ziel, 500 bis 800 € Grundkosten auf euch zu, die es zu stemmen gilt. Man sollte sich also dementsprechend einen gewissen Betrag zur Seite legen.

Von Exkursionen erfährt man außer aus dem Vorlesungsverzeichnis häufig über Aushänge und die Anmeldung erfolgt in vielen Fällen über einen Listeneintrag bei dem/der zuständigen DozentIn.

\section{Seminarplatzvergabe/Prüfungsanmeldungen}

Die Anmeldung zu Veranstaltungen verläuft grundsätzlich über \textbf{HISLSF}, welches einen Überblick über sämtliche Vorlesungen und Seminare, die in dem jeweiligen Semester angeboten werden, bietet. Es handelt sich hierbei sozusagen um das \textbf{digitale Vorlesungsverzeichnis} mit allen wichtigen Infos zu den Veranstaltungen. Über das HISLSF meldet man sich für belegungspflichtige Veranstaltungen an. Das sind in der Regel Übungen und Seminare. Für Vorlesungen muss man sich meistens nicht anmelden. Die Anmeldephase bei HISLSF unterscheidet sich von der QISPOS-Anmeldephase und liegt meist am Ende des vorherigen Semesters. Informiert euch zudem über eventuelle \textbf{weitere Anmeldungsformalitäten}, da teilweise instituts- bzw. dozentenabhängig weitere Vorgaben gelten!

Zudem dient eine weitere Rubrik dieses Portals, das so genannte \textbf{QISPOS}, der digitalen \textbf{Prüfungsverwaltung und -anmeldung}. Hierfür wird in jedem Semester über einen gewissen Zeitraum hinweg das System für Anmeldungen geöffnet. Dieser Zeitraum beginnt meist einige Wochen nach Semesterbeginn. Versäumt ihr eine Anmeldung in diesem Zeitraum auf dem Portal, ist eine Teilnahme an der jeweiligen Prüfungsleistung nicht möglich. Seit einiger Zeit besteht auch nicht mehr die Möglichkeit, dass der/die betreffende DozentIn euch nachträglich anmeldet, sodass eine pünktliche Prüfungsanmeldung in QISPOS unbedingt nötig ist. Ihr findet dieses Prüfungssystem unter folgendem Link: \textbf{\url{https://studium.uni-muenster.de/qisserver/}}. Ebenfalls werden auf dieser Internetseite die genauen Daten der Anmeldephase bekannt gegeben.

\section*{Die Computerräume (StudLabs) und das ZDM}
Die Auseinandersetzung mit der harten Computerwirklichkeit bleibt niemandem erspart, denn ein gewisses Grundwissen über den Umgang mit Computern wird inzwischen eigentlich überall vorausgesetzt. Unsere Lehreinheit (also IfG, ILÖK, IfDG, IfGI) besitzt drei StudLabs und einen Medienraum (ZDM). Die drei StudLabs findet ihr im ersten Stock des Geo-Gebäudes (Heisenbergstr. 2) in den Räumen 125, 126 und 130. Alle StudLabs sind grundsätzlich allen Studierenden frei zugänglich. Bevor man eine StudLab betritt, sollte man auf dem an der Tür klebenden Belegungsplan nachschauen, ob der Raum auch wirklich frei ist. Dozenten schauen schon ein wenig verärgert, wenn der 20igste Studi innerhalb von 15 Minuten das Seminar stört.

Um an den Computern im StudLab arbeiten zu können, braucht ihr eure Uni-Kennung, die gleichzeitig eure Zugangsberechtigung ist (\url{m_must01} und euer dazugehöriges Passwort). Ihr bekommt mit euren Studienbescheinigungen und dem Semesterticket einen Abschnitt geliefert, in dem eure Uni-Kennung vorgegeben wurde. Auch euer erstes Zugangspasswort wird euch mit den Studienunterlagen zugeschickt. Dazu erhaltet ihr eine zur Kennung passende Emailadresse \url{m_must01@uni-muenster.de}. Über diese Emailadresse seid ihr dann für alle Uni-Angelegenheiten (und natürlich auch privat) gut zu erreichen. Da man sich aber weder die vom Computer für euch generierte Emailadresse noch das dazu gehörige Passwort merken kann, ist es ratsam, auf der Homepage des Rechenzentrums (\url{http://www.uni-muenster.de/ZIV}) eine Alias-Emailadresse einzurichten (z.B. \url{max.mustermann@uni-muenster.de}) und das Passwort so für sich zu ändern, dass man es sich merken kann. Wie das funktioniert findet ihr auf der Rechenzentrumsseite detailliert erklärt.\\ \textbf{BEACHTE}: Der Mann hinter der Glasscheibe im Rechenzentrum ist ziemlich gereizt, wenn der 50igste Studi am Tag sein Passwort vergessen hat und einen neuen Zugang haben will\dots \mbox{ }Er wird wirklich böse\dots \\
\textbf{ACHTUNG}: Die Uni benutzt eure Uni-Emailadressen, um euch Informationen zukommen zu lassen, die unter Umständen sehr wichtig sein können! Wenn ihr lieber nur euren alten E-Mail-Account (GMX, WEB, etc.) weiternutzen wollt, dann stellt eine Weiterleitung ein. Ihr bekommt dann die Mails direkt in euer altes Postfach weitergeleitet und verpasst nix. Auch dies ist schnell und einfach unter \url{http://www.uni-muenster.de/ZIV} zu erledigen. Es ist extrem wichtig, dass ihr unter eurer Uniadresse erreichbar seit, also testet bitte, ob die Weiterleitung wirklich funktioniert und beachtet, dass das Postfach eurer normalen Emailadresse eventuell nicht genug Speicherplatz hat. Solltet ihr, aus welchen Gründen auch immer, keine E-Mail Adresse zugewiesen bekommen haben, dann könnt ihr sie im Zentrum für Informationsverarbeitung (Unirechenzentrum) in der Einsteinstraße 60 beantragen.

Im ZDM (Zentrum für Digitale Medien, Heisenbergstr. 2, Raum 11) braucht ihr auch eure Kennung mit Passwort, um an den Computern zu arbeiten. Im Vergleich zu den StudLabs kann man im ZDM scannen, brennen, (bunt-)drucken, schneiden etc. – echt praktisch! Ihr arbeitet dort mit den neuesten Programmen und aktueller Technik und solltet auf jeden Fall mal dort vorbeischauen. Die Öffnungszeiten und Serviceangebote des ZDM findet ihr unter \url{http://zdmweb.uni-muenster.de/}.

  \cleardoublepage
  \chapter{Bachelor of Science (B.Sc.) Geographie}
\lohead{\footnotesize{\textbf{Geographie} - Landschaftsökologie - Geoinformatik}}
\rehead{\footnotesize{\textbf{Geographie} - Landschaftsökologie - Geoinformatik}}

\section{Allgemeines}
Aus den alten Studienrichtungen „Physische Geographie“ und "`Sozialgeographie"' wurden 1992, im Zuge der Gründung des Institutes für Landschaftsökologie (IfL), flugs zwei verschiedene Studiengänge gemacht: Dipl. Geogr. und Dipl. LÖK. Später kam dann noch die Geoinformatik als eigenständiges Studium hinzu. Im Zuge der Europäisierung der Hochschulen (auch bekannt unter dem Namen Bologna-Prozess) wurden die Studiengänge auf das Bachelor/Master-System umgestellt und erfolgreich akkreditiert. Ihr seid mittlerweile der sechste Jahrgang des B.Sc. Geographie.

Planmäßig wird dieses Studium in sechs Semestern absolviert, danach könnt ihr entweder direkt ins Arbeitsleben eintauchen oder noch einen Master draufsetzen. Im Laufe dieses Studiengangs sollt ihr eine umfassende Grundbildung in Physio- und Anthropogeographie, mit einem deutlichen Übergewicht in letzterem, erhalten. Zudem sollen fundierte Kenntnisse in der wissenschaftlichen Methodik der Geographie vermittelt werden. Euer Nebenfach wählt ihr aus einer vorgegebenen Auswahl an Fächern aus, im Bereich der General Studies steht euch ein umfangreiches Kursangebot zur Verfügung, das individuell zusammengestellt werden kann.

Soweit die Theorie. Wie euer Studium konkret aussieht, wird durch die Prüfungsordnung festgelegt. Diese erhaltet ihr auf der Homepage des Instituts für Geographie (IfG). Wir haben in diesem Heft eine Übersicht des Studienverlaufsplans und eine Kurzbeschreibung der einzelnen Module eingefügt, jedoch ersetzt diese \textbf{IN KEINEM FALLE} die Prüfungsordnung. Wenn ihr hierzu Fragen habt, wendet euch an die zuständigen Dozenten bzw. Professoren oder fragt einfach uns – eure Fachschaft. Nachdem ihr euch dann irgendwann durch die Vorschriften und Studienordnungen gefuchst habt, stellt sich vielleicht auch irgendwann mal die Frage: Was macht eigentlich einen Geographen oder eine Geographin aus? Es ist gar nicht so einfach, diese Frage kurz und prägnant zu beantworten, aber versuchen wollen wir es trotzdem einmal.

Die Chance der Geographie liegt zum einen in ihrem breiten Themenfeld (von Natur- über Sozial- bis hin zu Geisteswissenschaften, zum anderen in ihrem variablen Maßstab (von der einzelnen Immobilie bis zum Globus samt Atmosphäre). GeographInnen werden in ihrem Studium mit derart unterschiedlichen Fächern und Fragestellungen konfrontiert, wie es wohl in keinem anderen Fach vorkommt. Natürlich, das behaupten viele Fächer von sich, aber fragt die doch dann einfach mal, ob sie an einem Tag Chemie (in der Bodenkunde), Marketing-Konzepte (in der Geographie des Einzelhandels) und Geschichte (in der Regionalen Geographie) behandeln. Diese Vielfalt erzeugt, wenn man es zulässt, ein breites, aber tendenziell etwas oberflächliches Wissen. Zudem wird vor allem die Fähigkeit geschult, verschiedene Ansätze, Fächer und Denkschulen miteinander zu verknüpfen, um bestimmte Probleme zu lösen.

Dieses vernetzte Denken ist die große Stärke der GeographInnen, die uns von Absolventen anderer Fächer unterscheidet. Es wird immer Menschen geben, die in gewissen Teilbereichen eine größere Kompetenz haben als wir. Es gilt aber, unsere Fähigkeiten mit allem Nachdruck und Selbstbewusstsein zu verkaufen, um nicht eines Tages bei Günther Jauch auf dem Stuhl zu sitzen mit dem Gedanken: "`Bitte keine Frage nach der Hauptstadt von Samoa!"'(Na, wisst ihr sie doch?). Oder buntfarbene Kartierungen als Picassos spätkubistische Phase auszuzeichnen, um der/dem Liebsten längere Erklärungen zu ersparen. In diesem Sinne: Lasst den Globus rotieren!

\section*{Was ist Geographie?}
Das ist wohl eine der schwersten Fragen, auf die jeder Geographiestudent vor oder während seines Studiums eine Antwort sucht. Einsteigen wollen wir bei der Suche nach Antworten mit einer Definition des Deutschen Verbandes für Angewandte Geographie (DVAG): „Geographie ist die Wissenschaft von den räumlichen Unterschieden und dynamischen Systemen“. Das ist erst einmal ein wenig abstrakt.

Man kann es sich vielleicht so vorstellen: Wenn man einen laufenden Film anhält und sich nur das Standbild betrachtet, kann man damit meistens nicht viel anfangen. Man versteht es erst, wenn man die Handlung und die Rollen der Akteure in ihrem Zusammenspiel kennt. Die Geographie forscht nach dem Zusammenspiel der Akteure im Film. „Wie gestalten wir unseren Lebensraum am besten?“ Unter Geographie versteht man also eine Vielzahl von Einzelaspekten, die alle Anteil am Gesamtsystem haben. Die meisten anderen Fachrichtungen beschränken sich auf einen Mosaikstein, während die Geographie versucht, viele dieser Mosaiksteine zusammenzulegen.

Dieses Kunststück schafft sie, wenn überhaupt, nur dadurch, dass sie sich in zwei recht unterschiedliche Bereiche teilt: die Physische Geographie (Natur und Umwelt) und die Anthropogeographie (Mensch und Umwelt). In der Geographie finden sowohl naturwissenschaftliche als auch sozialwissenschaftliche Methoden ihre Anwendung. Zusammenfassend kann also gesagt werden, dass die Geographie fächerübergreifend arbeitet und auch viele Gebiete beinhaltet, die auch von anderen wissenschaftlichen Disziplinen bearbeitet werden.

Eine noch aufschlussreichere Antwort auf die Frage \enquote{Was ist Geographie?} kann man vielleicht bekommen, wenn man aufzeigt, womit sich Geographiestudenten während ihres Studiums alles so beschäftigen.

\section*{Start in Münster}
Ihr möchtet jetzt wahrscheinlich wissen, was euch in der nächsten Zeit erwartet, wie euer Studienplan aussieht etc. Viele dieser Fragen werden bei der Einführungswoche (Ersti-Woche) geklärt, daher solltet ihr nach Möglichkeit daran teilnehmen. Dort stellen sich die Dozenten und die Fachschaft vor und versuchen Euch einen Einblick in das Studium zu verschaffen. Außerdem gibt es immer den einen oder anderen Kneipenabend und anschließend auch noch das sagenumwobene Ersti-Wochenende, um seine Kommilitonen ein wenig kennen zu lernen. Wir von der Fachschaft helfen euch, wo wir können, denn wir wissen wie es ist, als Erstsemester ein Studium in einer (für viele) neuen Stadt anzufangen.

Im ersten Semester steht in erster Linie die vierstündige Vorlesung \enquote{Einführung in die Humangeographie} an. Dazu gibt es noch die Ringvorlesung \enquote{Physische Geographie I} sowie weitere Veranstaltungen. Vielleicht kommt euch euer Stundenplan etwas dürftig vor, aber Uni ist eben doch nicht dasselbe wie Schule, daher plant vor allem für \enquote{Einführung in die Humangeographie} eine Menge Zeit zum Vor- und Nachbereiten ein, da eine Menge Lesestoff anliegt und die Klausur am Ende des Semesters sehr lernaufwendig ist. Am besten ihr bleibt von Anfang des Semesters an am Ball, damit nicht irgendwann der Berg immer größer und die Zeit immer knapper wird. Lasst euch aber nicht von einer allgemeinen Panik-Mache mitreißen. Wenn man vernünftig für die Klausur lernt, dann besteht man sie auch!

\section{Studienverlaufsplan}
Der B.Sc. Geographie ist, wie bereits erwähnt, in sechs Semester unterteilt. Einige der Module erstrecken sich jedoch über zwei Semester, daher kann man eher von einer Einteilung in drei Jahre sprechen. Die ersten beiden Semester dienen dem Erlernen der Grundlagen der Geographie. Das zweite Studienjahr dient dem Aufbau des bereits erlangten Wissens, während die letzten beiden Semester dieses vertiefen sollen.

In diesen sechs Semestern müssen insgesamt 16 Module abgeschlossenen werden. Auf der folgenden Seite findet ihr den/einen Verlaufsplan eures Studiums. Wir empfehlen dringend, das erste Jahr wie im Verlaufsplan vorgegeben zu absolvieren, da die hier zu absolvierenden Module Voraussetzung für folgende Module sind. In den weiteren beiden Jahren können Kurse zum Teil auch früher oder später als im Verlaufsplan vorgegeben absolviert werden. Ihr solltet dabei jedoch immer im Auge behalten, dass ihr bestimmte Module benötigt, um andere belegen zu können. Die Zeiteinteilung des Nebenfaches ist unabhängig von den übrigen Modulen und kann sehr unterschiedlich ausfallen. Die Zeiteinteilung der General Studies könnt ihr komplett frei bestimmen.

Generell gilt: Der Studienaufbau ist stärker vorgegeben als dies noch beim Diplom der Fall war, allerdings gibt es hier und da auch Möglichkeiten etwas nach vorn oder hinten zu schieben. Letzteres kann aber auch schnell mal in die Hose gehen, so dass ein oder zwei Semester dran gehängt werden müssen.

%------------------------GRAPHIK FOLGT NOCH-----------------------------------------
%\newpage
%\includegraphics[angle=90, scale=0.5]{modulGEO}
%\newpage
%------------------------GRAPHIK FOLGT NOCH-----------------------------------------

\section{Modulbeschreibung}
\begin{enumerate}
 \item \textbf{Modul 1 "Humangeographie 1a"}\\ besteht aus:
  \begin{enumerate}
   \item Vorlesung: Einführung in die Humangeographie
   \item Übung: Humangeographie
   \item Tagesexkursion
   \item Tagesexkursion
  \end{enumerate}
  Notenvergabe:
  \begin{enumerate}
   \item Klausur \textbf{SEHR LERNINTENSIV!!!}
   \item Übungsaufgaben; Präsentation oder Hausarbeit
   \item Exkursionsprotokoll
  \end{enumerate}    % [] bewirkt, dass die nummerierung fehlt
  \item[] Note = a) 60\% der Note, b) 40\% der Note

 \item \textbf{Modul 2 "Humangeographie 1b"} \\ besteht aus:
  \begin{enumerate}
   \item Übung: Einführung in das Studium der Geographie
   \item Übung Humangeographie
   \item Tagesexkursion
  \end{enumerate}
  Notenvergabe:
  \begin{enumerate}
   \item Übungsaufgaben; Präsentation
   \item Übungsaufgaben; Präsentationen oder Hausarbeit
   \item Exkursionsprotokoll
  \end{enumerate}
  \item[] Note = b) 100\% der Note

 \item \textbf{Modul 3 "Physische Geographie I"}  \\ besteht aus:
  \begin{enumerate}
   \item Vorlesung: Physische Geographie
   \item Übung: Physische Geographie
  \end{enumerate}
  Notenvergabe:
  \begin{enumerate}
   \item Klausur
   \item unbenotetes aber umfangreiches Gruppen"=Protokoll
  \end{enumerate}
  \item[] Note = a) 100\% der Note

 \item \textbf{Modul 4 "Geographische Erhebungs- und Analysetechniken"}  \\ besteht aus:
  \begin{enumerate}
   \item Seminar: Kartographie und Karteninterpretation
   \item Seminar: Methoden der empirischen Humangeographie
   \item Übung: E-Learning"=Einheit zu „Kartographie und Karteninterpretation“
   \item Übung: E-Learning"=Einheit zu „Methoden der empirischen Humangeographie“
  \end{enumerate}
  Notenvergabe:
   \begin{enumerate}
    \item[] a) Übungsaufgaben; kartographische Arbeit
    \item[] b) Übungsaufgaben; Klausur
  \end{enumerate}
  \item[] Note = a) und b) jeweils 50\% der Note

 \item \textbf{Modul 5 "Einführung in die Raumplanung"}  \\ besteht aus:
  \begin{enumerate}
   \item Vorlesung: Grundlagen der Raumplanung
   \item Seminar: Einführung in die räumliche Planung
   \item Tagesexkursion
  \end{enumerate}
  Notenvergabe:
  \begin{enumerate}
   \item Klausur
   \item Übungsaufgaben, Präsentation (inkl. schriftl. Ausarbeitung) und Planspiel
   \item Protokoll
  \end{enumerate}
  \item[] Note = a) 45\% b) 55\%

 \item \textbf{Modul 6a "Geoinformatik 1a: Grundlagen"}  \\ besteht aus:
  \begin{enumerate}
   \item Vorlesung: Einführung in die Geoinformatik
   \item Übung: Einführung in die Geoinformatik
  \end{enumerate}
  Notenvergabe:
  \begin{enumerate}
   \item Klausur
   \item Übungsaufgaben
  \end{enumerate}
  \item[] Note = a) 100\%

 \item \textbf{Modul 6b "Geoinformatik 1b: GIS Anwendungen"}  \\ besteht aus:
  \begin{enumerate}
   \item Übung: GIS-Grundkurs
   \item Übung: Angewandte Kartographie
  \end{enumerate}
  Notenvergabe:
  \begin{enumerate}
   \item Übungsaufgaben
   \item Thematische Karte
  \end{enumerate}
  \item[] Note = b) 100\%
    
 \item \textbf{Modul 7 "Geoinformatik 2: Geostatistik"}  \\ besteht aus:
  \begin{enumerate}
   \item Vorlesung: Einführung in die Geostatistik
   \item Übung: Einführung in die Geostatistik
  \end{enumerate}
  Notenvergabe:
  \begin{enumerate}
   \item Klausur
   \item Übungsaufgaben
  \end{enumerate}
  \item[] Note = a) 100\%

 \item \textbf{Modul 8 "Ökologische Planung"}  \\ besteht aus:
  \begin{enumerate}
   \item Vorlesung: Grundlagen der ökologischen Planung
   \item Übung: Grundlagen der ökologischen Planung
  \end{enumerate}
  Notenvergabe:
  \begin{enumerate}
   \item Klausur
   \item Umweltbericht (Gruppenarbeit)
  \end{enumerate}
   \item[] Note = a) 100\%

 \item \textbf{Modul 9 " Angewandte Geographie"}  \\ besteht aus:
  \begin{enumerate}
   \item Vorlesung: Angewandte Geographie
   \item Seminar 1: Angewandte Geographie
   \item Seminar 2: Angewandte Geographie
  \end{enumerate}
  Notenvergabe:
  \begin{enumerate}
   \item[b) \& c)] Präsentation
   \item[b) \ODER c)]Modul-Hausarbeit
  \end{enumerate}
  \item[] Note = b) oder c) 100\%

 \item \textbf{Modul 10 "Geographie und Praxis"}  \\ besteht aus:
  \begin{enumerate}
   \item Übung: Berufsfelder der Geographie
   \item Seminar: Praktikumskolloquium (hören und Praktikum vorstellen)
   \item Praktikum (mindestens 4 Wochen)
  \end{enumerate}
  Notenvergabe:
  \begin{enumerate}
   \item[b) Poster-Präsentation oder Praktikumsbericht
   \item[] Note = b) 100\%

 \item \textbf{Modul 11 "Projektbezogenes Geländeseminar"}  \\ besteht aus:
  \begin{enumerate}
   \item Seminar: Projektseminar
   \item Projektbericht
  \end{enumerate}
  Notenvergabe:
  \begin{enumerate}
    \item Ausarbeitung und Präsentation
   \end{enumerate}
   \item[] Note = b) 100\%

 \item \textbf{Modul 12 "Regionale Geographie"}  \\ besteht aus:
  \begin{enumerate}
   \item Vorlesung: Regionale Geographie
   \item Seminar: Regionale Geographie 1
   \item Seminar: Regionale Geographie 2
   \item Exkursion (6 Tage)
  \end{enumerate}
  Notenvergabe:
  \begin{enumerate}
   \item Präsentation
   \item Präsentation
   \item Präsentation und Exkursionsprotokoll
  \end{enumerate}
  \item[]  d) 100\% 

 \item \textbf{Modul 13 "Humangeographie 2"}  \\ besteht aus:
  \begin{enumerate}
   \item Vorlesung: Humangeographie 2
   \item Seminar: Humangeographie 2a
   \item Seminar: Humangeographie 2b
  \end{enumerate}\newpage
  Notenvergabe:
  \begin{enumerate}
   \item[] b) und c) Präsentation oder Hausarbeit
   \item[] Mündliche Modulabschlussprüfung (45 Min.)
  \end{enumerate}
  \item[] Note = Mündliche Modulabschlussprüfung 100\%

 \item \textbf{Modul 14 "Allgemeine Studien"}  \\ besteht aus:
  \begin{enumerate}
   \item[] Veranstaltung aus dem Angebot der WWU
  \end{enumerate}
  Notenvergabe:
  \begin{enumerate}
   \item[] Abhängig von der jeweiligen Veranstaltung
  \end{enumerate}

 \item \textbf{Modul 15 "Wahlbereich/Nebenfächer"}  \\
  Notenvergabe:
   \begin{enumerate}
    \item[] Abhängig von der jeweiligen Veranstaltung
   \end{enumerate}
 
 \item \textbf{Modul 16 "Bachelorarbeit"}  \\ besteht aus:
  \begin{enumerate}
   \item[] Anfertigung eurer Bachelorarbeit 
   \item[] Umfang: 8.000 – 12.000 Worte (entspricht etwa 35-55 Seiten)
   \item[] Themenabsprache mit Dozenten
  \end{enumerate}




\end{enumerate}

\newpage

Mindestens ein außeruniversitäres Praktikum musst du im Laufe deines Studiums auf alle Fälle absolvieren. Dies soll laut Studienordnung mindestens 4 Wochen dauern. Wir empfehlen dir aber, das Praktikum länger durchzuführen. Wo und wann du dich um Stellen bewirbst ist dir überlassen. Es sei in diesem Rahmen darauf hingewiesen, dass das Praktikum einen Einblick in ein späteres Berufsleben geben soll. Du solltest diese Gelegenheit nutzen um dir erstens einen Überblick über das weite Spektrum der Tätigkeitsfelder von B.Sc./M.Sc Geographen/Innen zu verschaffen und zweitens um festzustellen, ob dir eine spätere (lebenslängliche?) Tätigkeit in einem Bereich überhaupt zusagt.

Der beste Zeitpunkt für ein Praktikum ist vermutlich nach dem dritten oder vierten Semester. Dann hat man schon ein bisschen Ahnung, wovon man eigentlich redet, kann aber auf der anderen Seite auch sein Studium noch etwas beeinflussen. Es empfiehlt sich jedoch, mehr als nur dieses eine Pflichtpraktikum zu absolvieren, da praktische Erfahrung zum einen euch selber mehr zu Gute kommt als vielleicht eine weitere absolvierte theoretische Veranstaltung, zum anderen wird diese Erfahrung durch spätere Arbeitgeber gern gesehen. Zudem können durch vernünftig durchgeführte Praktika Kontakte geknüpft werden, welche für den Berufseinstieg sehr hilfreich sind.

\section*{Wahlbereich/Nebenfächer}
Im so genannten Wahlbereich müsst ihr 30 Credit Points in einem bzw. mehreren Nebenfächern belegen. Derzeit können öffentliches Recht, VWL, Politikwissenschaften, Geoinformatik, Geowissenschaften, Niederlande-Studien und Landschaftsökologie als Nebenfach belegt werden. Die ersten beiden genannten Fächer verlangen, dass die gesamten 30 CP in ihrem Fach belegt werden. Bei den letzteren fünf können diese aus Modulen à 10 CP zusammengesetzt werden. Die Wahl des Nebenfachs erfolgt erst nach Vorlesungsbeginn. In der ersten Woche könnt ihr ruhig mehrere Einführungsvorlesungen besuchen und dann schauen, was euch am Besten gefällt. Denkt bei der Auswahl am Ende zum einen daran, was euch am meisten interessiert, zum anderen aber auch, was davon zu einem etwaigen späteren Tätigkeitsfeld passen könnte. Ausführlichere Informationen zu eurem Studiengang inklusive der Nebenfächer findet ihr unter: \url{www.uni-muenster.de/Geographie/studium/studiengang}

\section*{Das Berufsfeld der Geographie}
Um es vorwegzunehmen: es gibt kein eindeutiges Berufsfeld für Absolventen der Geographie. Die Möglichkeiten, im Berufsleben unterzukommen, sind äußerst vielfältig (das ist schön), das Arbeitsplatzangebot ist gering (das ist schade) und die Konkurrenz mit anderen qualifizierten Menschen aus anderen qualifizierten Bereichen ist nicht zu unterschätzen (das ist normal). Soll heißen: es ist alles und nichts möglich, vom Flughafendirektor (z.B. MS-OS) über den Verkehrs- oder Raumplaner im Stadtplanungsamt bis hin zum Journalist werden viele Spektren der Gesellschaft durch Geos frequentiert. Auch eine Karriere als selbständige/r GeographIn ist denkbar und hat schon zu durchaus erfolgreichen Ergebnissen geführt. Daher soll hier auch auf diese o. ä. Fragen keine Antwort gegeben werden, schließlich soll die Antwort das Ergebnis deines Studiums sein (O.K..., das klingt ja doch ein wenig pathetisch...). Im Laufe deines Studiums wirst du wahrscheinlich genügend ökologische Nischen finden, in denen du dir dein späteres, werktätiges Leben vorstellen kannst, womit wir zum Berufsmarkt kommen: Als ich zu Beginn meines Studiums jemandem von meinem Fach erzählte, hörte ich: „Wieso? Wir wissen doch inzwischen, dass die Erde eine Kugel ist?!“.

Dass Geographen keine Entdeckungen mehr tätigen, ist inzwischen bekannt. Seit den 60ern hat sich dafür die Raum- und Umweltplanung als relativ großer Arbeitsmarkt für Geographen etabliert. Da bei der öffentlichen Hand allerdings immer mehr eingespart wird und sich andererseits mittlerweile auch hierfür Spezialisten (Studiengang Raumplanung) entwickelt haben, ist es nicht mehr so leicht „auf dem Amt“ eine Stelle zu bekommen – und sicherlich auch nicht für jedermann erstrebenswert. Gerade in diesem Arbeitsfeld, der planerischen Gestaltung unserer Lebensumwelt, hat sich aber dennoch vieles getan. Es bilden sich immer mehr regionale oder lokale Netzwerke, zum Beispiel durch den Zusammenschluss von Kreisen oder Städten zu einer Region, bei lokalen Einzelhandelsgemeinschaften oder auch bei der interkommunalen Zusammenarbeit über Staatsgrenzen hinaus. In diesen informellen Netzwerken entstehen oft Arbeitsmöglichkeiten für GeographInnen, ob als Angestellte oder in selbstständigen Agenturen. Weitere Arbeitsfelder in diesem Umfeld sind der Tourismusbereich, die Wirtschaftsförderung von Kommunen und Kreisen oder das Stadtmarketing.

Seit einigen Jahren nimmt auch die Bedeutung der Verarbeitung von geographischen Daten stetig zu. Hier reicht das Spektrum von der klassischen Kartenerstellung über die Entwicklung und Wartung von Navigationssystemen und Routenplanern sowie die Erfassung von Vegetation und Bodenqualität bis hin zur kompletten Erstellung von geoinformatischen Datenbanken und deren Verwaltung. Darüber hinaus sind Geographen aber auch in Entwicklungszusammenarbeit und Katastrophenhilfe, Erwachsenenbildung, Unternehmensberatung, öffentlicher Verwaltung, Lobbyarbeit, Politik, Presse und Medien, Immobilienwirtschaft und vielen anderen Bereichen anzutreffen.

Hier konkurriert man natürlich immer mit den „Spezialisten“, also BWLer, Pädagogen, Politikwissenschaftler, Ingenieure, Juristen usw. Gerade hier ist es also wichtig, das breite Wissen und die Fähigkeit zum interdisziplinären Denken offensiv zu vertreten. Also, ihr seht schon: Es gibt keine vorgezeichnete Karriere für Geographen, wie etwa bei Medizinern oder Lehrern. Man sollte deshalb aber nicht in Panik oder Fatalismus verfallen. Macht euch einfach ein paar Gedanken, in welche Richtung ihr gehen wollt. Informiert euch, was es da für Firmen, Organisationen oder Behörden gibt. Macht vielleicht auch ein oder zwei Praktika mehr, als es in eurer Studienordnung mindestens gefordert wird, es lohnt sich in jedem Fall! Und merkt euch immer: Wir Geographen sind die letzten Spezialisten fürs Allgemeine! Weitere Informationen über mögliche Berufsfelder findet ihr auf der Homepage des Deutschen Verbands für angewandte Geographie (DVAG): \url{http://www.geographie.de/dvag/}

\section*{Exkurs: Geheimnisvoller Geograph}
Angenommen, bei einem Sektempfang gibt sich einer der Gäste als Quantenphysiker zu erkennen: Wer würde da nicht vor Ehrfurcht erstarren und sofort das Thema wechseln, um sich keine Blöße zu geben? Was aber, wenn die Partybekanntschaft kein Quantenphysiker ist, sondern ein Geograph? Das wäre weniger problematisch, schließlich hatte jeder mal Erdkunde in der Schule, musste drei Nebenflüsse der Weser aufsagen, den höchsten Berg von Frankreich nennen und hatte zu lernen, wo die Apfelsinen wachsen.

Das einzige, was einem jetzt noch schleierhaft sein kann, wäre, wie jemand mit solchem Wissen Geld verdient. Wollte er das erklären, müsste der neue Bekannte die Geographie beschreiben, müsste von Biotoppflege, Erosionsforschung und Bodenverdichtung berichten, müsste von Satellitenbildern erzählen und von Umweltgutachten, von Raumplanung und Dorferneuerung, von Stadtmarketing und politischer Beratung, von Standortgutachten und Moderation.

Da er das schon so oft hat herbeten müssen, könnte er unwirsch behaupten, Geographen seien so etwas Ähnliches wie Geologen. Zwar wissen längst nicht alle Menschen, dass Geologen Spezialisten für den Aufbau der Erde sind, für die Bildung von Gesteinen und die Lage von Bodenschätzen. Gerade deswegen stößt ihr Beruf in der Gesellschaft auf ähnlichen Respekt wie der des Quantenphysikers. Indem er sich als Geologe ausgibt, würde der Bekannte zudem eine Menge Zeit sparen, weil man ihn am Tag nach dem Sektempfang ohnedies dafür halten würde - ganz so, wie sein Frisör das tut oder seine alte Tante, auch wenn sie den Unterschied schon fünfmal erklärt bekam.

Das Ansehen der Geographen leidet darunter, dass so viele Menschen früher den Erdkundeunterricht genossen oder besser erlitten haben und meinen, die moderne Geographie sei dasselbe. Dummerweise gingen auch Personalchefs früher Mal zur Schule. Wer ihnen als Stellenbewerber den Job etwas erleichtern möchte, gibt sich am besten gleich als Geologe aus. Als solcher wird er zwar auch abgelehnt, aber der Personalchef hätte wenigstens das gute Gefühl zu wissen, wen er da wieder nach Hause geschickt hat. Schuld am verschwommenen Berufsprofil ist, dass Geographen sich für alles zuständig fühlen, womit sie so falsch nicht liegen. Denn ihre Ausbildung streift außer Mikroelektronik und indoiranischer Linguistik so ziemlich alles, was Universitäten an Fächern zu bieten haben.

Böse Zungen behaupten, Geographie studiere nur, wer seit einem misslungenen Knallgas-Versuch am Gymnasium ein gebrochenes Verhältnis zur Naturwissenschaft habe, oder für Betriebswirtschaft nicht in Frage kommt, weil vom Vater kein Geschäft zu übernehmen ist. Wenn sie unter sich sind, bezeichnen sich Geographen selber als Universaldilettanten. Das hält sie nicht davon ab, über andere Disziplinen zu spotten, wo man über unendlich dimensionale Hilbert-Räume promovieren kann, ohne zu wissen, wie man einen Dreisatz berechnet. Zu Fachidioten können aber selbst Geographen werden, die letzten Spezialisten fürs Ganze: Einer schrieb bestimmt eine Diplomarbeit über die Verteilung der Imbissbuden oder Zigarettenautomaten in einer Großstadt. Man sagt das nicht gerne. Aber auch das ist Geographie. 

(Die Zeit Nr. 20 / 14. Mai 1993 – Autor: Walter Schmidt - leicht verändert)

\section*{Schlusswort}
Wie ihr seht, ist euer Studium relativ stark durchgeplant. Daher ist es besonders wichtig, dass ihr euch den Ablauf verinnerlicht. Genau so wichtig ist es, dass ihr euch an Fristen und Termine haltet und diesbezüglich immer auf dem Laufenden seid. Durch verpennte Termine oder verpatzte Klausuren kann man schnell viel Zeit verlieren. Trotz der vielen Formalien hoffen wir, euch nicht zu sehr vom Studium abgeschreckt zu haben. Studieren ist, bis auf die Prüfungen und manchmal viel Arbeit mit Referaten, Hausarbeiten etc., doch eine ganz nette Angelegenheit. Ihr solltet euch auf jeden Fall im ersten Semester neben dem Arbeiten in der Uni genug Zeit nehmen, um Leute kennen zu lernen und Münsters nicht zu unterschätzendes Nachtleben zu erforschen. Wer ausgiebige Informationen über Parties und Locations haben möchte, kann sich auch hier gerne an die Fachschaft richten. Wir haben da auf jeden Fall genügend Erfahrungen. 
Die Fachschaft wünscht euch viel Spaß und wenn’s Probleme gibt, kommt einfach zu uns: wir werden versuchen euch mit Rat und Tat zur Seite zu stehen!

  \cleardoublepage
  \chapter{Bachelor of Science (B.Sc.) Landschaftsökologie}
\lohead{\footnotesize{Geographie - \textbf{Landschaftsökologie} - Geoinformatik}}
\rehead{\footnotesize{Geographie - \textbf{Landschaftsökologie} - Geoinformatik}}

\section{Allgemeines}
Landschaftsökologie?! ...von diesem Studienfach haben viele Leute noch nie gehört. Was wird man denn, wenn man das studiert und was ist das überhaupt? Dies sind Fragen, die man als angehender Landschaftsökologe wohl noch sehr oft zu hören bekommen wird. Was euch genau erwartet, wie eure Prüfungsordnung im Detail aussieht und was am ILÖK sonst noch so los ist, könnt ihr auf den folgenden Seiten nachlesen.

\subsection*{Bewerbung}
Die Bewerbung erfolgt direkt über die Universität Münster (nicht über die ZVS). Es stehen zurzeit etwa 60 Studienplätze zur Verfügung. Anfangen kann man nur im Wintersemester. Da einige Studierende ihren Studienplatz nicht antreten, gibt es in der Regel mehrere Nachrückverfahren. Außerdem gibt es auch immer einige, die während der ersten Semester das Studium abbrechen oder das Fach wechseln.

\section{Studienaufbau}
Den Studienaufbau sowie Informationen zu ALLEM findet ihr in eurer Prüfungsordnung unter www.uni-muenster.de/Landschaftsoekologie/\\studieren/bsc\_loek.html. Ihr werdet die Ersten sein die nach der neuen Studienordnung von 2013 studieren, die war noch nicht ganz fertig als dieses Heft entstand, also können kleine Abweichungen auftreten. Dieses Erstiheft dient nur als Information, die Prüfungsordnung schreibt fest was ihr für ein erfolgreiches Studium tun müsst. Im Laufe eures Studiums wird es voraussichtlich zu geringfügigen Änderungen der Prüfungsordnung kommen. Über diese werdet ihr rechtzeitig informiert und könnt sie zu gegebener Zeit natürlich auch auf der Homepage des ILök nachlesen.
Die Studienordnung für den Bachelor Landschaftsökologie ist „modularisiert“ aufgebaut. Die Regelstudienzeit beträgt 3 Jahre, also 6 Semester. Das Bachelorstudium besteht aus insgesamt 27 Modulen wovon 26 studiert werden müsssen und schließt ein Berufspraktikum, ein Studienprojekt und die Bachelorarbeit ein. Insgesamt müssen für den Bachelorabschluss 180 Credit Points (ECTS) erreicht werden. Für die meisten Module gibt es 5 bis 15 Credit Points. Ein Credit Point entspricht einer Arbeitszeit von ca. 30 Stunden (inklusive Kontakt- und Präsenzzeiten, Vorbereitung, Nachbereitung und Prüfungsvorbereitung). 

Module bestehen aus inhaltlich aufeinander abgestimmten oder sich ergänzenden Lehrveranstaltungen. Die Module erstrecken sich in der Regel über zwei aufeinander folgende Semester, also ein Jahr. Die einzelnen Veranstaltungen eines Moduls werden abgeprüft. Dazu gibt es grundsätzlich mehrere Möglichkeiten, z. B. mündliche Prüfung, Klausur, Test, Vorträge, Protokolle usw. In welchem Modul wie viele oder welche Leistungsnachweise erbracht werden müssen, kann sehr unterschiedlich sein und hängt auch vom Dozenten ab. Fest steht, dass jedes Modul abgeprüft werden muss, und dass ihr diesen „prüfungsrelevanten Leistungsnachweis“ nur zwei mal wiederholen dürft! Insgesamt hat man also drei Versuche. Es gibt allerdings wenige Ausnahmen wie die Chemie. In Chemie dürft ihr die erste Klusur unendlich oft wiederholen. Wie die „Modulabschlussprüfung“ des jeweiligen Moduls vermutlich aussehen wird, könnt ihr aus der Modulübersicht auf den nächsten Seiten entnehmen. Dort steht auch, welche dieser Leistungsnachweise relevant für die Abschlussnote sind. Weitere Infos findet ihr in der Prüfungsordnung.

Vor oder bei der ersten Sitzung jeder Veranstaltung müssen die Lehrenden bekannt geben, welchen Leistungsnachweis sie verlangen. Im Zweifel fragt direkt nach! Aus diesem Grund ist generell wichtig, gerade in der ersten Veranstaltung anwesend zu sein. Hier werden außerdem Termine festgelegt, Plätze und ggf. Referatsthemen vergeben und der Zugang zu online hinterlegten Präsentationen und Informationen herausgegeben.
Die meisten Noten, die in den Modulen erbracht werden, gehen unterschiedlich gewichtet in die Bachelornote ein. Einige Module (zum Beispiel die naturwissenschaftlichen Grundlagenfächer Chemie, Mathe und Physik) müssen nur bestanden werden und gehen nicht in die Abschlussnote mit ein. Andere Fächer, die für das Studium der Landschaftsökologie von besonderer Relevanz sind, gehen sogar doppelt gewichtet in die Bachelornote ein. Zu diesen Fächern gehören meist die Fächer die auch mehr credit Points geben wie Zoologische Formenkenntnis unf Tierökologie, Landschaft und Lebensräume und raum und Umweltplanung. Auch das steht alles in der Prüfungsordnung... Merkst du was? Bevor du jetzt die Prüfungsordnung direkt zu deiner allabendlichen Bettlektüre machst, lies erst mal entspannt weiter...

\subsection*{Definition}
Die Landschaftsökologie beschäftigt sich mit der Beziehung zwischen belebter und unbelebter Umwelt, wobei qualitative und quantitative Analysen von Ökosystemen und deren Lebensgemeinschaften im Zentrum der Betrachtung stehen. Es werden die Stoff- und Energiekreisläufe in ihnen und die Wechselbeziehungen untereinander in räumlicher und zeitlicher Entwicklung betrachtet.

Unser Institut besitzt fachliche Schwerpunkte auf den Gebieten der Angewandten Landschaftsökologie/Ökologische Planung, Biozönologie, Hydrologie und Bodenkunde, Klimatologie, Ökosystemforschung sowie Waldökologie, Forst- und Holzwirtschaft. Dies sind auch die sechs Arbeitsgruppen am ILÖK. Im Vergleich zur Geoökologie, die stärker abiotisch ausgerichtet ist, gibt es bei der Landschaftsökologie in etwa ein Gleichgewicht zwischen abiotischem und biotischem Themenspektrum (biotisch = belebte Natur; abiotisch = unbelebte Natur z.B. Gestein, Wasser-Chemie, etc.).

\section*{Historisches}
Ab 1987 konnte man in Münster am Institut für Geographie Diplom"=Geographie mit den Studienschwerpunkten Landschaftsökologie (LÖK) oder Sozialgeographie (SOZ) studieren. Das Lehrangebot war dem heutigen in weiten Teilen ähnlich. In jenen Tagen gab es noch keinen NC, und somit war der Andrang an Studierenden groß. Für die Landschaftsökologie−Geos bedeutete dies, dass nicht alle Studierenden das volle Lehrangebot nutzen konnten (z.B. Geländepraktika, Hauptseminare), sondern nur beschränkt auszuwählen vermochten.

1992 wurde dann das Große Schwert gezückt, und der alte Studiengang Geographie wurde in zwei völlig unterschiedliche Teile geschlagen: (Sozial-) Geographie und Landschaftsökologie als eigenständige Diplom"=Studiengänge. Jetzt kam auch Meister Numerus Clausus vorbei, um der BewerberInnen"=Flut Herr zu werden. 1994 ist das alte Institut für Geographie (IfG) aufgelöst worden und es entstanden das (neue) Institut für Geographie (IfG), das Institut für Geoinformatik (IfGI) und das Institut für Landschaftsökologie (IfL, heute ILÖK), sowie eine verbindende Betriebseinheit für die Verwaltung.
An der Uni Münster blickt man daher schon auf eine gewisse landschaftsökologische Tradition zurück. Der Studienverlauf hat sich mit den Jahren einige Male geändert und das Diplom Landschaftsökologie ist inzwischen durch die neuen modularen Studiengänge Bachelor und Master ersetzt worden.

\section{Modulübersicht}
\begin{longtable}{p{0.75\textwidth} p{0.1\textwidth} p{0.1\textwidth}}
 & Typ & ECTS \\ 
\textbf{B1 Geologie} & & \textbf{5}\\ 
Die Erde & V & 3\\
Gesteinskunde & Ü & 2\\
& &\\
\textbf{B2 Bodenkunde}& & \textbf{5}\\
Einführung in die Bodenkunde & V & 2\\
Geländepraktikum Boden & Ü & 3\\
& &\\
\textbf{B3 Allgemeine Biologie}& & \textbf{5}\\
Biologie II & V &5\\
& &\\
\textbf{B4 Botanische Formenkenntnis} & & \textbf{5}\\
Bestimmungsübungen Botanik & Ü & 5\\
& &\\
\textbf{B5 Zoologische Formenkenntnis und Tierökologie} & &\textbf{10}\\
Enführung in die Tierökologie & V & 2\\
Bestimmung Wirbelloser & Ü & 4\\
Bestimmung von Wirbeltieren & Ü & 4\\
& &\\
\textbf{B6 Chemie für Naturwissenschaftler} & & \textbf{10}\\
Chemie für Naturwissenschaftler & V & 4\\
Chemie Übung & Ü & 2\\
Einführungspraktikum & P & 4 \\
&&\\
\textbf{B7 Mathematik} & &\textbf{5}\\
Mathe für Naturwissenschaftler & V & 2\\
Mathematik & Ü & 3\\
&&\\
\textbf{B8 Physik} & & \textbf{5}\\
Physik für Löks & V & 3 \\
Experimentalphysik für Löks & Ü & 2\\
&&\\
\textbf{B9 Vegetationsökologie} & & \textbf{5}\\
Einführung in die Vegetationsökologie & V & 2 \\
Geländeübung Vegetationsökologie & Ü & 3\\
&&\\
\textbf{B10 Exkursionen}&&\textbf{8}\\
Mindestens 8 Tage plus begleitendes Seminar (Wahlpflicht) & Exk. & 8\\
Mindestens 12 Tage (Wahlpflicht) & Exk. & 8\\
&&\\
\textbf{B11 Allgemeine Studien I (Arbeitstechniken)}& & \textbf{5}\\
Studien- und Arbeitstechniken & S & 2\\
Tutorium zu Studien- und Arbeitstechniken & T & 1\\
Fachenglisch & S & 1\\
Berufliche Orientierung & S & 1\\
&&\\
\textbf{B12 Allgemeine Studien II (Projektmanagement)}&& \textbf{5}\\
ersetzbar durch Modul B22 (Ergänzungsmodul III)\\ 
Grundlagen des Projektmanagements & S & 2\\
Praxisprojekt & S & 3\\
&&\\
\textbf{B13 Klimatologie} && \textbf{5}\\
Einführung in die Klimatologie & V & 2\\
Übung Klimatologie & Ü & 3\\
&&\\
\textbf{B14 Hydrologie} && \textbf{5}\\
Einführung in die Hydrologie & V & 2\\
Übung Hydrologie &Ü&3\\
&&\\
\textbf{B15 Biogeochemie} && \textbf{5}\\
Einführung in die Wasser- und Bodenchemie & V & 2\\
Laborpraktikum Wasser- und Bodenchemie & P & 3\\
&&\\
\textbf{B16 Landschaft und Lebensräume} & & \textbf{10}\\
Ökosysteme und Lebensgemeinschaften & V & 2\\
Landschaftszonen der Erde & V & 2\\
Landschaftsökologische Übung & Ü & 6\\
&&\\
\textbf{B17 Geostatistik} && \textbf{5}\\
Geostatistik & V & 2\\
Geostatistik & Ü & 3\\
&&\\
\textbf{B18 Geoinformatik} && \textbf{10}\\
Einführung in die Geoinformatik & V & 2\\
Geoinformatik für Geowissenschaftler & Ü & 3\\
Angewandte Karthographie & Ü & 5\\
&&\\
\textbf{B19 Methoden der Landschaftserfassung}&&\textbf{10}\\
Einführung in die Fernerkundungsmethoden in den Geowissenschaften (Pflicht) & V & 2\\
Fernerkundungsmethoden in den Geowissenschaften (Wahlpflicht) & Ü & 3\\
GPS Methoden & Ü & 3\\
Biotop- und FFH-Lebensraumkartierung (Wahlpflicht) & V + Ü & 3\\
Boden und Wasseranalytik (Wahlplicht) & Ü & 3\\
Wissenschaftliches Rechnen (Wahlplicht) & Ü & 3\\
Tierökologische Erfassungsmethoden (Wahlplicht) & Ü & 3\\
GIS-Grundkurs (Wahlplicht)  & Ü & 3\\
evtl. weitere Angebote\\
&&\\
\textbf{B20 Ergänzungsmodul I} && \textbf{5}\\
Verschiedene Veranstaltungen & & 5\\
&&\\
\textbf{B21 Ergänzungsmodul II} && \textbf{5}\\
Verschiedene Veranstaltungen & & 5\\
&&\\
\textbf{B22 Ergänzungmodul III} && \textbf{5}\\
möglicher Ersatz für B12 (Allgemeine Studien II (Projektmanagement))\\
Verschiedene Veranstaltungen & & 5\\
&&\\
\textbf{B23 Raum- und Umweltplanung}&&\textbf{10}\\
Grundlagen der Raumplanung & V & 2\\
Grundlagen der Raumplanung & Ü & 3\\
Grundlagen der Ökologischen Planung & V & 2\\
Grundlagen der Ökologischen Planung & Ü & 3\\
&&\\
\textbf{B24 Angewandte Landschaftsökologie} && \textbf{15}\\
Studienprojekt & & 18\\
&&\\
\textbf{B25 Berufsorientiertes Praktikum} & & \textbf{10}\\
&&\\
\textbf{B26 Wissenschaftliches Arbeiten} & & \textbf{5}\\
&&\\
\textbf{B27 Bachelorarbeit} && \textbf{12}\\
\end{longtable}

\subsection*{Naturwissenschaftliche Grundlagen}
\subsubsection*{Mathematik und Physik}
...sind nur jeweils ein Semester lang als Pflichtkurse zu erledigen. Ein Mathe"=Modul (Vorlesung+Übung+Hausaufgaben+Klausur) steht dann schon im 1. Semester an. Ebenso darf man sich im 1. Semester Physik gönnen (Vorlesung+Übung+Klusur). Die Physikübungen werden als Block in den Semesterferien angeboten und sind wirklich keine Hürde. Die Physik-„Klausur“ kommt getarnt als Multiple-Choice-Test mit Kopfrechnen daher – wir erinnern uns an die theoretische Führerscheinprüfung. Die Grundlagen, die man hier lernt, können einem im weiteren Studium aber immer mal wieder begegnen.

\subsubsection*{Chemie}
...ist natürlich auch mit dabei. Hier sind die Vorlesungen und eine Übung im Lehrplan vorgesehen. Eine Eingangsklausur entscheidet über die Zulassung zum Praktikum; dieses findet in den Winter- oder Sommersemesterferien statt. Den Abschluss des zweiwöchigen, ganztägigen Praktikums krönt eine Abschlussklausur. Für Leute ohne jegliche Vorbildung in Chemie: Keine Panik, aber bitte frühzeitig mit dem Lernen anfangen, dann ist Chemie durchaus zu bewältigen! Später werden die Kenntnisse im neuen Modul Biogeochemie vertieft und spezialisiert.

\subsubsection{Biologische Grundlagen}
Zunächst gibt es eine allgemeine Vorlesung gemeinsam mit Biologen, nämlich Biologie II. In der Vorlesung stehen ein Überblick über das Tier- und Pflanzenreich, das Thema Evolution und ein wenig Verhaltensbiologie auf dem Programm. Das Modul wird mit einer Multiple Choice-Klausur abgeschlossen. Die Vorlesung Biologie I muss von den LÖK-Studis nicht belegt werden, da es dort vornehmlich um Genetik und Zellbiologie geht. Solch kleine, komplizierte Dinge sind sicher interessant, aber für uns Löks (zum Glück?!) meist nicht so relevant...

Hinzu kommen die botanischen und zoologischen Bestimmungsübungen, in denen ihr Pflänzchen und Krabbeltiere unter die Lupe nehmt. In der Zoologie stehen ökologisch relevante Tiergruppen wie zum Beispiel Vögel, Heuschrecken, Laufkäfer und Amphibien auf dem Programm. In der Botanik werden die 10 bis 15 wichtigsten Pflanzenfamilien durchgenommen. Man lernt in den Übungen, wie man beim Bestimmen vorgeht, die Erweiterung und das Training der eigenen Artenkenntnisse sind überwiegend Teil des Selbststudiums. Zur botanischen Bestimmungsübung gehört das Anlegen eines Herbariums.

\subsubsection*{Landschaftsökologische Grundlagen}
Diese meist physiogeographisch und ökologischen Grundlagen werden durch verschiedene einführende Vorlesungen vermittelt, die von Vegetations- und Tierökologie über Bodenkunde, Klimatologie und Hydrologie bis hin zur Landschaftsökologie reichen (siehe Übersicht). Diese Module stellen den Kern des Studiums dar. Zu den Vorlesungen finden meist im darauf folgenden Sommersemester (oder auch parallel) Geländepraktika bzw. Übungen statt, in denen bestimmte praktische Anwendungen dieser Teildisziplinen vorgestellt und von den Studierenden selbst durchgeführt werden. Eure Untersuchungsergebnisse fasst ihr in Form von Protokollen zusammen. Meist bringen die Geländeübungen viel Spaß und eine gute Abwechslung zum Hörsaal oder Computerraum!

\subsubsection*{Geoinformatik und Geostatistik}
Grundlage bildet die Vorlesung \enquote{Einführung in die Geoinformatik} mit der dazugehörigen Übung. Zusätzlich ist dann noch die Übung \enquote{Angewante Kartographie}. Weiterer Bestandteil aus der Geoinformatik ist die Vorlesung „Geostatistik“ mit der dazugehörigen Übung. Alle Übungen werden am Computer durchgeführt - aber keine Sorge: sie sind alle kein Hexenwerk und gut machbar.

\subsection*{Wahlmöglichkeit}
Obwohl der Bachelorstudiengang wesentlich durchstrukturierter ist und weniger Wahlmöglichkeiten als der Diplomstudiengang lässt, geben zwei bzw. drei Ergänzungsmodule die Möglichkeit sich je nach Interessenlage mit anderen Fächern zu beschäftigen und eigene Initiative zu zeigen. Diese sollten dennoch einen Bezug zur Landschaftsökologie haben und daher mit den DozentInnen des ILÖK abgesprochen werden. Diese können euch auch beraten, wie sinnvoll eure Wahl des Ergänzungsmoduls ist. Für jedes der drei Ergänzungsmodule \` 5 Leistungspunkte kann man sich ein Fach aussuchen. Wichtig ist hierbei mit den Ergänzungsmodulen rechtzeitig (am besten ab dem 3. Fachsemester) anzufangen, da man für einige Module mehrere Semester einplanen muss. Eine Übersicht über mögliche Ergänzungsmodule findet ihr unter:

\url{http://www.uni-muenster.de/imperia/md/content/}\\\url{landschaftsoekologie/frontoffice_geoloek/}\\\url{info_bsc_loek_ergaenzungsmodule_6.2.09.pdf}

Wenn ihr euch unsicher seid, kommt doch mal in der Fachschaft, im Front Office oder bei eurem Fachstudienberater vorbei und lasst euch beraten. Beispiele für wählbare Module sind Rechtswissenschaften, Politikwissenschaften, Pädagogik/ Erziehungswissenschaften, Wirtschaftswissenschaften, Geoinformatik, Geographie und ggf. Sprachkurse. Die Lehrveranstaltungen variieren je nach Modul und sind ebenso wie die Prüfungsmodalitäten und die Punkte-Vergabe vom jeweiligen Fach festzulegen. Aus organisatorischen Gründen stehen einige Veranstaltungen jedoch nicht in HISLSF.

\subsubsection*{Angewandte Landschaftsökologie}
Das Studienprojekt ist eine Gruppenarbeit (à ca. 10 Personen). Sie soll auf die Bachelorarbeit vorbereiten, an selbständiges Arbeiten heranführen und gleichzeitig das Arbeiten im Team schulen. So ist es beispielsweise möglich, dass sich die Studierenden durch die praktische Bearbeitung von Teilflächen an der Erstellung von Umweltverträglichkeitsprüfungen beteiligen, indem sie floristische und faunistische Kartierungen, Boden- oder Wasseruntersuchungen und klimatologische Messungen vornehmen, um das jeweilige Gebiet charakterisieren zu können. Meist bietet jede des sechs Arbeitsgruppen ein Studienprojekt an. Die Themen werden zu Beginn des vierten Semesters vorgestellt. Das Studienprojekt unterscheidet sich deutlich von den anderen Veranstaltungen und fordert u.a. etwas soziales Geschick, also TEAM-Work (= Toll Ein Anderer Macht’s).

\subsection*{Berufsorientiertes Praktikum}
Es sollen (mindestens) 6 Wochen ganztägiges, außeruniversitäres Praktikum geleistet werden. Du kannst das Praktikum zum Beispiel in der Verwaltung (kommunal, landesweit usw.), Planungsbüros, Biostationen oder sonstigen Unternehmen der freien Wirtschaft machen. Die Stelle sollst du dir selber suchen, aber vielleicht findest du auch wen, der dir eine gute Stelle empfehlen kann. Es ist durchaus möglich, die Praktikumszeit auf zwei Stellen aufzuteilen, jedoch wird von vielen Stellen eine Mindest-Dauer von vier Wochen gefordert, so dass du dann vielleicht 8 Wochen leisten musst. Das ist aber alles andere als schlimm, denn diese Praktika gelten als „Eingangstüren“ in einen Job. Also lieber mehr, denn weniger machen!

\subsection*{Wissenschaftliches Arbeiten und Bachelorarbeit}
Das Modul Wissenschaftliches Arbeiten soll als direkte Vorbereitung zur Bachelorarbeit dienen, wobei es von der Themenfindung über den Rechercheprozess bis zum eigentlichen Schreiben geht. Die Prüfungsordnung schreibt als Voraussetzung 100 Leistungspunkte vor um mit der Bachelorarbeit zu beginnen. Weiterhin steht dort: „Die Studierenden sind in der Lage, eine konkrete Fragestellung aus dem Gebiet der Land- schaftsökologie fachlich kompetent mit wissenschaftlichen Methoden selbständig und in vor- gegebener Frist zu bearbeiten.“. Das Thema oder Themengebiet mit dem dazugehörigen Betreuer/Prüfer kann man sich frei aussuchen. Die Bearbeitungszeit beträgt ca. 9 Wochen und die Arbeit erbringt 12 Leistungspunkte.

\subsection*{Exkursionstage}
Neben Vorlesungsbesuchen und Geländepraktika gibt es im B.Sc. LÖK als weiteres Highlight die Exkursionstage. Es müssen insgesamt 8 Exkursionstage mit einem Vorbereitungsseminar für eine mehrtägige Exkursion oder 12 Exkursionstage nachgewiesen werden. Diese können entweder als mehrere kleinere Exkursionstage eingebracht werden oder man nimmt an einer „großen“ Exkursion von 1 bis 2 Wochen teil. Plätze sind begehrt, daher sollte man sich immer gut an der Exkursionspinnwand im Foyer über Exkursionsangebote informieren. Die Anmeldung erfolgt in der Regel über einen Listeneintrag bei den DozentInnen. In den meisten Fällen müssen im Rahmen der Exkursionen Referate gehalten werden und in der Nachbereitung werden Exkursionsberichte bzw. Protokolle als Leistungsnachweis verlangt. In der Regel machen Exkursion viel Spaß und das im Semester Gelernte kann  praktisch angewendet werden!

\subsection*{Verwandte Studiengänge}
Ökologie-Studiengänge unterschiedlichster Art schießen wie Pilze aus dem Boden. Inzwischen haben sich einige Studiengänge etabliert. Neben Landschaftsökologie in Münster gibt oder gab es beispielsweise noch:
\begin{itemize}
  \item Landschaftsökologie (in Greifswald, Oldenburg)
  \item Geoökologie (in Bayreuth, Braunschweig, Freiberg, Karlsruhe und Potsdam)
  \item Ökologie (in Essen)
  \item Umweltwissenschaften (in Greifswald, Oldenburg ...) etc.
\end{itemize}
Zusätzlich gibt es weitere Studiengänge, in denen planerische und ökologische Zusammenhänge gelehrt werden, die denen der Landschaftsökologie in Münster oft ähnlich sind. 

\subsection*{Berufsfelder, der Blick in die Zukunft}
Für Landschaftsökologen gibt es kein einheitliches Berufsbild, wie zum Beispiel für Juristen, Chemiker oder Lehrer. Die Einsatzmöglichkeiten auf dem Arbeitsmarkt sind vielfältig, einige sind bei
\begin{itemize}
  \item (Landschafts-) Planungsbüros oder Beratungsunternehmen
  \item diversen Umweltbehörden auf verschiedenen Ebenen von der Gemeinde bis zur EU
  \item Einrichtungen des Natur- und Umweltschutzes, sowohl staatlich als auch privat, z.B. Biologische Stationen, Naturschutzverbände, Museen oder Naturschutzzentren
  \item Fachabteilungen oder -stellen in Unternehmen, in denen fächerübergreifende Kompetenz gefragt ist, z. B. bei Versicherungen
  \item{Forschung}
\end{itemize}
Natürlich können sich auch Landschaftsplanern, Landschaftspflegern,\\ Landschaftsarchitekten, Geoökologen, Geographen, Geophysikern, Biologen, Umwelttechnikern etc. auf mögliche Stellen bewerben. Doch kein Grund zum Verzweifeln: durch die Interdisziplinarität unseres Studiengangs ergeben sich vielfältige Möglichkeiten und „grüne“ Berufe sollen ja bekanntlich Zukunft haben!

Und eins sollte man bei all der Zukunftsplanung nicht außer Acht lassen: 

\textbf{Es bringt einfach Spaß, Landschaftsökologie zu studieren!}

  \cleardoublepage
  \chapter{Bachelor of Science Geoinformatik}
\lohead{\footnotesize{Geographie - Landschaftsökologie - \textbf{Geoinformatik}}}
\rehead{\footnotesize{Geographie - Landschaftsökologie - \textbf{Geoinformatik}}}

\section{Allgemeines}

\emph{"`Fabian W. war mit 3 Freunden hier: Institute for Geoinformatics (ifgi)"'}.\\
So einen ähnlichen Satz wird wohl jeder schon einmal in einem sozialen Netzwerk gelesen haben. Und wenn man mal nicht weiß wie man zur WG-Party am Rudolf-Harbig-Weg kommt, hilft einem schnell das Smartphone. Auch der beste Weg zum Institut für Landschaftsöklogie ist mit Hilfe von GoogleMaps direkt gefunden. Und aus Autos sind Navigationsgeräte sowieso nicht mehr wegzudenken.\\
Meistens macht man sich schon gar keine Gedanken mehr, wie und warum diese Technologien funktionieren, weil sie heutzutage bereits ein fester Bestandteil unseres Lebens geworden sind. Doch irgendwo müssen die ganzen Informationen herkommen und irgendjemand muss diese Daten so aufbereiten, dass das Handy sie korrekt anzeigen kann, oder das Navi einen nicht an der nächsten Ecke in den Aasee fahren lässt.\\
Die Aufbereitung dieser Daten, die Sicherung der Datenqualität, und die Bereitstellung gehören zu eurem Aufgabengebiet. Und der Weg von einem Satellitenbild zu einem abstrakten dynamischen Graphen mit Straßeninformationen ist lang. Für die korrekte Einbindung aktueller Stau- und Wetterdaten sind sowohl Kenntnisse aus den Geowissenschaften als auch aus der Informatik notwendig. Und ein Routenplaner ist nur ein Beispiel aus der großen und ständig wachsenden Welt der Geoinformatik.\\
Geoinformatik ist anwendungsbezogen und kann als Schnittstelle zwischen etablierten geowissenschaftlichen Studiengängen (Geographie, Geologie, Landschaftsökologie, etc.) und den naturwissenschaftlich ausgerichteten Fächern (Mathematik, Informatik) betrachtet werden. Die Grenzen zwischen den Anwendungsgebieten verschwimmen oftmals und durch das interdisziplinäre Studium wird eine hohe Flexibilität erreicht, so dass ein ausgebildeter Geoinformatiker in einem breiten Aufgabenspektrum einsetzbar ist. Er verfügt über die benötigten Kenntnisse in den Geowissenschaften und der Informatik und so stehen ihm -- im Gegensatz zu hoch spezialisierten Abgängern anderer Studienfächer -- zahlreiche Möglichkeiten des beruflichen Werdegangs offen. Nicht ungern wird in unserem Institut immer wieder auf "`die 80 \%"' verwiesen.\\
Diese magische Zahl soll der Anteil der Fragestellungen sein, welche einen Raumbezug hat. Bei diesen Fragestellungen kann ein Geoinformatiker helfen eine bessere Antwort zu suchen. Typische Berufsfelder lassen sich in drei grobe Bereiche trennen. Behörden zeigen großes Interesse an jemandem, der bei Stadtplanung und mehr softwaretechnisch aushelfen kann. An den Universitäten wird in diesem jungen Forschungszweig immer nach neuen Wissenschaftlern gesucht. Und in der freien Wirtschaft werden die Anwendungen entwickelt, die woanders erfolgreich zum Einsatz kommen. Diese Bereiche sind aber auch wieder verzahnt, und für die möglichst reibungslose Kommunikation könnten und sollten an beiden Enden Geoinformatiker sitzen, denn nur wir verstehen sowohl die Fachsprache und Denkweise der Informatiker als auch die der Geowissenschaftler.

\section*{Geoinformatik studieren}
In Münster Geoinformatik zu studieren bedeutet weit mehr als Vorlesungen und Klausuren. Münster ist eine Stadt der Studentinnen und Studenten und es ist praktisch jeden Abend etwas los. Ist gerade mal keine Mathe- oder Geoparty, kann man sich in der Jüdefelder oder am Hawerkamp die Zeit vertreiben.

Aber zurück zum Studium – BSc. Geoinformatik: Der Bachelor in Geoinformatik hat natürlich das bereits beschriebene Grundgerüst aus Modulen, Zensuren, Creditpoints und der abschließenden Bachelorarbeit. In den ersten beiden Semestern solltet ihr nach Plan möglichst den Grundstein in Mathematik und Informatik, sowie in den geoinformatischen Disziplinen, wie Geostatistik legen. Im zweiten Studienjahr erwarten euch die Grundbausteine in den Geowissenschaften. Die Kenntnisse in Informatik und Geoinformatik werden vertieft. Im letzten Studienjahr entscheidet ihr euch je nach Interesse für einige der angebotenen Seminare und arbeitet somit auf eure Bachelorarbeit und euren akademischen Abschluss hin.

Wenn ihr euch mit alten Freunden aus der Schule oder neuen Freunden an der Uni unterhaltet, die geisteswissenschaftliche Fächer studieren, werdet ihr schnell merken, dass Übungszettel charakteristisch für euer Studium sind. Das sind wöchentliche Hausaufgaben, die es mit 2 oder 3 Leuten zu bearbeiten gilt. Auch hierbei lernt man nette Leute kennen und kann weitere Kontakte knüpfen. Eure vorhin erwähnten Kommilitonen müssen dafür Unmengen an Hausarbeiten schreiben während ihr eure erste und einzige Hausarbeit im BSc. Geoinformatik im vierten Semester im Rahmen des Wahlfachs Humangeographie verfassen müsst. Diese ist auch die einzige methodisch-praktische Vorbereitung auf eure spätere Abschlussarbeit. Deswegen gibt es auch noch das "`General Studies/Allgemeinen Studien"'-Modul, dessen Kurse teilweise vorgegeben sind, aber auch selbst gestaltet werden können. So könnt ihr beispielsweise eine neue Sprache lernen oder eben einen Kurs wählen, der euch methodisch auf eure Abschlussarbeit vorbereitet. Die meisten Veranstaltungen gehen mit in die Endnote ein, das heißt, dass ihr möglichst alle Veranstaltungen gut abschließen solltet, aber andererseits fällt eine vergeigte Prüfung nicht so stark ins Gewicht.

Wenn alle Module geschafft sind und die Bachelorarbeit geschrieben ist, hast du die Möglichkeit auch noch den Master of Science Geoinformatics zu absolvieren und so eine noch höhere Qualifikation zu erlangen. Dazu sind einige Zugangsvoraussetzungen, wie ein Englischtest, erforderlich (Der Master-Studiengang wird komplett auf englisch bestritten). Dazu und zu allen anderen Fragen bzgl. des Masters wirst du aber in deinen drei Jahren Bachelorstudium die passenden Antworten bekommen. Ihr könnt euch jederzeit gerne telefonisch oder per Mail (davon wirst du eine ganze Menge in deinem Studium schreiben) an uns wenden. Am besten ist es aber immer, wenn ihr persönlich bei uns vorbei kommt. Wo ihr uns findet ist im Abschnitt "`Fachschaft"' beschrieben. Wenn wir euch einmal nicht weiterhelfen können, könnt ihr euch gerne an unseren Bachelorstudienberater Prof. Christian Kray wenden.

Im Folgenden sind alle Module aufgelistet, die ihr in eurem Studium zu absolvieren habt. Diese könnt ihr ebenfalls auf der ifgi-Webseite (\url{http://www.bachelor-geoinformatik.de/inhalte-des-studiums /pruefungsordnung}) finden.

\newpage

\section{Modulübersicht}

%Dies ist die vorläufige Modulübersicht des Bsc. Geoinformatik. Uns lag zum Zeitpunkt des Drucktermins noch nicht die finale Version vor, es sollte sich allerdings nicht mehr viel ändern. Bei Fragen wendet euch direkt an uns!

\begin{longtable}{p{0.7\textwidth} p{0.12\textwidth} p{0.1\textwidth}}
 & Typ & ECTS \\
\textbf{Geoinformatik 1 -- Grundlagen} & & \textbf{5}\\
Einführung Geoinformatik & V & 3\\
Einführung Geoinformatik & Ü & 2\\
& &\\
\textbf{Geoinformatik 2 -- Angewandte Kartographie} & & \textbf{7}\\
GIS Grundkurs & Ü & 2\\
Angewandte Kartographie & Ü & 5\\
& &\\
\textbf{Geoinformatik 3 -- Geostatistik} & & \textbf{5}\\
Einführung in die Geostatistik & V & 2\\
Einführung in die Geostatistik & Ü & 3\\
&&\\
\textbf{Geoinformatik 4 -- Dynamische räumliche Prozesse} & &\textbf{5}\\
Einführung in die Modellierung dynamischer räumlicher Prozesse & V & 2 \\
Einführung in die Modellierung dynamischer räumlicher Prozesse & Ü & 3\\
&&\\
\textbf{Geoinformatik 5 -- Fernerkundung} && \textbf{5}\\
Einführung in die Fernerkundung & V & 2\\
Einführung in die Fernerkundung & Ü & 3\\
&&\\
\textbf{Geoinformatik 6 -- Interoperabilität}&& \textbf{10}\\
Geodateninfrastrukturen und Geoinformationsdienste (SII) & V & 2\\
Geodateninfrastrukturen und Geoinformationsdienste (SII) & Ü & 3\\
Reference Systems for Geoinformation & V & 2\\
Reference Systems for Geoinformation & V & 3\\
&&\\
\textbf{Geoinformatik 7 -- Softwareentwicklung}&& \textbf{15}\\
Geosoftware I & P & 6\\
Geosoftware II & P & 9\\
&&\\
\textbf{Geoinformatik 8 -- Perspektiven}&& \textbf{8}\\
Geoinformatik Seminar & S & 3\\
Ausgewählte Probleme der Geoinformatik (Wahlpflicht) & V/Ü/S & 5\\
&&\\
\textbf{Mathematik}&& \textbf{20}\\
Analysis für Informatiker  & V\,+\,Ü & 10\\
Lineare Algebra für Informatiker & V\,+\,Ü & 10\\
&&\\
\textbf{Informatik 1: Grundlagen der Programmierung} & & \textbf{12}\\
Informatik 1 & V\,+\,Ü & 5+4\\
Java Programmierkurs & V\,+\,Ü & 3\\
&&\\
\textbf{Informatik 2: Algorithmen und Datenstrukturen} & & \textbf{12}\\
Informatik 2 & V\,+\,Ü & 5+4\\
C/C++ Programmierkurs & V\,+\,Ü & 3\\
&&\\
\textbf{Informatik 2: Datenbanken} & & \textbf{7}\\
Datenbanken & V\,+\,Ü & 4+3\\
&&\\
\textbf{Informatik 4: Software-Entwicklung}& &\textbf{6}\\
Software-Entwicklung & V\,+\,Ü & 4+2\\
&&\\
\textbf{Informatik 5: Vertiefung} & & \textbf{6}\\
Diskrete Strukturen (Wahlpflicht) & V\,+\,Ü & 6\\
Computergrafik (Wahlpflicht) & V\,+\,Ü & 6\\
Computer Vision (Wahlpflicht) & V\,+\,Ü & 6\\
Algorithmische Geometrie (Wahlpflicht) & V\,+\,Ü & 6\\
&&\\
\textbf{Geowissenschaften 1: Physische Geographie} & & \textbf{10}\\
Physische Geographie & V & 5\\
Geländeübung Physische Geographie & Ü & 5\\
&&\\
\textbf{Geowissenschaften 2a: Humangeographie} & & \textbf{10}\\
Einführung Humangeographie & V & 5\\
Eine Übung zur Humangeographie & Ü & 4\\
Exkursion & & 1\\
&&\\
\textbf{Geowissenschaften 2b: Orts-, Regional- und Landesplanung} & & \textbf{10}\\
Grundlagen der Raumplanung & V & 3\\
Einführung in die räumliche Planung & S & 6\\
Exkursion & & 1\\
&&\\
\textbf{Geowissenschaften 3a: Vertiefung Geologie} & & \textbf{5}\\
Die Erde & V & 5\\
&&\\
\textbf{Geowissenschaften 3b: Vertiefung LÖK} & & \textbf{5}\\
Klimatologie\,/\,Hydrologie\,/\,Vegetations- oder Tierökologie& V\,+\,Ü & 5\\
&&\\
\textbf{General Studies} & & \textbf{18}\\
Präsentation, Rhetorik, Fremdsprachen & V/Ü/S/P & 8\\
Projektplanung und Projektmanagement & Ü & 5\\
Projekt & P & 5\\
&&\\
\textbf{Thesis} & & \textbf{14}\\
Bachelor-Abschlussarbeit & & 12\\
Blockkurs Vorbereitung Bachelor-Abschlussarbeit & S & 2\\

\end{longtable}

Alle Angaben ohne Gewähr. (Vergleiche \url{http://www.bachelor- geoinformatik.de/inhalte-des-studiums/pruefungsordnung})

\newpage

\section*{Fachschaft}
Seit dem Wintersemester 2001/02 hat unser Studiengang eine eigene Fachschaft, die sich um die speziellen Fragen und Wünsche der Geoinformatikstudierenden kümmert. Die Fachschaft besteht aus gewählten Vertretern der Studenten eines Faches. In der Fachschaft versuchen wir die Interessen der Studierenden gegenüber der Hochschule zu vertreten und Konflikte und Unklarheiten zu beseitigen. Also sind wir eine Anlaufstelle für Studierende und Dozenten gleichermaßen, die die Kommunikation unter den Studierenden und mit den Dozenten fördert, z.B. durch übergreifende Veranstaltungen. Die Fachschaft will die Interessen der Studierenden nach bestem Wissen und Gewissen kundtun, vertreten und verteidigen. Dies gilt natürlich besonders für euch Studienanfänger. Dieses Heft zum Beispiel wurde von Fachschaftsmitgliedern entworfen und wir hoffen, euch damit einen ersten Überblick über einen innovativen und jungen Studiengang gegeben und vielleicht sogar euer Interesse am Mitwirken in der Fachschaft geweckt zu haben. Abschließend möchten wir euch auf diese beiden Homepages aufmerksam machen:

\begin{center}
\textbf{\url{ifgi.uni-muenster.de}}\\
\end{center}

ifgi steht für "`Institut für Geoinformatik"'. Hier findet ihr alle Informationen zum Studiengang wie die Studienordnung, eine Übersicht über die Module oder ein einfaches FAQ. Die Website bietet ausreichend Informationen bezüglich aller Kurse und Vorlesungen die am ifgi angeboten werden.

\begin{center}
\textbf{\url{geofs.uni-muenster.de}}\\
\end{center}

Die Homepage der Fachschaft. Sie wird von den Fachschaftlern ständig aktualisiert und ergänzt. Wir bemühen uns, den Studierenden, gerade Studienanfängern, mit dieser Homepage das Studium zu erleichtern. Also schaut euch die Seite an und zögert bei Fragen nicht, eine E-Mail an einen Vertreter oder die Fachschaft direkt zu schicken. Uns liegt viel daran, euch zu helfen und eventuelle Fragen oder Probleme schon früh zu klären.

Zum Schluss wünschen wir euch ein erfolgreiches und unterhaltsames Studium!

\newpage

\section*{FAQ} % alles in bf sind die fragen
\textbf{Buäh, jetzt brummt mir aber der Kopf. Die ganzen Infos muss ich erst einmal verdauen. Wenn ich noch andere Fragen bzgl. des Studiums habe, wo seid ihr zu finden?}

Wir als Fachschaft sind direkt gegenüber des Hörsaals im Erdgeschoss des GEO1 an der Heisenbergstraße zu finden. Bitte erkundigt euch vorher auf unserer Homepage nach unseren Präsenzzeiten!\\

\textbf{Wie stark unterscheidet sich Geoinformatik von der normalen Informatik? Ich habe irgendwie die Hoffnung, dass das alles etwas praxisorientierter und nicht ganz so bieder und trocken ist...}

Es ist richtig, dass Informatik teilweise bieder und trocken ist. Es ist auch richtig, dass Mathematik ziemlich langweilig sein kann. Es ist aber auch richtig, dass ORL (Orts-, Regional- und Landesplanung) und Gesteinskunde trockene Lernfächer sind. Abgesehen davon ist das alles subjektiv. Die ersten Semester der Kerninformatik sind denen der Geoinformatik (sowie der Wirtschaftsinformatik) recht ähnlich - und immer schwierig. Die Spezialisierung findet später statt, und ist auch bei den Informatikern sicherlich spannend. Wenn man Mathematik immer langweilig fand und mit Computern nichts am Hut hat, sollte man sich Geoinformatik aus dem Kopf schlagen. Ansonsten lässt sich mit einer gehörigen Portion Sturheit jede Informatikprüfung schaffen.\\

\textbf{Wie sieht's mit Mathevorkenntnissen aus? Ich habe nur Mathematik Grundkurs gehabt. Reicht das aus?}

Es kommt vor allem auf den Willen an, sich mit dem Stoff auseinanderzusetzen. Das Schulwissen ist mit den Mathematikveranstaltungen an der Universität nicht vergleichbar; das komplette Gebiet wird noch mal vollständig aufgerollt. Der frühere Unterschied zwischen Grundkurs und Leistungskurs ist fast gar nicht mehr vorhanden. Die Mathematik-Vorlesungen sind dennoch sehr anspruchsvoll und zeitintensiv, weshalb ihr hier mit völliger Mathematikphobie fehl am Platz seid. Bei uns hatten bei weitem nicht alle einen Mathe-LK. Einige waren sogar immer richtig schlecht in Mathe ;-)\\

\textbf{Gibt es auch eine Erstifahrt?}

Ja. Natürlich. Wo denkst du hin?\\

\textbf{Lohnt sich das Ersti-Wochenende? Was passiert da so im Allgemeinen?}

Klar. Man rennt die ganze Zeit im Wald rum, lernt seine Mitstudis kennen und lieben und hat jede Menge Spaß dabei. Wer nicht mitfährt ist selbst Schuld. Wer mehr wissen will, fragt einfach mal jemanden aus der Fachschaft oder schaut auf der Homepage nach.\\

%\textbf{Wo oder was ist StudLab A?}

%Ein StudLab ist ein großer Raum mit ziemlich vielen, manchmal funktionierenden Rechnern. Wir Geoinformatiker haben aber die dumme Angewohnheit, die meiste Zeit vor diesen grauen Kisten zu sitzen. Weil man im ersten Semester normalerweise noch keinen Laptop mit dem kompletten Set an Geoinformatik-Software im Regal liegen hat, trifft man sich in den StudLabs (früher wurden diese CIP-Pools genannt). Da lernt man dann auch, ICQ-Nicks Gesichtern zuzuordnen.

%Das StudLab A findet sich im ersten Stock des Instituts für Landschaftsökologie an der Robert-Koch-Straße 28, direkt neben dem Raum der Fachschaft GeoLök.

%Ausserdem gibt es noch die StudLabs B und C, diese befinden sich in zwei Containern am Ende der Robert-Koch-Str. Im ifgi gibt es im 5. Stock ebenfalls ein StudLab.\\

\textbf{Hat Geoinformatik was mit Geologie zu tun?}

Wenn man möchte. Man kann zwischen verschiedenen Themen der Landschaftsökologie und einer Vorlesung in der Gesteinskunde wählen. Die Geoinformatik hat aber nichts mit dem Berufsbild eines Geologen zu tun. Der Geologe geht raus ins Gelände und schaut, ob das Grundstück passend für ein 80-Stockwerke-Hochhaus ist (nachdem er auf die Altlasten hingewiesen hat). Ein Geoinformatiker sitzt dann später irgendwo in diesem Hochhaus. Und beide werden sich nicht mehr an die Geologievorlesung erinnern können. Außer daran, dass sie wirklich spannend war. Zu der soliden Grundbildung eines Geoinformatikers gehört sicherlich auch die Geologie. Aber genauso wichtig sind die Geographie- und Landschaftsökologieveranstaltungen.\\

\textbf{Ich hab da was gehört, wenn ich dreimal durch eine Klausur falle, bin ich exmatrikuliert, stimmt das?}

Grundsätzlich gibt es Studien und Prüfungsleistungen in eurem Studium. Klausuren die als Studienleistung zählen, können beliebig oft wiederholt werden. Tatsächlich ist diese 3-mal-Durchfallregelung (schönes Wort) mit anschließender Exmatrikulation nur bei Prüfungsleistungen der Fall. Grundsätzlich gilt: Klausuren in Kursen, welche nicht in die Modulabschlussnote einfliessen, dürfen beliebig oft wiederholt werden (Studienleistungen). Dennoch sollte man dies nicht als Freifahrtschein ansehen, da gerade diese Kurse einen wichtigen Grundstein für das weitere Studium legen.
Sollte tatsächlich der Fall eintreten, dass ihr eine wichtige Klausur zum dritten Mal nicht bestanden habt, entscheidet am Ende immer noch der Prüfungsausschuss der Geoinformatik, wer exmatrikuliert wird. Wenn ihr z.B. überall gute Noten habt. nur dreimal in Humangeographie durchgefallen seid müsst ihr euch noch nicht nach einem anderen Studienplatz umsehen, sondern solltet erstmal mit den verantwortlichen Dozenten reden.\\

\textbf{Beißen Dozenten?}

Nein, zumindest wurde noch keiner dabei beobachtet. Scheut euch also nicht auch mal nach einer Veranstaltung mit wichtigen Fragen zu eurem Dozenten zu gehen oder euch mal in eine Sprechstunde zu setzen.

Des Weiteren gibt es auch auf der offiziellen ifgi-Webseite\footnote{http://ifgi.uni-muenster.de} eine solche Fragensammlung. Für weitere Fragen solltet ihr einfach mal in der Fachschaft vorbeischauen. Die aktuellen Präsenzzeiten findet ihr immer auf unserer Webseite\footnote{https://geofs.uni-muenster.de}.

%\begin{center}
%\includegraphics[scale=0.3]{cartoonC}
%\end{center}

  \cleardoublepage
  \chapter{Zwei-Fach-BA Geographie}
\lohead{\footnotesize{\textbf{2FB Geographie} - Landschaftsökologie - Geoinformatik}}
\rehead{\footnotesize{\textbf{2FB Geographie} - Landschaftsökologie - Geoinformatik}}

\section*{Was ist eigentlich der 2-Fach-Bachelor und wofür gibt es dieses "`Ding"'?}
Seit der Umstellung der Studiengänge von Magister, Diplom und Staatsexamen auf Bachelor und Master in Münster gibt es den Zwei-Fach-Bachelor an der Westfälischen Wilhelms-Universität auch im Fachbereich Geowissenschaften. Während der "`einfache"' Bachelor den Diplom-Studiengang ablösen soll, ersetzt der 2-Fach-Bachelor den Magister-Studiengang und den Weg zum Lehramt. Nachdem der Modelversuch abgeschlossen ist startet nun die zweite Generation des neuen Zwei-Fach-Bachelors. Wenn ihr nicht sicher seid, von welchen Bachelor die Rede ist, kann man dies gut an der unterschiedlichen Schreibweise erkennen: 2-Fach-Bachelor (Abkürzung 2FB) ist der Modelversuch und Zwei-Fach-Bachelor (Abkürzung ZFB) seid ihr. 

Die meisten Studierenden wählen den Zwei-Fach-Bachelor, weil nach dem Abschluss des Bachelorstudiums der Master of Education folgen kann, welcher zum Lehramtsberuf führt. Es bietet sich aber auch die Möglichkeit, nach dem Zwei-Fach-Bachelor eben nicht einen Master of Education zu studieren, sondern sich über ein anderes Master-Studium zu spezialisieren. Ein solcher Bachelor-Abschluss wird "`polyvalent"' genannt, weil den Studierenden mehrere Wege offen stehen, sich weiter auszubilden. In welche Richtung es dabei gehen sollte, bleibt dabei dir überlassen. In den ersten Jahrgängen in Münster waren zum Beispiel einige Personen mit Ziel Journalismus oder Klimaforschung dabei und Geographie als Querschnittsdisziplin lässt sich sicherlich für viele Berufsfelder verwenden.

Ein weiterer Vorteil, den der Zwei-Fach-Bachelor bietet, ist die Möglichkeit, sich erst nach Abschluss des Bachelors zu entscheiden, in welche berufliche Richtung es gehen soll. Wer noch am Zweifeln ist, ob der Lehrerberuf die richtige Entscheidung ist, oder sich eben diese Zukunft noch offen halten möchte, zieht Vorteile aus diesem Studien-System.

\section*{Wie ist der 2-Fach-Bachelor aufgebaut?}
Der Bologna-Prozess, also die Vereinheitlichung der Studien-Systeme in Europa, hat ein Instrument hervorgebracht, mit dem sich das Vergleichen von Leistungen verbessern soll. Mit Umstellung auf die Bachelor- und Masterstudiengänge wurden die sogenannten ECTS-Punkte bzw. Credit Point, oder auch Leistungspunkte (LPs) eingeführt. ECTS heißt "`European Credits System"' und ein Credit spiegelt (i.d.R.) 30 Stunden Arbeit wieder. An der Uni sollte man sich daher daran gewöhnen, dass der Arbeitsaufwand in Leistungspunkte angegeben wird. Eine Studienleistung für die man 3 LPs bekommt wird also in der Regel weniger umfangreich sein, als eine Studienleistung für die man 5 LPs bekommt.

Im Zwei-Fach-Bachelor muss man in den sechs Semestern Regelstudienzeit 180 Punkte erreichen, im Schnitt also 30 Punkte pro Semester. Diese Punkte ergeben sich wiefolgt:
\begin{itemize}
\item 75 Punkte Fach 1
\item 75 Punkte Fach 2
\item 20 Punkte Bildungswissenschaften/Allgemeine Studien
\item 10 Punkte Bachelor-Arbeit
\end{itemize}
Pro Studienfach müsst ihr also 75 Punkte erbringen, diese werden für Veranstaltungen wie Seminare, Vorlesungen, Übungen etc. vergeben und hängen dabei wiederum von der Leistung ab, die man in dieser Veranstaltung erbringt. Referat, Planspiel, Klausur usw. geben jeweils unterschiedliche Punkte. Veranstaltungen werden wiederum in Modulen zusammengefasst und somit in einen thematischen Block gebunden.

Zwei Fächer und jeweils 75 Punkte –- das sind insgesamt 150. Zwanzig weitere Punkte werden in den Bildgungswissenschaften bzw. Allgemeinen Studien vergeben. Das ist ein Bereich, in dem man, wenn man nicht auf Lehramt studiert, "`frei"' wählen und unterschiedlichste Veranstaltungen besuchen kann, an denen man interessiert ist. Das Angebot ist enorm vielfältig und erstreckt sich vom Fremdsprachenerwerb über Rhetorik- und Vermittlungskompetenz bis hin zur Berufsvorbereitung. Leider geben viele Fächer mittlerweile verpflichtend vor, dass eine bestimmte Veranstaltung besucht werden muss. In der Geographie ist das aber nicht so.

Im Bereich der Lehramtsstudiengänge ist der Bereich der Allgemeinen Studien den Bildungswissenschaften zugeordnet. Das Studium der Bildungswissenschaften stellt im Rahmen der Lehrerausbildung einen eigenständigen Teil neben den zu studierenden (Unterrichts-) Fächern dar. Jede/r Studierende, die/der im Anschluss an das erfolgreich absolvierte Bachelor-Studium in den Master of Education-Studiengang für das Lehramt an Gymnasien und Gesamtschulen wechseln möchte, muss während der Bachelor-Phase auch Bildungswissenschaften studiert haben. Diese setzen sich aus drei Modulen zusammen: Einführung in die Grundfragen von Erziehung, Bildung und Schule (7 LPs: Vorlesung + Seminar), Orientierungspraktikum (6 LPs: Seminar + 120 Stunden Praktikum) und ein Berufsfeldpraktikum (7 LPs: Seminar + 150 Stunden Prakiktum).

Selbstverständlich kannst du auch mehr Kurse belegen als du später im Rahmen deines Bachelors anrechnen lassen kannst, sofern die Zeit es zulässt (z.B. über Blockkurse in der vorlesungsfreien Zeit). Du solltest dir bei den Allgemeinen Studien jedefalls bewusst machen, dass sie ein gutes und v.a. kostenfreies Angebot sind, bestimmte Kompetenzen zu erwerben, die andernfalls vielleicht zu kurz kommen würden.
Wahlmöglichkeiten bietet das Vorlesungsverzeichnis im Bereich \enquote{Allgemeine Studien} (bzw. \enquote{General Studies}, dieser Begriff wird an der Uni synonym verwendet); vor allem Sprachen bieten sich für diesen Bereich an. Weitere Informationen dazu findest du hier\footnote{\url{www.uni-muenster.de/studium/studienangebot/allgemeinestudien.html} oder \url{http://egora.uni-muenster.de/ew/studieren/bindata/WS1112-1_Studiengangsinfo-21.pdf}}.

Die letzten zehn Punkte erreicht man dann mit seiner Bachelor-Arbeit. Hierbei kannst du dir aussuchen, in welchem deiner beiden Fächer du sie schreiben möchtest.

\section*{Was muss ich in meinem Geographiestudium machen?}
Wie oben beschrieben, müssen im Studienfach Geographie 75 Punkte erbracht werden. Das erreicht man über das Studieren folgender Module:
\begin{itemize}
	\item Humangeographie I
	\item Physische Geographie I
	\item Geoinformatik I
	\item Humangeographie II
	\item Physische Geographie II	
	\item Geographische Erhebungs- und Analysetechniken
	\item Regionale Geographie
	\item Wahlpflichtbereich I, zwei der drei Module:
		\begin{itemize}
			\item Raumplanung/Angewandte Geographie
			\item Geoinformatik II
			\item Physische Geographie III 
		\end{itemize}
	\item Wahlpflichtbereich II, eins der zwei Module:
		\begin{itemize}
			\item Geographiediaktik I (verpflichtend für Lehramtsstudierende)
			\item Wissenschaftskommunikation
		\end{itemize}
\end{itemize}
Im ersten und zweiten Semester müssen die Module Humangeographie I, Physische Geographie I und Geoinformatik I besucht werden. Die genauen Informationen zu allen Modulen und die offizielle Beschreibung des Studiengangs findet man in der Studienordnung. Die Studienordnung ist auf unserer Homepage verlinkt (\url{http://geofs.uni-muenster.de/geoloek/doku.php?id=studieninfos:lehramt}), aber man findet sie natürlich auch auf der Seite der Zentralen Studienberatung (\url{http://zsb.uni-muenster.de/material/m858b_3.htm}).

An dieser Stelle soll nur eine kurze Übersicht mit einigen Anmerkungen zu den einzelnen Modulen erfolgen:
\section{Modulübersicht}
\textbf{Modul "`Humangeographie I"'}
	\begin{itemize}
		\item \textbf{V} "`Einführung in die Humangeographie"'
		\item \textbf{S} "`Humangeographie"'
	\end{itemize}
\emph{Die Vorlesung (4 SWS) findet im Wintersemester statt. Eine Empfehlung unsererseits: Die Klausur sollte man nicht auf die leichte Schulter nehmen und wirklich frühzeitig beginnen, dafür zu lernen. Das Seminar (2 SWS) findet nachfolgend im Sommersemester statt. Hierbei habt ihr die Wahl zwischen verschiedenen Seminarthemen wie z.B. Wirtschaftsgeographie.}\\

\textbf{Modul "`Physische Geographie I"'}
	\begin{itemize}
		\item \textbf{V} "`Einführung in die Physische Geographie"'
		\item \textbf{Ü} "`Physisch-geographische Geländeübung"'
	\end{itemize}
\emph{Die Vorlesung (4 SWS) findet im Wintersemester statt. Die Geländeübung gliedert sich in wenige Vorlesungstermine im Sommersemester und findet sonst als Exkursion an zwei Wochenenden statt. Das Modul schließt mit einer Modulabschlussprüfung (über Vorlesung und Übung) ab!}\\

\textbf{Modul "`Geoinformatik I"'}
	\begin{itemize}
		\item \textbf{V} "`Einführung in die Geoinformatik"'
		\item \textbf{Ü} "`Einführung in die Geoinformatik"'
	\end{itemize}
\emph{Die Vorlesung (2 SWS), sowie die Übung finden im Wintersemester statt.}\\

\textbf{Modul "`Humangeographie II"'}
	\begin{itemize}
		\item \textbf{V} "`Humangeographie II"'
		\item \textbf{S} "`Humangeographie IIa"'
		\item \textbf{S} "`Humangeographie IIb"'
	\end{itemize}

\textbf{Modul "`Physische Geographie IIa"'}
	\begin{itemize}
		\item \textbf{V} "`Einführung in die Klimatologie"'
		\item \textbf{V} "`Landschaftszonen der Erde"'
		\item \textbf{S} "`Physische Geographie IIa"'
		\item \textbf{S} "`Physische Geographie IIb"'
	\end{itemize}	
	
\emph{Im Bereich der Seminare müssen zwei der vier folgenden Wahlmöglichkeiten belegt werden: Landschaftszonen, Mensch-Umwelt-Beziehung, Klimageographie und Übung Klimatologie.}\\\\
\textbf{Modul "`Einführung in geographische Erhebungs- und Analysetechniken"'}
	\begin{itemize}
		\item \textbf{S} "`Methoden der empirischen Humangeographie"' oder "`Einführung in die Kartenerstellung, -analyse und -interpretation"'
		\item \textbf{Ü} E-Learning-Einheit
	\end{itemize}
\emph{Wer mit dem Gedanken spielt, seine Bachelor-Arbeit in Geographie zu schreiben, sollte sich für das Methoden-Seminar entscheiden, um Kenntnisse für die Durchführung einer empirischen Arbeit zu besitzen.}\\\\
\textbf{Modul "`Regionale Geographie"'}
	\begin{itemize}
		\item \textbf{V} Regionale Geographie (aus dem Vorlesungsverzeichnis wählen)
		\item \textbf{S} Regionale Geographie (aus dem Vorlesungsverzeichnis wählen)
		\item Exkursion (10 Tage)
	\end{itemize}
\emph{Aus den angebotenen Exkursionen ist eine Exkursion zu wählen. Am besten bemüht man sich schon frühzeitig im Studium um einen Exkursionsplatz, da diese immer heiß begehrt sind und es ärgerlich sein könnte, sein Studium nur wegen einer fehlenden Exkursion nicht pünktlich beenden zu können. In der Studienordnung steht, dass eine Exkursion immer passend zu einem Seminar gewählt werden sollte. Leider werden in der Realität nicht genügend Seminare mit zugehöriger Exkursion angeboten, sodass man hier durchaus thematisch unterschiedliche Veranstaltungen belegen kann.}
\\\\
\textbf{Wahlpflichtbereich I (zwei von drei)}\\
\emph{In diesem Bereich müssen von den drei zur Verfügung stehenden Modulen zwei Module gewählt werden.}\\

\textbf{Modul "`Raumplanung/Angewandte Geographie"'}
	\begin{itemize}
		\item \textbf{V} Grundlagen der Raumplanung oder Angewandte Geographie (aus dem Vorlesungsverzeichnis wählen)
		\item \textbf{S} Einführung in die räumliche Planung oder Angewandte Geographie (aus dem Vorlesungsverzeichnis wählen)
	\end{itemize}


\textbf{Modul "`Geoinformatik II"'}
\begin{itemize}
	\item \textbf{V} "`Digitale Kartographie"' und
	\item \textbf{Ü} "`Digitale Kartographie"' oder
	\item \textbf{S} "`Projektseminar Teil 1"' und
	\item \textbf{S} "`Projektseminar Teil 2"'
\end{itemize}

\textbf{Modul "`Physische Geographie III a"'}
	\begin{itemize}
		\item \textbf{V/Ü} "`Bodenkunde"'
		\item \textbf{V/Ü} "`Hydrologie"'
		\item \textbf{V/Ü} "`Vegetationsökologie"'
		\item \textbf{V/Ü} "`Tierökologie"'
	\end{itemize}
\emph{In diesem Modul muss jeweils eine Vorlesung (werden bis auf Bodenkunde nur im Wintersemester angeboten; Bodenkunde dafür nur im Sommersemester) und die dazugehörige Übung (findet nur im Sommersemester statt) belegt werden.}\\\\
\textbf{Wahlpflichtbereich II (eins von zwei)}\\
\emph{In diesem Bereich müssen von den zwei zur Verfügung stehenden Modulen ein Modul gewählt werden, wobei für alle Studenten mit dem Ziel Lehramt das Modul Geographiedidaktik verpflichtend ist.}\\

\textbf{Modul "`Geographiedidaktik"'}
	\begin{itemize}
		\item \textbf{S} Einführung in die Geographiedidaktik
		\item \textbf{S} Einführung in die Unterrichtsplanung
	\end{itemize}

\textbf{Modul "`Wissenschaftskommunikation"'}
\begin{itemize}
	\item \textbf{S} "`Vermittlung geographischer Erkenntnisse"' 
	\item \textbf{S} "`Übung mit Geländetagen (2 Tage)"' 
\end{itemize}

\section*{Studienplanung}
Zur Studienplanung gibt es dann auch noch eine wichtige Anmerkung:
Die Veranstaltungen des ersten und zweiten Semesters sind im Fach Geographie verpflichtend und es besteht keine Wahlmöglichkeit. Die in der Studienordnung ausgegebenen "`Teilnahmevorraussetzungen"' für einzelne Module durch das verpflichtende Bestehen vorheriger Module kann nachgereicht werden. Außerdem solltest du dich von der visuellen Darstellung des Studienverlaufsplans in der Studienordnung nicht irritieren lassen: Eigentlich sind alle Module nach dem zweiten Semester (bei bestandenen Klausuren) zeitlich flexibel zu belegen und nicht starr an Semesterzahlen gebunden; der Studienverlaufsplan stellt hierbei also nur einen in dieser Form machbaren Vorschlag dar, den du -- gerade auch mit Blick auf dein 2. Fach -- ggf. abwandeln kannst.

Es ist also zum Beispiel möglich, das Modul \enquote{Regionale Geographie} schon im dritten Semester zu wählen. Solltet ihr einen Auslandsaufenthalt planen, befasst euch also ausgiebig mit den Möglichkeiten, Module zu belegen. Es sind genügend Freiräume vorhanden, sich die Veranstaltungen im Fach Geographie flexibel zu legen. Zum guten Schluss natürlich der wichtige Hinweis, dass ihr euch bei Fragen immer gerne an uns oder das Front Office wenden könnt! Bei Fragen zu Studienverlaufsplanung, zu Seminarwahlen, zu Klausuren oder zu allem anderen könnt ihr gerne in unseren Präsenzzeiten vorbeischauen, anrufen oder einfach eine Email schreiben. Wir haben selber vor ganz ähnlichen Problemen gestanden. Scheut euch also nicht, Fragen zu stellen und unsere Angebote zu nutzen. Alte Beispielklausuren für viele Veranstaltungen gibt es bei uns in der Fachschaft auch, sodass wir auch hier mit Rat und Tat zur Seite stehen können. (Im Gegenzug freuen wir uns über jede Klausur und jedes Gedächtnisprotokoll, das ihr uns mitbringt!) Ansonsten bleibt uns nur, dir und euch einen guten Studienstart zu wünschen!
  \cleardoublepage
  \chapter*{Bachelor mit Ausrichtung auf fachübergreifende Bildungsarbeit mit Kindern und Jugendlichen ("`Bachelor KiJu"')}
\lohead{\footnotesize{Geographie - Landschaftsökologie - Geoinformatik}}
\rehead{\footnotesize{Geographie - Landschaftsökologie - Geoinformatik}}
Aus organisatorischen Gründen findet ihr an dieser Stelle keine Beschreibung des Studiengangs "`Bachelor KiJu"'. Umfassende und aktuelle Informationen erhaltet ihr von der für euch zuständigen Fachschaft GHR. Bei Unsicherheiten vermitteln wir euch gerne einen entsprechenden Kontakt. Für nähere Informationen:\\
\url{http://www.uni-muenster.de/FachschaftGHR/}
Bei geographischen Fragen helfen wir euch natürlich gern jederzeit weiter!

%\begin{center}
% \includegraphics[scale=0.5]{cartoonD}
%\end{center}

\addtocontents{toc}{\protect\newpage}

  \cleardoublepage
  \chapter{Interessantes}

\section{HowTo--Altklausur}
Altklausuren sind, wie der Name schon sagt, alte Klausuren, die Dozenten in ihren Veranstaltungen schon einmal gestellt haben. Nützlich sind sie deswegen, weil man mit ihnen eine Orientierung hat, welcher Stoff aus der Vorlesung wie intensiv abgefragt werden könnte und weil ihr euer bereits angehäuftes Wissen überprüfen könnt.\\
Da Dozenten sich aber nicht ständig neue Fragen ausdenken möchten, geben sie diese Altklausuren ungern heraus und es ist entsprechend schwierig, an Altklausuren zu kommen. Deswegen hier einige Ideen, wie man trotzdem an die Klausuren kommt - euren Nachfolgern tut ihr dadurch auf jeden Fall einen großen Gefallen, so wie eure Vorgänger euch einen Gefallen getan haben und deren Vorgänger ihren Nachfolgern usw.\\
Dazu gibt es fünf Möglichkeiten:\\
\\
\textbf{1. Nachfragen:}\\
Ausnahmen bestätigen die Regel und dem ein oder anderen Dozenten liegt der Lernerfolg kommender Generationen so sehr am Herzen, dass er doch ein Klausurexemplar herausrückt, wenn ihr ihn nett darum bittet. Es gilt die Devise: Fragen kostet ja nichts!\\
\\
\textbf{2. Kopieren:}\\
Die sauberste und am wenigsten aufwendigste Methode. Ihr habt bei jeder Klausur bis zu 2 Wochen nach Bekanntgabe der Ergebnisse Recht auf Einsicht in eure Klausur, meist an einem festen Termin, auf dessen Bekanntgabe ihr achten solltet! Lässt euch der Dozent mit der Klausur aus dem Raum, heißt es schnell zum Kopierer und die Klausur kopieren. Den Namen und die Matrikelnummer später unkenntlich machen. Das gilt auch für eure Antworten, wenn ihr nicht wollt, dass sie andere lesen. Und dann ab damit in die Fachschaft. Die Original-Klausur müsst ihr natürlich beim Dozenten wieder abgeben. Meistens lassen euch die Dozenten aber nicht aus den Augen, weshalb diese Strategie nur selten klappt.\\
\\
\textbf{3. Fotografieren:}\\
Jedes Handy hat mittlerweile eine eingebaute Kamera, mit der ihr während der Klausureinsicht in einem günstigen Moment unbemerkt die Klausurblätter fotografieren könnt. Wenn der Dozent dazwischen kommt, ist es nicht schlimm, wenn ihr auch nur einen Teil der Klausur abfotografieren konntet. Die Klausur muss aber noch nachbearbeitet werden, weil diese Fotos eine schlechte Qualität haben und ihr vermutlich eure Daten unkenntlich machen wollt.\\
\\
\textbf{4. Abschreiben:}\\
Natürlich nicht von eurem Nebenmann... Aber Unerschrockene können die Klausur während der Klausur auf einen Zettel abschreiben und diesen nachher zur Fachschaft bringen. Bequemere Variante für Harte: mit der Klausur zum Kopierer. Außerdem habt ihr vielleicht Gelegenheit, ein ungenutztes Klausurexemplar aus dem Hörsaal zu entführen, worüber sich die Umwelt freut und keine zusätzlichen Kopierkosten anfallen. Ihr könnt auch am Anfang der Klausur ein Exemplar in eurer Tasche verschwinden lassen und behaupten ihr hättet noch keine Klausur bekommen. Das ist frech, aber auch eine effektive Methode.\\
\\
\textbf{5. Gedächtnis abrufen:}\\
Die nächste Methode ist die aufwendigste, klappt dafür aber immer, egal wie penibel die Dozenten darauf achten, dass die Klausuren auch ja nicht mitgenommen werden. Organisiert euch dazu einfach in einer kleinen oder großen Runde direkt nach der Klausur und schreibt die gestellten Fragen auf. Versucht euch möglichst genau an die Antwortmöglichkeiten bei Multiple"=Choice"=Fragen zu erinnern. Diese Art der Altklausurbeschaffung kann natürlich auch über die Email-Liste des Kurses gehen. Ist dieses Gedächtnisprotokoll erstellt, dann schickt es einfach der Fachschaft. 
Die letzte Methode kann auch euch noch mal helfen wenn ihr euch nicht sicher seid ob ihr bestanden habt. Dann habt ihr schon einige Fragen, mit denen ihr euch auf die Nachschreibklausur vorbereiten könnt. 

\section{Erasmus-Erfahrungsbericht Salamanca 2012/2013}

Salamanca. Eine mittelgroße Stadt in der Hochebene Kastiliens, drum herum nur Einöde, kaum ein Strauch. Im Sommer Hitze, im Winter gar nicht mal so warm. Was bringt einen dazu hier Erasmus zu machen, grade wenn auch Ziele wie Las Palmas de Gran Canaria zur Verfügung stehen? 

Die Antwort will ich versuchen im folgenden Bericht zu geben, doch erst einmal mal wieder alles auf Anfang. Die Planung und Bewerbung lief unkompliziert, bedingt durch die gute Hilfe durch das Erasmus Büro und auch keine besonderen Probleme bei der Uni Salamanca. Da dies schnell unter Dach und Fach war, hieß es nur noch: Flug buchen, (von Weeze nach Madrid), Bus buchen (direkt vom Terminal 1 des Flughafen Madrids mit avanzabus.com nach Salamanca) und sich in einem der zahlreiche Hostels einbuchen. Zwei bis drei Nächte sind dabei ausreichend, denn die Wohnungssuche gestaltet sich dank großem Angebot sehr einfach. Tipps dazu gibt es auch (zu mindestens auf Nachfrage) beim Check-in im Erasmus-Büro der Uni Salamanca. Man sucht am besten nach einer WG- Unterschied zu Deutschland: Man lernt die Mitbewohner meist nicht beim Casting kennen, sondern der Vermieter übernimmt die Auswahl. Auf Nachfrage kann man aber zu mindestens rausfinden ob die Mitbewohner männlich oder weiblich sind, spanisch sind oder auch Erasmus machen. Mit der Wohnung in der Tasche, kann man sich auf die Wahl seiner Kurse konzentrieren- denn eingerichtet sind die Wohnungen schon! Man sollte auf eine gute Küchenausstattung achten, denn die Mensa ist recht teuer. Auch Bettzeug sollte man besser von Zuhause mitbringen, Matratze ist aber meistens vorhanden. 
Zur Auswahl der Kurse: Man hat zwei Wochen Zeit sich alle Kurse anzugucken. Dabei sollte man vor allem auf eine gute Verständlichkeit der Profs achten, denn inhaltlich wird man sowieso nicht an das gewohnte deutsche Niveau herankommen- was aber nicht bedeutet das man weniger Arbeit hat: Hausarbeiten, Exkursionsberichte, Hausaufgaben und dann noch Klausuren werden meist gleichzeitig gefordert. Auch werden in einigen Kursen Exkursionen angeboten, so zum Beispiel im Kurs „Geografia de Turismo“, was natürlich ideal zum Entdecken des Landes ist. Der Kurs ist zwar inhaltlich nicht überragend, aber durch die Exkursionen und die Hausarbeit doch recht empfehlenswert. Nicht wundern sollte man sich wenn nur Erasmusstudenten im Kurs sitzen: Die ist bei Wahlpflichtfächern öfters der Fall- und bringt auch Vorteile, da allem „im selben Boot“ sitzen- und natürlich trotzdem Spanisch gesprochen wird. Zum Einschreiben sollte man genügend Passfotos bereithalten (pro Kurs eins), wobei diese auch einfach auf Papier ausgedruckt werden können 
Der Alltag gestaltet sich einfach in Spanien. Die anfangs noch unsicheren Spanischkenntnisse werden schnell zu mindestens in Alltagssituationen sicher und Supermärkte und Fruterias gibt es an jeder Ecke. Nur die Siesta lässt einen zu Anfang schimpfen. die Läden machen am frühen Nachmittag für 2-3h dicht. Fahrräder können von der Uni für relativ wenig Geld ausgeliehen werden, aber innerhalb der Stadt läuft man meist zu Fuß. 
Ausflüge kann man vor allem in die vielen historischen Städte Kastiliens mit dem Bus machen, auch ein Ausflug mit Mietwagen in andere Städte Spaniens oder Portugals bietet sich an. Wer in die Natur will, wird in der direkten Umgebung enttäuscht, dafür bieten sich aber Exkursionen der Uni an. Die Berggruppe der Uni bietet 1-2 mal im Monat Wanderexkursionen zu je 16€ an, dabei sind Wanderschuhe Pflicht! Die Exkursionen lohnen sich wirklich gut um das Umland und vor allem die 
Berge kennenzulernen, auch trifft man nette Leute. Die Anmeldung läuft über das Sportbüro der Uni, wo man auch Informationen über das weitere Sportangebot findet. Ich habe zum Beispiel noch einen Pádel-Kurs gemacht, eine Mischung zwischen Tennis und Squash. Es ist wirklich ein sehr spaßiges Spiel, nur leider kann ich den Sport in Deutschland nicht fortführen, er ist ein typisch spanisches Phänomen. Schläger muss man sich übrigens selbst besorgen (gilt auch für den Tenniskurs), es gibt aber einen Decathlon (sehr günstiger Sport-Großmarkt) in der Stadt. Für Schwimmer: Wer die städtischen Bäder nutzen will sollte beachten, dass Badekappe, Schwimmbrille und Badelatschen Pflicht sind, also gleich mitbringen. 
Nun will ich abschließend versuchen nochmal direkt auf die anfangs gestellte Frage antworten: Nein, Salamanca ist kein Überwinterungsparadies, es regnet auch mal eine Woche oder es friert bei Nacht- aber dann hat es Anfang Januar auch mal 20 Grad. Nein, in Salamanca kann man nicht surfen oder eben mal aus der Stadt in den Wald oder in die Berge fahren, drum herum ist wirklich nichts, und es ist auch keine Großstadt sondern man kann die Stadt in zwei Tagen sehen und in einem Monat kennenlernen. 
Ja, Salamanca hat eine gewisse Anziehungskraft, durch sein historische Uni und sehr schöne Altstadt, durch sein Katalanisch und durch das muntere Studenten und Erasmusleben, grade bei Nacht. Und man kann unglaublich gut Tapas essen gehen, eine meiner Lieblingsbeschäftigungen. Alles ist zu Fuß zu erreichen und man hat kein Problem die neugewonnen Freunde zu besuchen, man trifft sich sogar immer wieder zufällig auf der Straße. Die Uni ist nicht schlecht organisiert, aber das spanische wissenschaftliche Niveau im Allgemeinen ist zu mindestens in der Geografie echt nicht gut. 
Diese negativen und positiven Seiten hat und am Ende auch der Grund war um zu sagen: Es war schön, aber ein halbes Jahr hat (für mich) gereicht- auch zum Spanisch lernen. 
 \\
\\
Für Fragen rund um das ERASMUS/SOKRATES Programm steht euch das Erasmus-Büro zu Verfügung. Hier bekommt ihr individuelle Beratung und alles Wissenswerte zum Bewerbungsverfahren.\\ \\
\textbf{Pia Eibes und Steffen Marziniak}\\
\textbf{Erasmus-Büro}   \url{erasmus@uni-muenster.de}	Tel.:\,83--33988\\ 

\section{Geld zu verschenken – Stipendienmöglichkeiten für Studis}
Miete, Internet und Handyrechnungen, außerdem Bücher, Exkursionen, Kaffe, Tiefkühlpizza – im Laufe eures Studiums kommen so einige Ausgaben auf euch zu. Wenn dann noch das eine oder andere ,Erfrischungsgetränk' zu sich genommen wird, herrscht auch schnell mal Ebbe im Portemonnaie. Um den Tidenhub wieder in vernünftige Bahnen zu bringen, gibt es ziemlich viele Möglichkeiten: Mama und Papa anpumpen, jobben gehen, Bafög beantragen, etc. Eine ergänzende Alternative ist ein Stipendium. Das klingt soweit super, aber was ist das, wer fördert eigentlich und wie kommt man da ran?
Stipendien können sehr verschieden aussehen. Manche Stiftungen unterstützten ‚nur’ finanziell, andere legen großen Wert auf Teilnahme an Seminaren oder Workshops. Auch die Bemessungsgrundlage der finanziellen Förderung variiert stark. Meist wird ein Betrag von \unit[60-100]{\officialeuro} pro Monat als so genanntes Büchergeld unabhängig von der finanziellen Situation des Stipendiaten ausgezahlt, eine etwaige weitergehende Förderung orientiert sich oftmals an den Bafög-Kriterien.
Die bekanntesten Stipendiengeber sind sicherlich die zwölf ‚großen’ Begabtenförderungswerke:
\begin{itemize}
 \item Cusanuswerk
 \item Deutschland-Stipendium
 \item Evangelisches Studienwerk e. V. Villigst
 \item Friedrich-Ebert-Stiftung
 \item Friedrich-Naumann-Stiftung
 \item Hans-Seidel-Stiftung
 \item Hans-Böckler-Stiftung
 \item Heinrich-Böll-Stiftung
 \item Konrad-Adenauer-Stiftung
 \item Rosa-Luxemburg-Stiftung
 \item Stiftung der Deutschen Wirtschaft
 \item Studienstiftung des Deutschen Volkes 
\end{itemize}
Diese vergeben die größte Anzahl Stipendien, verzeichnen allerdings auch die höchsten Bewerberzahlen. Daher ist häufig eine Bewerbung bei einer der unzähligen kleineren Stiftungen aussichtsreicher. Einige der Stiftungen haben aufgrund ihres geringen Bekanntheitsgrades sogar so wenige Bewerbungen, dass nicht alle Stiftungsmittel ausgezahlt werden. Zum Teil sind die Förderungen regional begrenzt oder nur für bestimmte Fachrichtungen offen, daher kann es eine ganze Weile dauern bis man eine passende Stiftung gefunden hat. Um die Suche ein wenig abzukürzen, hat e-fellows.net, selbst ein Stipendiengeber, eine Datenbank erstellt, in der viele Stiftungen mit einer Kurzbeschreibung verzeichnet sind. Das ganze finde ihr unter folgendem Link:\\ 
\\
\url{http://www.e-fellows.net/show/detail.php/5789}\\
\\
Die genauen Anforderungen, Bewerbungsabläufe und Fristen sind sehr unterschiedlich und es kann eine ganze Weile dauern, bis man alle erforderlichen Unterlagen beieinander hat. Zudem ist eine Bewerbung teilweise nur bis zum 2. oder 3. Fachsemester möglich. Man sollte daher relativ frühzeitig mit der Suche beginnen, um nicht irgendwelche Fristen zu verpassen. Ganz wichtig ist auch, sich nicht abschrecken zu lassen durch die Anforderungen der Stiftungen oder den Aufwand, der mit einer Bewerbung verbunden ist. Sollte am Ende eine Zusage dabei herauskommen, hat sich der Aufwand in jedem Fall gelohnt und immer dran denken: Die Anderen kochen auch nur mit Wasser!

\section{Die Hilfskraft – Studenticus helpissimus}
Wer kennt sie nicht, die ominöse „Hilfskraft“?!? Sie begegnet einem in Seminaren, im Vorzimmer des Professors, an der Kaffeemaschine, in der Bibliothek oder am Kopierer. Aber was macht so eine Hilfskraft eigentlich wirklich und wieso steht sie dann doch manchmal an der Kaffeemaschine anstatt schlaue Bücher mit zu verfassen? Also in erster Linie geht es aus unserer Sicht darum, ein bisschen Geld zu verdienen. Je nachdem, wie viele Stunden pro Woche ableistet werden, kann eine SH (Studentische Hilfskraft) bis zu 350 Euro im Monat dazu verdienen. Das hört sich gut an, aber der Weg zum Geld kann doch recht steinig sein, vor allem sollte man nicht unterschätzen, dass man diese Zeit bei Frei- und Studienzeit abziehen muss. Außer dem Geld bekommt man aber auch spannende Einblicke in den Lehr- und Forschungsbetrieb an der Uni, je nachdem, um was für eine Stelle man sich bemüht hat.\par
Hilfskräfte gibt es an verschiedenen Orten: in den Arbeitsgruppen der Professorinnen und Professoren (Kopieren, Recherchieren, Korrektur"=Lesen, Schreibarbeit, etc.), in der Bibliothek oder im ZDM (Aufsicht, Hilfe, …) usw. Es empfiehlt sich zumeist, bereits ein bisschen Uni-Luft geschnuppert zu haben, also vielleicht nicht gleich im ersten Studienjahr anzufangen. Die Verfügbarkeit offener Stellen ist sehr unterschiedlich. Während beispielsweise die Geoinformatiker recht viele freie Hilfskraftstellen anbieten können, die dann über die Monitore flackern oder am Schwarzen Brett rumhängen, so muss man sich in den meisten Fällen selbst darum kümmern, indem man einfach nachfragt und seine Arbeitskraft anbietet. Am besten dort, wo einen das Thema oder die Arbeit auch ein bisschen interessiert. Und hier lautet die Devise, nicht enttäuscht zu sein, wenn es nicht sofort klappt, aber vielleicht kommt die- oder derjenige ja darauf zurück. Also erstmal eine Anfrage starten und Namen und Adresse da lassen, häufig wird dann später doch was draus. 
\par
Wie man letztendlich eingesetzt wird, hängt dann ganz vom Chef und den anstehenden Arbeiten ab. Tendenziell wachsen die Aufgaben mit dem fachlichen Wissen der Hilfskraft, was aber nicht bedeutet, dass jede Hilfskraft als Tellerwäscher anfängt. Zurzeit bekommt man neun Euro pro Stunde, wobei hier Vorsicht geboten ist, denn bei so manchen Chefs ist „eine Stunde Verdienst“ schnell auch mal zu mehr als zwei Stunden Arbeit geworden. Ob man das dauerhaft mit sich machen lässt, ist einem selbst überlassen. Zumeist sind die Arbeitsbedingungen allerdings wirklich fair und es herrscht eine gute Stimmung. Man darf den Kaffee also auch mittrinken, was die Motivation ihn zu kochen doch deutlich steigert, oder? \par
Letztendlich sind Hilfskraftstellen für all diejenigen zu empfehlen, die gerne etwas tiefer in die Forschungsarbeit oder die Lehre mit einsteigen wollen und die Uni nicht nur als Lernanstalt sehen. Zugegebenermaßen sind nicht alle Stellen anspruchsvoll, aber zum Geldverdienen taugen sie schon. Und wenn man feststellt, man mag die Arbeit, ergibt sich daraus vielleicht auch nach dem Studium noch eine berufliche Option, denn viele Doktoranden haben sich in ihrer Studienzeit bereits als Hilfskraft verdient gemacht. Sobald man übrigens einen fertigen Abschluss hat, kann man bereits als WH (Wissenschaftliche Hilfskraft) eingestellt. Das gibt dann einige Euro mehr pro Stunde. 

\newpage

\section{Ein Tag meines Lebens als Student}
(unbekannter Autor, aber war nach kleiner Bearbeitung sehr passend und durfte deshalb nicht fehlen)\\
\begin{verse}
\textbf{1. Semester}\\ 
%\\
05:30 Der Quarz-Uhr-Timer mit Digitalanzeige gibt ein zaghaftes "`Piep-Piep"' von sich. Bevor sich dieses zu energischem Gezwitscher entwickelt, sofort ausgemacht \linebreak und aus dem Bett gehüpft. Um die Promenade gejoggt, mit einem Besoffenen zusammengestoßen, anschließend eiskalt geduscht. \\
%\\
06:00 Beim Frühstück Greenpeace Magazin auswendig gelernt und Umweltpolitik der Grünen analysiert. Danach kritischer Blick in den Spiegel, Outfit genehmigt.\\ 
%\\
07:00 Zur Uni gehetzt. Hörsaal erreicht. Pech gehabt: erste Reihe schon besetzt. Niederschmetternd. Beschlossen, morgen doch noch eher aufzustehen.\\ 
%\\
07:30 Vorlesung. Keine Disziplin! Einige Kommilitonen lesen Sportteil der Zeitung oder gehen zum Frühstücken. Alles mitgeschrieben; Füller leer, aber über die Witzchen des Dozenten mitgelacht.\\ 
%\\
08:00 Vorlesung. Verdammt! Extra neongrünen Pulli angezogen und trotz eifrigem Fingerschnippens nicht drangekommen.\\
%\\
10:45 Nächste Vorlesung. Nachbar verlässt mit Bemerkung "`Sinnlose Veranstaltung"' den Raum. Habe mich für ihn beim Prof. entschuldigt.\\ 
%\\
12:00 Mensa Stammessen II. Nur unter größten Schwierigkeiten weitergearbeitet, da in der Mensa zu laut.\\ 
%\\
12:45 In Fachschaft gewesen. Skript immer noch nicht fertig. Wollte mich beim Vorgesetzten beschweren. Keinen Termin bekommen. Daran geht die Welt zugrunde.\\ 
%\\
13:00 Fünf Leute aus meiner 0-Gruppe getroffen. Gleich für drei AG's zur Klausurvorbereitung verabredet.\\
%\\
13:30 Dreiviertelstunde im Copyshop gewesen und die Klausuren der letzten 10 Jahre mit Lösungen kopiert. Dann Tutorium: Ältere Semester haben keine Ahnung.\\ 
%\\
15:30 In der Bibliothek mit den anderen gewesen. Durfte aber statt der dringend benötigen 18 Bücher nur vier mitnehmen.\\ 
%\\
16:00 Proseminar. War gut vorbereitet. Hinterher den Assi über seine Irrtümer aufgeklärt.\\ 
%\\
18:30 Anhand einschlägiger Quellen die Promotionsbedingungen eingesehen und erste Kontakte geknüpft.\\ 
%\\
19:45 Abendessen. Verabredung im "`Blauen Haus"' abgesagt. Dafür Vorlesungen der letzten paar Tage nachgearbeitet.\\ 
%\\
23:00 Videoaufzeichnung von Terra Nova angesehen und im Bett noch den Campbell gelesen. Festgestellt, 18\textminus{}Stunden\textminus{}Tag zu kurz. Werde demnächst die Nacht hinzunehmen.                                                                                                                                                                      \end{verse}

\newpage

\begin{verse}
\textbf{ 13. Semester }\\
10.30 Aufgewacht! Kopfschmerz. Übelkeit. Zu deutsch: KATER.\\ 
%\\
10.45 Der linke große Zeh wird Freiwilliger bei der Zimmertemperaturprüfung. (arrgh!) Zeh zurück. Rechts Wand, links kalt; Ich bin gefangen.\\ 
%\\
11.00 Kampf gegen den inneren Schweinehund: Aufstehen oder nicht -- das ist hier die Frage.\\ 
%\\
11.30 Schweinehund schwer angeschlagen, wende Verzögerungstaktik an und schalte Fernseher ein\\ 
%(inzwischen auch schon verkabelt).\\ 
%\\
12.05 Mittagsmagazin beginnt. Originalton Moderator: "Guten Tag liebe Zuschauer Guten Morgen liebe Studenten." - Auf die Provokation hereingefallen und aufgestanden.\\ 
%\\
13.30 In der Cafeteria der Mensa am Aasee beim Skat mein Mittagessen verspielt.\\ 
%\\
14.30 Im Gasolin hereingeschaut. Direkt Geld gepumpt und 'ne Kleinigkeit gegessen: Das Bier schmeckt auch wieder! Kurze Diskussion mit ein paar Leuten über die letzte Entwicklung des Dollar-Kurses und der Weltpolitik.\\ 
%\\
15.45 Kurz in der Bibliothek gewesen. Nur weg hier, total von Erstsemestern überfüllt.\\ 
%\\
16.00 Fünf Minuten im Tech gewesen. Nichts los! Keine Zeitung, keine Flugblätter - nichts wie raus.\\ 
%\\
%17.00 Stammkneipe hat immer noch nicht geöffnet.\\ 
%\\
18.15 Wichtiger Termin zuhause: Star Trek!\\ 
%\\
18:20 Mist! Kein Star Trek! Stattdessen Live-Übertragung von Stöhn-Seles. SAT 1 war auch schon besser...\\ 
%\\
19.10 Komme zu spät zum Date mit der hübschen blonden Erstsemesterin im Barzillus. Immer dieser Stress!\\ 
%\\
01.00 Die Kneipen schließen auch schon immer früher... Umzug in's Amp.\\ 
%\\
04.20 Tagespensum erfüllt. Das Bett lockt.\\ 
%\\
05.35 Auf der Promenade von Erstsemester über'n Haufen gerannt worden. Hat mich gemein beschimpft.\\ 
%\\
06.45 Bude mühevoll erreicht. Insgesamt \unit[15]{\officialeuro}  ausgegeben. Mehr hatte die Kleine nicht dabei.\\ 
%\\
07.05 Ich schlucke schnell noch ein paar Alkas und schalte kurz das Radio ein. Stimme des Sprechers: "Guten Morgen liebe Zuhörer, gute Nacht liebe Studenten."\\
\end{verse}

\newpage
\section{Gängige Abkürzungen im Uni-Dschungel}

\begin{longtable}{p{0.2\textwidth} p{0.7\textwidth}}
  \textbf{AOR} & Akademischer Oberrat (der „Mittelbau“)\\
  \textbf{AR} & Akademischer Rat\\
  \textbf{AStA} & Allgemeiner Studierendenausschuss\\
  \textbf{ASV} & Ausländische Studierendenvertretung\\
  \textbf{AVZ} & Allgemeines Verfügungszentrum\\
  \textbf{Ba / BSc} & Bachelor / Bachelor of Science\\
  \textbf{Bib} & Bibliothek\\
  \textbf{c.t.} & cum tempore = akademisches Viertel (\unit[9]{Uhr} c.t. = \unit[9:15]{Uhr})\\
  \textbf{Dipl.} & Diplom\\
  \textbf{DSW} & Deutsches Studentenwerk\\
  \textbf{ECTS} & European Credit Transfer System\\
  \textbf{EGEA} & European Geographic Association\\
  \textbf{ERASMUS} & Europäisches Austauschprogramm\\
  \textbf{ESG} & Evangelische Studierendengemeinde\\
  \textbf{Exk.} & Exkursion\\
  \textbf{FB} & Fachbereich\\
  \textbf{FBR} & Fachbereichsrat\\
  \textbf{FH} & Fachhochschule\\
  \textbf{FK} & Fachschaftenkonferenz\\
  \textbf{FO} & Frontoffice\\
  \textbf{FP} & Fachprüfung (meistens mündlich, kann schriftlich sein)\\
  \textbf{FS} & Fachschaft \tiny{(oder Fachsemester)}\\ 
  \textbf{FSR} & Fachschaftsrat\\
  \textbf{FSV} & Fachschaftsvertretung\\
  \textbf{fsz} & Freier Zusammenschluss von Studierenden (\underline{der} Dachverband)\\       
  \textbf{GeLaGe} & Geographisch-Landschaftsökologische Gemeinschaftsliste\\ 
  \textbf{GelPr.} & Geländepraktikum\\
  \textbf{Geofs} & Fachschaft Geoinformatik\\
  \textbf{Geogr.} & Geographie oder Geograph/Geographin\\
  \textbf{GeoLök} & Fachschaft Geographie u. Landschaftsökologie\\
  \textbf{GHR} & Grund-, Haupt-, Realschule\\
  \textbf{GPI} & Geologisch-Paläontologisches Institut (Corrensstr.)\\
  \textbf{Gym/Ges} & Gymnasium/ Gesamtschule\\
  \textbf{HD} & Hochschuldozent\\
  \textbf{HISLSF} & siehe LSF\\
  \textbf{Hiwi} & Hilfswissenschaftler; Hilfskraft\\
  \textbf{HoMaLa} & Horstmarer Landweg (Studentenwohnheime)\\
  \textbf{HRG} & Hochschulrahmengesetz\\
  \textbf{HS} & Hörsaal (Heisenbergstr.)\\
  \textbf{HSP} & Hochschulsport\\
  \textbf{IfDG} & Institut für Didaktik der Geographie (Heisenbergstr.)\\
  \textbf{IfG} & Institut für Geographie (Heisenbergstr.)\\
  \textbf{ifgi} & Institut für Geoinformatik (Heisenbergstr.)\\
  \textbf{ILök} & Institut für Landschaftsökologie (Heisenbergstr.)\\
  \textbf{IO} & International Office\\
  \textbf{IVV} & Informationsverarbeitungsversorgung\\
  \textbf{KHG} & Katholische Hochschulgemeinde (am Kardinal-von-Galen-Ring)\\
  \textbf{Kom-Voz} & Kommentiertes Vorlesungsverzeichnis\\
  \textbf{KLSA} & Kommission für Lehre und studentische Angelegenheiten\\
  \textbf{KSHG} & Katholische Studierenden- und Hochschulgemeinde (Frauenstr.)\\
  \textbf{LA} & Lehramt\\
  \textbf{LABG} & Lehrerausbildungsgesetz\\
  \textbf{LN} & Leistungsnachweis, auch "`Schein"' genannt\\
  \textbf{LP} & Leistungspunkt, entspricht 30 Arbeitsstunden, sieht ECTS\\ 
  \textbf{LPO} & Lehramtsprüfungsordnung\\
  \textbf{Lök} & Landschaftsökologie\\
  \textbf{LSF / HISLSF} & Elektronisches Vorlesungsverzeichnis der Uni Münster „Lehre, Studium, Forschung“ (s.o.)\\
  \textbf{Ma / MSc} & Master / Master of Science\\
  \textbf{MAP} & Modulabschlussprüfung\\
  \textbf{N.N.} & Nomen Nominandum (der Name des Dozierenden wird noch bekannt gegeben)\\
  \textbf{Prakt.} & Praktikum\\
  \textbf{Prof.} & Professor\\
  \textbf{Proj.} & Projekt\\
  \textbf{QisPos} & Elektronisches System für alle Bachelor zur Anmeldung und Registrierung (s.o.)\\
  \textbf{RHW} & Rudolf-Harbig-Weg (Studiwohnheime)\\
  \textbf{Sepl} & Seminarplatzvergabe (online)\\
  \textbf{S I/II} & Sekundarstufe I/II\\
  \textbf{SP / StuPa} & Studierendenparlament\\ 
  \textbf{SpSt} & Schulpraktische Studien (Büro Scharnhorststr. 100 / Platz der weißen Rose)\\
  \textbf{StuPa} & siehe SP\\
  \textbf{SoSe} & Sommersemester (bitte \textbf{keine} andere Abkürzung verwenden)\\
  \textbf{s.t.}	& sine tempore = ohne akademisches Viertel pünktlich (\unit[9]{Uhr} s.t. = \unit[9:00]{Uhr})\\
  \textbf{StudLab} & Studenten Labor = Computerraum\\
  \textbf{SWS} & Semesterwochenstunden\\
  \textbf{TN} & Teilnahmenachweis\\
  \textbf{Ü\,/\,ÜB} & Übung\\
  \textbf{UB\,/\,ULB} & Universitäts- und Landesbibliothek (Krummer Timpen)\\
  \textbf{Vorl.\,/\,V} & Vorlesung\\
  \textbf{VV} & Vollversammlung oder Vorlesungsverzeichnis\\
  \textbf{WS} & Wintersemester\\
  \textbf{ZDM/MD} & Zentrum für Digitale Medien und Mediendidaktik\\
  \textbf{ZSB} & Zentrale Studienberatung (Schloßplatz)\\
\end{longtable} 

\newpage
\section{Die studentische Selbstverwaltung}
\textbf{Studierendenschaft}
\begin{itemize}
 \item alle Studierenden der Universität
 \item wählen jedes Wintersemester VertreterInnen ins SP
\end{itemize}

\textbf{Studierendenparlament (SP)}
\begin{itemize}
  \item hat 31 Mitglieder, VertreterInnen verschiedener hochschulpolitischer Listen, werden für ein Jahr gewählt
  \item wählt aus seiner Mitte den Vorsitz für das SP
  \item wählt AStA-Vorstand und die ordentlichen AStA-Referate
  \item bestätigt die autonomen Referate, die von ihrer jeweiligen Interessengruppe gewählt werden
  \item oberstes Beschlussfassendes Gremium der Studierendenschaft
\end{itemize}

\textbf{Allgemeiner Studierenden Ausschuss (AStA)}
\begin{itemize}
  \item Exekutivorgan der Studierendenschaft (wie Merkel mit ihren Bundesministerien)
  \item besteht aus Vorstand und Referaten:
    \begin{enumerate}
      \item ordentliche Referate \\ (Finanz-, Hochschulpolitik-, Wohn-, Sozial"~, Öffentlichkeits-, Ökologie-, Kultur-, Frieden/Internationalismus- Referate)
      \item autonome Referate \\ (Frauen-, Lesben-, Schwulen- und Behindertenreferat)
      \item halbautonome Referate (Fachschaftenkonferenz)
    \end{enumerate}
  \item auskunftspflichtig gegenüber dem SP
\end{itemize}

\textbf{Fachschaft}
\begin{itemize}
  \item alle Studierenden eines Fachbereiches
  \item wählen jedes Sommersemester die Fachschaftsvertretung                                                  
\end{itemize}

\textbf{Fachschaftsvertretung (FSV)}
\begin{itemize}
  \item entspricht strukturell dem SP, auch hier können verschiedene Listen zur Wahl antreten
  \item wählt eigenen Vorsitz
  \item beschließt Satzung der Fachschaft
  \item wählt den FSR
\end{itemize}

\textbf{Fachschaftsrat (FSR)} 
\begin{itemize}
  \item ausführendes Organ der Fachschaft
  \item Koordinierung der studentischen Politik am Fachbereich
  \item Veröffentlichung von Infos aus dem Fachbereich, dem universitären und dem überregionalen Bereich
  \item Engagement in den Gremien des Fachbereichs
  \item ErstsemesterInnenarbeit, Serviceleistungen
  \item schickt VertreterIn zur Fachschaftenkonferenz
\end{itemize}

\textbf{Fachschaftenkonferenz (FK)} 
\begin{itemize}
  \item Organ, in dem VertreterInnen aller Fachschaften der Universität Münster zusammenkommen
  \item Bindeglied zwischen AStA, SP auf der einen und den Fachschaften auf der anderen Seite
  \item wählt FK-ReferentIn
  \item beratende Mitglieder sind VertreterInnen aus Uni-Kommissionen und dem AStA
  \item FK-ReferentInnen sind dem AStA auskunftspflichtig und der FK rechenschaftspflichtig
\end{itemize}

\section{Gremien der Universität}

\textbf{Institutsvorstand}
\begin{itemize}
 \item Professoren eines Institutes gehören ihm automatisch an
 \item auf jeden vierten Professor  kommt ein Mitglied der anderen Statusgruppen
 \item Umsetzung der Prüfungsordnungen in Studienordnungen 
  \item studentischen VertreterInnen werden von den studentischen Mitgliedern im FBR gewählt
  \item wählt aus der Gruppe der Professoren den/ die Direktor/in
\end{itemize}

\textbf{FBR (Fachbereichsrat)}
\begin{itemize}
  \item höchstes Beschlussfassendes Gremium des Fachbereiches
  \item entscheidet in allen Belangen des Fachbereiches: Berufungen, Finanzen, Lehrangebot
  \item Vorsitz hat der/die DekanIn; wird aus der Gruppe der Profs gewählt
  \item Mitglieder werden von jeweiliger Statusgruppe gewählt
  \item hat mehrere Ausschüsse, zum Beispiel:
    \begin{itemize}
	\item \textbf{AFWN} (Ausschuss für Forschung und wissenschaftlichen \linebreak Nachwuchs)
	\item \textbf{ALsA} ( Ausschuss für Lehre und studentische Angelegenheiten)
	\item \textbf{DPA} (Diplomprüfungsausschuss)
	\item Prüfungsausschuss LA SI/II
	\item Berufungskommissionen
	\item Bachelor/Master-Ausschuss
    \end{itemize}
\end{itemize}

\textbf{Dekan}
\begin{itemize}
 \item vollzieht Promotionen, Habilitationen
  \item hat Eilkompetenz in wichtigen Angelegenheiten
  \item lädt zu den Sitzungen des Fachbereichrates ein
  \item hat Rederecht im Senat
  \item aktuelle Dekanatsregelung im Fachbereich Geowissenschaften:
      \begin{itemize}
	\item 1 Dekan
	\item 2 Prodekane: Finanzdekan und Studiendekan (den Job des Studiendekans kann auch ein Student machen!!!)
      \end{itemize}
\end{itemize}

\textbf{Gleichstellungsbeauftragte/r des Fachbereichs}
\begin{itemize}
 \item FBR stellt eine/n Gleichstellungsbeauftragte/n
  \item offiziell: ein/e Gleichstellungsbeauftragte/n und bis zu drei StellvertreterInnen -- meistens aber eher ein/e studentische/r, ein/e nichtwissenschaftliche/r sowie ein/e wissenschaftliche/r Gleichstellungsbeauftragte/r
  \item für alle Belange zuständig, die Gleichstellung innerhalb des Fachbereiches betreffen
  \item Rede- und Teilnahmerecht in allen Gremien des Fachbereiches, soweit es um die Belange der Gleichstellung geht
\end{itemize}

\textbf{Fakultätsrat}
\begin{itemize}
 \item setzt sich aus Mitgliedern der einzelnen Fachbereichrate zusammen
  \item Vorsitz hat FakultätsdekanIn: Gewählt aus Gruppe der Profs
  \item beschäftigt sich mit fächerübergreifenden Angelegenheiten
\end{itemize}

\textbf{Senat}
\begin{itemize}
 \item hat am 11.7.07 bzw. am 7.2.08 nahezu alle bedeutenden Entscheidungskompetenzen an den Hochschulrat abgegeben
\end{itemize}

\textbf{Hochschulrat}
\begin{itemize}
 \item höchstes Beschlussfassendes Gremium der Universität
  \item Entscheidungen über Verteilung der Stellen und Finanzen, Einrichtung/Aufhebung von Fachbereichen
  \item Beschlüsse über Satzungen und Ordnungen der Universität
  \item Anträge an den Konvent
  \item besteht aus 5 uni-externen und 3 uni-internen Mitgliedern
  \item soll einem Aufsichtsrat entsprechen
  \item wählt Hochschulleitung, stimmt über Hochschulentwicklungs- \linebreak und Wirtschaftsplan ab und kann Einrichtung und Schließung von Studiengängen beschließen
  \item tagt nicht öffentlich
  \item weiteres Beispiel für den Wandel an den Hochschulen zu Kaderschmieden der Industrie, da nun eine Mehrheit aus Wirtschaftsvertretern gänzlich ohne Mitsprache des Mittelbaus oder der Studierenden, die wichtigsten Entscheidungen der Uni trifft
\end{itemize}

\textbf{Rektorat}
\begin{itemize}
 \item hat Rederecht im Senat
  \item besteht aus ProrektorInnen, KanzlerIn und RektorIn
  \item bereitet Senatssitzungen vor
  \item dem Senat gegenüber rechenschaftspflichtig
  \item entscheidet in Verwaltungsangelegenheiten
  \item ProrektorInnen haben ständigen Vorsitz in Kommissionen des Senats
  \item RektorIn beruft Senatssitzungen ein und führt dessen Beschlüsse aus
\end{itemize}

\newpage

\section{Unsere Professoren}
\begin{small}
\begin{longtable}{p{0.4\columnwidth} p{0.3\columnwidth} p{0.2\columnwidth}}
  Name & eMail und Telefon & Institut\\ \hline \hline
  Prof.\,Dr.\,Michael Hemmer & \url{michael.hemmer} \newline Tel.:\,83--39365 & Didaktik Geographie\\
  Prof.\,Dr.\,Gabriele Schrüfer & \url{gabriele.schruefer} \newline Tel.:\,83--39349 & Didaktik Geographie\\  \hline
  Prof.\,Dr.\,Ulrike Grabski-Kieron & \url{kieron} \newline Tel.:\,83--33922 & Geographie\\
  Prof.\,Dr.\,Paul Reuber & \url{p.reuber} \newline Tel.:\,83--30035 & Geographie\\
  Prof.\,Dr.\,Gerald Wood & \url{geosek} \newline Tel.:\,83--30026 & Geographie\\ \hline
  Prof.\,Dr.\,Angela Schwering & \url{angela.schwering} \newline Tel.:\,83--33059 & Geoinformatik\\
  Prof.\,Dr.\,Edzer Pebesma & \url{edzer.pebesma} \newline Tel.:\,83--33081 & Geoinformatik\\ 
  Prof.\,Dr.\,Christian Kray & \url{c.kray} \newline Tel.:\,83--33073 &  Geoinformatik \\ \hline
  Prof.\,Dr.\,Christian Blodau & \url{c.blodau} \newline Tel.:\,83--30209& Lök \\ 
  Prof.\,Dr.\newline Tillmann Buttschardt & \url{tillmann.buttschardt} \newline Tel.:\,83--30104 & Lök\\
  Prof.\,Dr.\,Norbert Hölzel & \url{norbert.hoelzel} \newline Tel.:\,83--33994 & Lök\\
  Prof.\,Dr.\,Otto Klemm & \url{otto.klemm} \newline Tel.:\,83--33921 & Lök\\
  Prof.\,Dr.\,Christoph Scherber \url{christoph.scherber} \newline Tel.:\,83–-33996 & Lök\\
  Prof.\,Dr.\,Andreas Schulte & \url{andreas.schulte} \newline \url{@wald-zentrum.de} \newline Tel.:\,83--30121 & Lök\\ 
 \hline
  \multicolumn{3}{l}{Bei den \url{eMail Adressen} ist jeweils} \\
  \multicolumn{3}{l}{``@uni-muenster.de'' zu ergänzen.}\\
\end{longtable}
\end{small}

%\newpage

%\textbf{StudienberaterInnen:}\\ \\
%\begin{tabular}{p{0.3\columnwidth} p{0.7\columnwidth}}
%Geographie & Dr. Christoph Scheuplein \newline \url{christoph.scheuplein}|Tel.:\,83--33925\\
%&\\
%Geographie \newline (2-Fach Bacherlor) & Prof.\,Dr.\,Gerald Woold \newline \url{geosek}|Tel.:\,83--30025\\
%&\\
%Geographie \newline (Lehrämter) & Dipl.\,Geogr.\,Katja Wrenger \newline \url{katja.wrenger}|Tel.:\,83--39364\\
%&\\
%Geoinformatik & Prof.\,Dr.\,Angela Schwering \newline \url{angela.schwering}|Tel.:\,83--33059\\
%&\\
%Lök & Dr. Andreas Vogel \newline \url{voghild}|Tel.:\,83--33698\\ \hline
%\multicolumn{2}{l}{Bei den \url{eMail Adressen} ist jeweils}\\
%\multicolumn{2}{l}{``@uni-muenster.de'' zu ergänzen!!!}\\
%\end{tabular} 


\end{document}
