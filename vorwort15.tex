
%\chapter*{Vorwort}
\addcontentsline{toc}{chapter}{Vorwort}
\thispagestyle{plain}
\lettrine[lines=3,loversize=0.2,slope=-15,lhang=0.2]{V}{or} euch liegt das neue Erstsemester-Infoheft 2015, in dem wir euch aktuelle Informationen zu euren Studiengängen, Modulen, Nebenfächern und vielem mehr bieten. Dieses Info-Heft eurer Fachschaft soll euch nicht nur im ersten Semester weiterhelfen, sondern auch während des Studiums immer wieder eine Nachschlagemöglichkeit bieten. Manche Dinge können sich natürlich mit der Zeit ändern, daher empfiehlt sich regelmäßig der Blick auf die Homepages der Institute und unserer Fachschaften. Bitte vergesst auch nicht, dass die Erstellung eines solchen Info-Heftes, besonders mit so vielen fleißig mithelfenden Menschen und so vielen beteiligten Studiengängen, immer eine Menge Arbeit bedeutet. Es ist wohl nicht zu vermeiden, dass sich der ein oder andere Fehler einschleicht oder eine Formatierung nicht optimal sitzt. Wir hoffen, ihr könnt dennoch etwas mit diesem Heft anfangen und wünschen euch viel Spaß beim Lesen!

Wie sich vielleicht schon herumgesprochen hat, sind die Institute erst vor kurzem in ein neues Gebäude umgezogen. Dies wurde zum Einen durch eine PCB Belastung des alten Gebäudes, zum anderen durch die Vergrößerung der Institute mit der Zeit, fällig. Das insgesamt etwa 30 Millionen Euro teure Gebäude berücksichtigt bewusst ökologische Aspekte wie Energieeffizienz und Nachhaltigkeit und wurde u.a. dafür von der Deutschen Gesellschaft für Nachhaltiges Bauen (DGNB) mit dem Vorzertifikat in Silber ausgezeichnet. Der Neubau bietet auf 6700 Quadratmetern Nutzfläche Platz für die Forschungsstelle für Paläobotanik, den Instituten für Didaktik der Geographie (inkl. neuem Lehr-/Lernatelier), Geographie, Geoinformatik und Landschaftsökologie. Dadurch können alle Studierende unseres Fachbereichs von deutlich verkürzten Wegen und Pendelzeiten profitieren.
%Die eine oder der andere von euch wird vielleicht schon erfahren haben, dass voraussichtlich im Frühjahr 2013 ein lang erwarteter Umzug bevorsteht. Bei der derzeitigen Situation, in der die Institute unseres Fachbereichs quer durch die Stadt verteilt sind, handelt es sich ja nur um eine Übergangslösung. Sie wurde nötig, nachdem man festgestellt hatte, dass ein Teil des nicht mehr ganz jungen Gebäudes an der Robert-Koch-Straße 26 mit polychlorierten Biphenylen (PCB) und Asbest belastet war. Am 14. Juli 2011 wurde jedoch der Grundstein für den Neubau der Geowissenschaften an der Ecke Corrensstraße/Mendelstraße gelegt - in angenehmer Nähe zur Mensa II am Coesfelder Kreuz. Das insgesamt etwa 30 Millionen Euro teure Gebäude berücksichtigt bewusst ökologische Aspekte wie Energieeffizienz und Nachhaltigkeit und wurde u.a. dafür von der Deutschen Gesellschaft für Nachhaltiges Bauen (DGNB) mit dem Vorzertifikat in Silber ausgezeichnet. Der Neubau wird auf 6700 Quadratmetern Nutzfläche neben der Forschungsstelle für Paläobotanik den Instituten für Didaktik der Geographie (inkl. neuem Lehr-/Lernatelier), Geographie, Geoinformatik und Landschaftsökologie ein neues zu Hause bieten. Für euch bedeutet das zwar auch, dass ihr euch nach etwa einem halben Jahr Studium noch einmal umgewöhnen müsst, dafür können dann alle Studierende unseres Fachbereichs von deutlich verkürzten Wegen und Pendelzeiten profitieren.

%Beim Personalbestand gab es bereits im vergangenen Semester einigen frischen Wind, insbesondere im Bereich der Geographie. Dort mussten verhältnismäßig viele Stellen neu besetzt werden, was gewisse Herausforderungen für den Lehrbetrieb mit sich brachte. Mittlerweile hat sich die Lage wieder normalisiert und wir hoffen, dass sich die neuen Mitarbeiter gut an der WWU eingelebt haben und ihr Know-How auch für euch gewinnbringend einsetzen. 

Noch ein letzter Absatz in eigener Sache: Um auch weiterhin erfolgreich zu arbeiten, benötigen wir eure Unterstützung. Wer wir sind und was wir eigentlich machen, findet ihr auf den folgenden Seiten oder könnt es auf unserer Homepage nachlesen. Wir würden uns freuen, wenn sich einige von euch angesprochen fühlen und Lust haben, einfach mal vorbeizukommmen und reinzuschnuppern. Wir sind ein lustiger Haufen, beißen nicht und spätestens bei einem gemütlichen Bierchen werdet ihr feststellen, dass Fachschaft nicht nur Arbeit bedeutet! Also, viel Spaß beim Lesen und vor allem bei eurem Studium!
\bigskip
\newline
Eure Fachschaften GeoLök \& Geoinformatik
