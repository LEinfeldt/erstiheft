\chapter{Zwei-Fach-BA Geographie}
\lohead{\footnotesize{\textbf{2FB Geographie} - Landschaftsökologie - Geoinformatik}}
\rehead{\footnotesize{\textbf{2FB Geographie} - Landschaftsökologie - Geoinformatik}}

\section*{Was ist eigentlich der 2-Fach-Bachelor und wofür gibt es dieses "`Ding"'?}
Seit der Umstellung der Studiengänge von Magister, Diplom und Staatsexamen auf Bachelor und Master in Münster gibt es den Zwei"=Fach"=Bachelor an der Westfälischen Wilhelms-Universität auch im Fachbereich Geowissenschaften. Während der "`einfache"' Bachelor den Diplom-Studiengang ablösen soll, ersetzt der 2-Fach-Bachelor den Magister-Studiengang und den Weg zum Lehramt. Nachdem der Modelversuch abgeschlossen ist, startet nun die zweite Generation des neuen Zwei-Fach-Bachelors. Wenn ihr nicht sicher seid, von welchen Bachelor die Rede ist, kann man dies gut an der unterschiedlichen Schreibweise erkennen: 2-Fach-Bachelor (Abkürzung 2FB) ist der Modelversuch und Zwei-Fach-Bachelor (Abkürzung ZFB) seid ihr. 

Die meisten Studierenden wählen den Zwei-Fach-Bachelor, weil nach dem Abschluss des Bachelorstudiums der Master of Education folgen kann, welcher zum Lehramtsberuf führt. Es bietet sich aber auch die Möglichkeit, nach dem Zwei-Fach-Bachelor eben nicht einen Master of Education zu studieren, sondern sich über ein anderes Master-Studium zu spezialisieren. Ein solcher Bachelor-Abschluss wird "`polyvalent"' genannt, weil den Studierenden mehrere Wege offen stehen, sich weiter auszubilden. In welche Richtung es dabei gehen sollte, bleibt dabei dir überlassen. In den ersten Jahrgängen in Münster waren zum Beispiel einige Personen mit Ziel Journalismus oder Klimaforschung dabei und Geographie als Querschnittsdisziplin lässt sich sicherlich für viele Berufsfelder verwenden.

Ein weiterer Vorteil, den der Zwei-Fach-Bachelor bietet, ist die Möglichkeit, sich erst nach Abschluss des Bachelors zu entscheiden, in welche berufliche Richtung es gehen soll. Wer noch am Zweifeln ist, ob der Lehrerberuf die richtige Entscheidung ist, oder sich eben diese Zukunft noch offen halten möchte, zieht Vorteile aus diesem Studien-System.

\section*{Wie ist der 2-Fach-Bachelor aufgebaut?}
Der Bologna-Prozess, also die Vereinheitlichung der Studien-Systeme in Europa, hat ein Instrument hervorgebracht, mit dem sich das Vergleichen von Leistungen verbessern soll. Mit Umstellung auf die Bachelor- und Masterstudiengänge wurden die sogenannten ECTS-Punkte bzw. Credit Points oder auch Leistungspunkte (LP) eingeführt. ECTS heißt "`European Credits System"' und ein Credit spiegelt (i.d.R.) 30 Stunden Arbeit wider. An der Uni sollte man sich daher daran gewöhnen, dass der Arbeitsaufwand in Leistungspunkten angegeben wird. Eine Studienleistungn für die man 3 LP bekommtn wird also in der Regel weniger umfangreich sein als eine Studienleistungn für die man 5 LP bekommt.

Im Zwei-Fach-Bachelor muss man in den sechs Semestern Regelstudienzeit 180 Punkte erreichen, im Schnitt also 30 Punkte pro Semester. Diese Punkte ergeben sich wiefolgt:
\begin{itemize}
\item 75 Punkte Fach 1
\item 75 Punkte Fach 2
\item 20 Punkte Bildungswissenschaften/Allgemeine Studien
\item 10 Punkte Bachelor-Arbeit
\end{itemize}
Pro Studienfach müsst ihr also 75 Punkte erbringen, diese werden für Veranstaltungen wie Seminare, Vorlesungen, Übungen etc. vergeben und hängen dabei wiederum von der Leistung ab, die man in dieser Veranstaltung erbringt. Referat, Planspiel, Klausur usw. geben jeweils unterschiedliche Punkte. Veranstaltungen werden wiederum in Modulen zusammengefasst und somit in einen thematischen Block gebunden.

Zwei Fächer und jeweils 75 Punkte –- das sind insgesamt 150. Zwanzig weitere Punkte werden in den Bildgungswissenschaften bzw. Allgemeinen Studien vergeben. Das ist ein Bereich, in dem man, wenn man nicht auf Lehramt studiert, "`frei"' wählen und unterschiedlichste Veranstaltungen besuchen kann, an denen man interessiert ist. Das Angebot ist enorm vielfältig und erstreckt sich vom Fremdsprachenerwerb über Rhetorik- und Vermittlungskompetenz bis hin zur Berufsvorbereitung. Leider geben viele Fächer mittlerweile verpflichtend vor, dass eine bestimmte Veranstaltung besucht werden muss. In der Geographie ist das aber nicht so.

Im Bereich der Lehramtsstudiengänge ist der Bereich der Allgemeinen Studien den Bildungswissenschaften zugeordnet. Das Studium der Bildungswissenschaften stellt im Rahmen der Lehrerausbildung einen eigenständigen Teil neben den zu studierenden (Unterrichts-) Fächern dar. Alle Studierende, die im Anschluss an das erfolgreich absolvierte Bachelor-Studium in den Master of Education-Studiengang für das Lehramt an Gymnasien und Gesamtschulen wechseln möchtem, müssen während der Bachelor-Phase auch Bildungswissenschaften studiert haben. Diese setzen sich aus drei Modulen zusammen: Einführung in die Grundfragen von Erziehung, Bildung und Schule (7 LPs: Vorlesung + Seminar), Orientierungspraktikum (6 LPs: Seminar + 120 Stunden Praktikum) und ein Berufsfeldpraktikum (7 LPs: Seminar + 150 Stunden Prakiktum).

Selbstverständlich kannst du auch mehr Kurse belegen, als du später im Rahmen deines Bachelors anrechnen lassen kannst, sofern die Zeit es zulässt (z.B. über Blockkurse in der vorlesungsfreien Zeit). Du solltest dir bei den Allgemeinen Studien jedefalls bewusst machen, dass sie ein gutes und v.a. kostenfreies Angebot sind, bestimmte Kompetenzen zu erwerben, die andernfalls vielleicht zu kurz kommen würden.
Wahlmöglichkeiten bietet das Vorlesungsverzeichnis im Bereich \enquote{Allgemeine Studien} (bzw. \enquote{General Studies}, dieser Begriff wird an der Uni synonym verwendet); vor allem Sprachen bieten sich für diesen Bereich an. Weitere Informationen dazu findest du hier\footnote{\url{www.uni-muenster.de/studium/studienangebot/allgemeinestudien.html} oder \url{http://egora.uni-muenster.de/ew/studieren/bindata/WS1112-1_Studiengangsinfo-21.pdf}}.

Die letzten zehn Punkte erreicht man dann mit seiner Bachelor-Arbeit. Hierbei kannst du dir aussuchen, in welchem deiner beiden Fächer du sie schreiben möchtest.

\section*{Was muss ich in meinem Geographiestudium machen?}
Wie oben beschrieben, müssen im Studienfach Geographie 75 Punkte erbracht werden. Das erreicht man über das Studieren folgender Module:
\begin{itemize}
	\item Humangeographie I
	\item Physische Geographie I
	\item Geoinformatik I
	\item Humangeographie II
	\item Physische Geographie II	
	\item Geographische Erhebungs- und Analysetechniken
	\item Regionale Geographie
	\item Wahlpflichtbereich I, zwei der drei Module:
		\begin{itemize}
			\item Raumplanung/Angewandte Geographie
			\item Geoinformatik II
			\item Physische Geographie III 
		\end{itemize}
	\item Wahlpflichtbereich II, eins der zwei Module:
		\begin{itemize}
			\item Geographiediaktik I (verpflichtend für Lehramtsstudierende)
			\item Wissenschaftskommunikation
		\end{itemize}
\end{itemize}
Im ersten und zweiten Semester müssen die Module Humangeographie I, Physische Geographie I und Geoinformatik I besucht werden. Die genauen Informationen zu allen Modulen und die offizielle Beschreibung des Studiengangs findet man in der Studienordnung. Die Studienordnung ist auf unserer Homepage verlinkt (\url{http://geofs.uni-muenster.de/geoloek/doku.php?id=studieninfos:lehramt}), aber man findet sie natürlich auch auf der Seite der Zentralen Studienberatung (\url{http://zsb.uni-muenster.de/material/m858b_3.htm}).

An dieser Stelle soll nur eine kurze Übersicht mit einigen Anmerkungen zu den einzelnen Modulen erfolgen:
\section{Modulübersicht}
\textbf{Modul "`Humangeographie I"'}
	\begin{itemize}
		\item \textbf{V} "`Einführung in die Humangeographie"'
		\item \textbf{S} "`Humangeographie"'
	\end{itemize}
\emph{Die Vorlesung (4 SWS) findet im Wintersemester statt. Eine Empfehlung unsererseits: Die Klausur sollte man nicht auf die leichte Schulter nehmen und wirklich frühzeitig beginnen, dafür zu lernen. Das Seminar (2 SWS) findet nachfolgend im Sommersemester statt. Hierbei habt ihr die Wahl zwischen verschiedenen Seminarthemen wie z.B. Wirtschaftsgeographie.}\\

\textbf{Modul "`Physische Geographie I"'}
	\begin{itemize}
		\item \textbf{V} "`Einführung in die Physische Geographie"'
		\item \textbf{Ü} "`Physisch-geographische Geländeübung"'
	\end{itemize}
\emph{Die Vorlesung (4 SWS) findet im Wintersemester statt. Die Geländeübung gliedert sich in wenige Vorlesungstermine im Sommersemester und findet sonst als Exkursion an zwei Wochenenden statt. Das Modul schließt mit einer Modulabschlussprüfung (über Vorlesung und Übung) ab!}\\

\textbf{Modul "`Geoinformatik I"'}
	\begin{itemize}
		\item \textbf{V} "`Einführung in die Geoinformatik"'
		\item \textbf{Ü} "`Einführung in die Geoinformatik"'
	\end{itemize}
\emph{Die Vorlesung (2 SWS), sowie die Übung finden im Wintersemester statt.}\\

\textbf{Modul "`Humangeographie II"'}
	\begin{itemize}
		\item \textbf{V} "`Humangeographie II"'
		\item \textbf{S} "`Humangeographie IIa"'
		\item \textbf{S} "`Humangeographie IIb"'
	\end{itemize}

\textbf{Modul "`Physische Geographie IIa"'}
	\begin{itemize}
		\item \textbf{V} "`Einführung in die Klimatologie"'
		\item \textbf{V} "`Landschaftszonen der Erde"'
		\item \textbf{S} "`Physische Geographie IIa"'
		\item \textbf{S} "`Physische Geographie IIb"'
	\end{itemize}	

\emph{Im Bereich der Seminare müssen zwei der vier folgenden Wahlmöglichkeiten belegt werden: Landschaftszonen, Mensch-Umwelt-Beziehung, Klimageographie und Übung Klimatologie. Hierbei ist es wichtig, dass die Seminare im Studienverlaufsplan erst für das vierte Fachsemester vorgesehen sind, jedoch schon im dritten belegt werdemn können. Bitte nehmt diese Möglichkeit wahr, da es sonst unter Umständen dazu kommt, dass ihr im Sommersemester keine Seminarplätze bekommt, weil die Seminare dann überlaufen sind. Sollte dies der Fall sein könnt ihr die Seminare auch noch im fünften Semester parallel zu 'Physische Geographie III' belegen. Das Modul 'Physische Geographie II' ist keine Voraussetzng für das Modul 'Physische Geographie III'.}\\\\
\textbf{Modul "`Einführung in geographische Erhebungs- und Analysetechniken"'}
	\begin{itemize}
		\item \textbf{S} "`Methoden der empirischen Humangeographie"' oder "`Einführung in die Kartenerstellung, -analyse und -interpretation"'
		\item \textbf{Ü} E-Learning-Einheit
	\end{itemize}
\emph{Wer mit dem Gedanken spielt, seine Bachelor-Arbeit in Geographie zu schreiben, sollte sich für das Methoden-Seminar entscheiden, um Kenntnisse für die Durchführung einer empirischen Arbeit zu besitzen.}\\\\
\textbf{Modul "`Regionale Geographie"'}
	\begin{itemize}
		\item \textbf{V} Regionale Geographie (aus dem Vorlesungsverzeichnis wählen)
		\item \textbf{S} Regionale Geographie (aus dem Vorlesungsverzeichnis wählen)
		\item Exkursion (10 Tage)
	\end{itemize}
\emph{Aus den angebotenen Exkursionen ist eine Exkursion zu wählen. Am besten bemüht man sich schon frühzeitig im Studium um einen Exkursionsplatz, da diese immer heiß begehrt sind und es ärgerlich sein könnte, sein Studium nur wegen einer fehlenden Exkursion nicht pünktlich beenden zu können. In der Studienordnung steht, dass eine Exkursion immer passend zu einem Seminar gewählt werden sollte. Leider werden in der Realität nicht genügend Seminare mit zugehöriger Exkursion angeboten, sodass man hier durchaus thematisch unterschiedliche Veranstaltungen belegen kann. Um auf die zehn Exkursionstage zu kommen, bietet sich aus immer an, an Eintagesexkursionen teilzunehmen, die regelmäßig angeboten werden. Die Exkursionen werden auf einem Exkursionspass festgehalten, den ihr im Frontoffice (R. 124) erhaltet.}
\\\\
\textbf{Wahlpflichtbereich I (zwei von drei)}\\
\emph{In diesem Bereich müssen von den drei zur Verfügung stehenden Modulen zwei Module gewählt werden.}\\

\textbf{Modul "`Raumplanung/Angewandte Geographie"'}
	\begin{itemize}
		\item \textbf{V} Grundlagen der Raumplanung oder Angewandte Geographie (aus dem Vorlesungsverzeichnis wählen)
		\item \textbf{S} Einführung in die räumliche Planung oder Angewandte Geographie (aus dem Vorlesungsverzeichnis wählen)
	\end{itemize}


\textbf{Modul "`Geoinformatik II"'}
\begin{itemize}
	\item \textbf{V} "`Digitale Kartographie"' und
	\item \textbf{Ü} "`Digitale Kartographie"' oder
	\item \textbf{S} "`Projektseminar Teil 1"' und
	\item \textbf{S} "`Projektseminar Teil 2"'
\end{itemize}

\textbf{Modul "`Physische Geographie III a"'}
	\begin{itemize}
		\item \textbf{V/Ü} "`Bodenkunde"'
		\item \textbf{V/Ü} "`Hydrologie"'
		\item \textbf{V/Ü} "`Vegetationsökologie"'
		\item \textbf{V/Ü} "`Tierökologie"'
	\end{itemize}
\emph{In diesem Modul muss jeweils eine Vorlesung (werden bis auf Bodenkunde nur im Wintersemester angeboten; Bodenkunde dafür nur im Sommersemester) und die dazugehörige Übung (findet nur im Sommersemester statt) belegt werden.}\\\\
\textbf{Wahlpflichtbereich II (eins von zwei)}\\
\emph{In diesem Bereich muss von den zwei zur Verfügung stehenden Modulen ein Modul gewählt werden, wobei für alle Studierende mit dem Ziel Lehramt das Modul Geographiedidaktik verpflichtend ist.}\\

\textbf{Modul "`Geographiedidaktik"'}
	\begin{itemize}
		\item \textbf{S} Einführung in die Geographiedidaktik
		\item \textbf{S} Einführung in die Unterrichtsplanung
	\end{itemize}

\textbf{Modul "`Wissenschaftskommunikation"'}
\begin{itemize}
	\item \textbf{S} "`Vermittlung geographischer Erkenntnisse"' 
	\item \textbf{S} "`Übung mit Geländetagen (2 Tage)"' 
\end{itemize}

\section*{Studienplanung}
Zur Studienplanung gibt es dann auch noch eine wichtige Anmerkung:
Die Veranstaltungen des ersten und zweiten Semesters sind im Fach Geographie verpflichtend und es besteht keine Wahlmöglichkeit. Die in der Studienordnung ausgegebenen "`Teilnahmevorraussetzungen"' für einzelne Module durch das verpflichtende Bestehen vorheriger Module können nachgereicht werden. Außerdem solltest du dich von der visuellen Darstellung des Studienverlaufsplans in der Studienordnung nicht irritieren lassen: Eigentlich sind alle Module nach dem zweiten Semester (bei bestandenen Klausuren) zeitlich flexibel zu belegen und nicht starr an Semesterzahlen gebunden; der Studienverlaufsplan stellt hierbei also nur einen in dieser Form machbaren Vorschlag dar, den du -- gerade auch mit Blick auf dein 2. Fach -- ggf. abwandeln kannst. Beachte dies vorallem im Modul 'Physische Geographie II' (s.o.).

Es ist also zum Beispiel möglich, das Modul \enquote{Regionale Geographie} schon im dritten Semester zu wählen. Solltet ihr einen Auslandsaufenthalt planen, befasst euch also ausgiebig mit den Möglichkeiten, Module zu belegen. Es sind genügend Freiräume vorhanden, sich die Veranstaltungen im Fach Geographie flexibel zu legen. Zum guten Schluss natürlich der wichtige Hinweis, dass ihr euch bei Fragen immer gerne an uns oder das Front Office wenden könnt! Bei Fragen zu Studienverlaufsplanung, zu Seminarwahlen, zu Klausuren oder zu allem anderen könnt ihr gerne in unseren Präsenzzeiten vorbeischauen, anrufen oder einfach eine E-Mail schreiben. Wir haben selber vor ganz ähnlichen Problemen gestanden. Scheut euch also nicht, Fragen zu stellen und unsere Angebote zu nutzen. Alte Beispielklausuren für viele Veranstaltungen gibt es bei uns in der Fachschaft auch, sodass wir auch hier mit Rat und Tat zur Seite stehen können. (Im Gegenzug freuen wir uns über jede Klausur und jedes Gedächtnisprotokoll, das ihr uns mitbringt!) Ansonsten bleibt uns nur, dir und euch einen guten Studienstart zu wünschen!
